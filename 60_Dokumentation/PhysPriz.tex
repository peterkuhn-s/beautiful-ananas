Bei der Recherche nach physikalischen Prinzipien wurden unterschiedlichse Techniken gefunden. hier ist eine Liste .

v
\begin{multicols}{2}
\textbf{Direktische Konstante}
\begin{itemize}
\item bei 20 MHz über eine Platte
\item zwischen einer Gabel
\item Resonanz in einem Zylinder
\end{itemize}

\textbf{Absorption elektromagnetischer Wellen}
\begin{itemize}
\item GPS
\item Radar
\item IR
\item Satellitenaufnahmen
\item Mikrowellen
\item Neutronen-Scattering
\end{itemize}

\textbf{ akustischer Wellen}
\begin{itemize}
\item Absorption Ultraschall
\item Absorption normaler Schall
\item Lamb-Welle
\item {Emission akustischer Wellen
\end{itemize}


\textbf{Elektrische Eigenschaften}
\begin{itemize}
\item ohmscher Widerstand
\end{itemize}

\textbf{Mechanische Eigenschaften}
\begin{itemize}
\item Scherkräfte
\item Dichte
\item Eigenschwingungen
\item Vibrationsübertragung
\item Eindrückwiderstand mit Vibration
\item Viskosität
\item Vibrationsbohrer
\end{itemize}


\textbf{Optische Eigenschaften}
\begin{itemize}
\item Reflexion
\item Refraktion
\item Polarimetrie
\end{itemize}

\columnbreak
\textbf{Thermische Eigenschaften}
\begin{itemize}
\item Schmelzenergie mit DSC
\item mit heissem Wasser
\item mit kalter Flüssigkeit
\item Heizung (elektrisch, Mikrowelle)
\item Taupunktspiegel
\item Leitfähigkeit von warme
\end{itemize}

\textbf{Separation}
\begin{itemize}
\item Zentrifuge
\item Quetschen
\item Absaugen
\end{itemize}

\textbf{Kapillarkräfte}
\begin{itemize}
\item 5559 Water Indicator Tape
\item Staub/Flüssigkeit beim Ausbreiten im Schnee beobachten (optisch, fluoreszent, elektrisch)
\item Event-Kamera
\item Oberflächenspannung
\end{itemize}

\textbf{Sonstiges}
\begin{itemize}
\item optische Beurteilung
\item Luftfeuchtigkeit verändern
\item Luftwiderstand
\item MRI
\item Nuclear Magnetic Resonance
\item Raman-Spektroskopie
\item Infrarot-Spektroskopie
\item Neutronen-Scattering
\end{itemize}
\end{multicols}
