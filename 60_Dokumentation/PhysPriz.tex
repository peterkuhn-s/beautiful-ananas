Bei der Recherche nach physikalischen Prinzipien zur Messung des LWC wurden unterschiedliche, derzeitige Techniken gefunden. Hier ist eine Liste:

%\begin{multicols}{2}
\textbf{Absorption elektromagnetischer Wellen}
\begin{itemize}
\item GPS \cite{Koch.2019}, \cite{Koch.2014}
\item Radar \cite{Bonnell.2021}, \cite{ilmsens-short-range-radar}
\item IR \cite{Donahue.2022}
\item Satellitenaufnahmen \cite{Tsang.2022}

\end{itemize}

\textbf{Dielektrische Konstante}
\begin{itemize}
\item bei 20 MHz über eine Platte \cite{slf}
\item zwischen einer Gabel \cite{fork}
\item Resonanz in einem Zylinder \cite{a2photonicsensors}, \cite{nasa-snowex-2020}
\item Sonstige Anordnungen aus der Agrikultur \cite{Mavrovic.2020}, \cite{PerezDiaz.2017}
\end{itemize}

\textbf{Akustische Wellen}
\begin{itemize}
\item Absorption normaler Schall \cite{Kinar.2007}
\end{itemize}


\textbf{Elektrische Eigenschaften}
\begin{itemize}
\item ohmscher Widerstand \cite{Abdelaal.2022}
\end{itemize}

\textbf{Mechanische Eigenschaften}
\begin{itemize}
\item Scherkräfte \cite{Hao.2021}, \cite{jstage-snow-density}
\item Dichte \cite{nasa-snow-density}

\end{itemize}

%\columnbreak
\textbf{Thermische Eigenschaften}
\begin{itemize}
\item Schmelzenergie mit DSC \cite{mt-density-meter}
\item mit heissem Wasser \cite{Fasani.2023}
\item mit kalter Flüssigkeit

\end{itemize}

\textbf{Kapillarkräfte}
\begin{itemize}

\item Oberflächenspannung \cite{AlamShibly.2017}
\end{itemize}

\textbf{Sonstiges}
\begin{itemize}
\item optische Beurteilung \cite{miro}
\item Luftfeuchtigkeit verändern \cite{joule-thomson-wiki}, \cite{sensirion-sht4xa-sensors}
\item Luftwiderstand
\item MRI \cite{Adachi.2020}, \cite{Nowogrodzki.2018}, \cite{Yamaguchi.2023}
\item Raman-Spektroskopie \cite{Reichardt.2022}
\item Neutronen-Scattering \cite{Lombardo.2023}
\end{itemize}
%\end{multicols}


Im Folgenden sind potenzielle Techniken aufgeführt, die zur Messung des LWC im Schnee verwendet werden könnten, aber zu denen keine spezifischen Veröffentlichungen gefunden wurden.

\textbf{Absorption elektromagnetischer Wellen}
\begin{itemize}
\item Mikrowellen
\end{itemize}



\textbf{Akustische Wellen}
\begin{itemize}
\item Absorption Ultraschall
\item Lamb-Welle \cite{lamb}
\item Emission akustischer Wellen
\end{itemize}


\textbf{Mechanische Eigenschaften}
\begin{itemize}
\item Eigenschwingungen
\item Vibrationsübertragung
\item Eindrückwiderstand mit Vibration
\item Viskosität
\item Vibrationsbohrer
\end{itemize}


\textbf{Optische Eigenschaften}
\begin{itemize}
\item Reflexion
\item Refraktion
\item Polarimetrie
\end{itemize}

%\columnbreak
\textbf{Thermische Eigenschaften}
\begin{itemize}
\item mit kalter Flüssigkeit
\item Heizung (elektrisch, Mikrowelle)
\item Taupunktspiegel
\item Leitfähigkeit von Wärme
\end{itemize}

\textbf{Separation}
\begin{itemize}
\item Zentrifuge
\item Quetschen
\item Absaugen
\end{itemize}

\textbf{Kapillarkräfte}
\begin{itemize}
\item Water Indicator Tape
\item Staub/Flüssigkeit beim Ausbreiten im Schnee beobachten (optisch, fluoreszent, elektrisch, Event-Kamera) \cite{kennedylabs-download}
\item Oberflächenspannung \cite{AlamShibly.2017}
\end{itemize}

\textbf{Sonstiges}
\begin{itemize}
\item optische Beurteilung \cite{miro}
\item Luftfeuchtigkeit verändern \cite{joule-thomson-wiki}, \cite{sensirion-sht4xa-sensors}
\item Luftwiderstand
\item Infrarot-Spektroskopie
\end{itemize}
%\end{multicols}
