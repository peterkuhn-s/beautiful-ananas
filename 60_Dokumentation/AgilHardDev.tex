die planung der Arbeit wird mit einem agilen Kanban board gemacht. Ziel ist auch ein agiles hardware development. das heist es wird versucht möglichst schnell zu einem Funktionsmuster zu kommen, und daraus für die nächste Iteration zu lernen.

Um schnell zu arbeiten wurde möglichst viel CAD (Cardboard aiden design) und möglichst wenig CAD (Computer adided design) benutzt. Für die funktiosmuster wurde in der rheinenfolge die materialen ausgesucht:

bestehendes Objekt benutzten und modifizieren
Karton
IR - laser cutter mit Sperrholz
3D Druck, in Fdm
teile bestellen
selber fertigen (manuell drehen, fräsen, töpfern usw.)
extern fertigen lassen.

ein endprodukt wird extern gefertigt werden müssen um die wertigkeit des Produkts an den Benutzter zu vermitteln. Die Seriegrösse wird klein, da es wahrscheinlich kein Kosumerprodukt werden wird. sonder ein forschungsinstrument.
