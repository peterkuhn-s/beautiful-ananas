

Die Planung der Arbeit wird mit einem agilen Kanban-Board gemacht. Ziel ist auch ein agiles Hardware-Development. Das heisst, es wird versucht, möglichst schnell zu einem Funktionsmuster zu kommen und daraus für die nächste Iteration zu lernen.

Um diese schnelle Arbeitsweise zu ermöglichen, habe ich folgende Reihenfolge zur Auswahl der Fertigungstechniken erstellt:

\begin{enumerate}
\item Bestehendes Objekt benutzen und modifizieren
\item Karton
\item IR-Lasercutter mit Sperrholz
\item 3D-Druck in FDM
\item Teile bestellen
\item Selber fertigen (manuell drehen, fräsen, töpfern usw.)
\item Extern fertigen lassen
\end{enumerate}

Ein Endprodukt wird extern gefertigt werden müssen, um die Wertigkeit des Produkts an den Benutzer zu vermitteln. Die Seriengröße wird klein sein, da es wahrscheinlich kein Konsumerprodukt werden wird, sondern ein Forschungsinstrument.
