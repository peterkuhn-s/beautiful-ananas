die planung der Arbeit wird mit einem agilen Kanban board gemacht. Ziel ist auch ein agiles hardware development. das heist es wird versucht möglichst schnell zu einem Funktionsmuster zu kommen, und daraus für die nächste Iteration zu lernen.

Um diese schnelle arbeitsweise zu ermoglichen habe ich folgende reinenfolge zur auswahl der fertigungstechniken erstellt.

\begin{enumerate}
\item bestehendes Objekt benutzten und modifizieren
\item Karton
\item IR - laser cutter mit Sperrholz
\item 3D Druck, in Fdm
\item teile bestellen
\item selber fertigen (manuell drehen, fräsen, töpfern usw.)
\item extern fertigen lassen.
\end{enumerate}

ein endprodukt wird extern gefertigt werden müssen um die wertigkeit des Produkts an den Benutzter zu vermitteln. Die Seriegrösse wird klein, da es wahrscheinlich kein Kosumerprodukt werden wird. sonder ein forschungsinstrument.
