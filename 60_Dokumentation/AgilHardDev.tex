

Die Planung der Arbeit wird mit einem agilen Kanban-Board durchgeführt. Ein Ziel der agilen Hardware Entwicklung ist möglichst schnell zu einem Funktionsmuster zu kommen und daraus für die nächste Iteration zu lernen.

Um diese schnelle Arbeitsweise zu ermöglichen, wurde folgende Priorisierung der Fertigungstechniken erstellt:

\begin{enumerate}
\item Bestehendes Objekt benutzen und modifizieren
\item Von Hand ausschneiden - Verwendung von Karton
\item IR-Lasercutter - Verwedung von Sperrhozl
\item 3D-Druck in FDM
\item Einkaufsteile kaufen
\item Weitere eigene Fertigungsweisen (manuell drehen, fräsen, töpfern usw.)
\item Extern fertigen lassen
\end{enumerate}

Ein Endprodukt wird extern gefertigt werden müssen, um die Wertigkeit des Produkts an den Benutzer zu vermitteln. Die Seriengrösse ist je nach Endprodukt dazugehöriger User Story sehr unterschiedlich.
