\textb{erster feldversuch}
\label{ErstFeldVer}
\caption{Wärmebildaufnahme der gekühlten Schneestell. }


\caption{Messstandort in Davos, unter dem Regenschrim ist das Tape gelagert} 

\caption{Ein bild von 'normalem' schnee}


\textb{zweiter Feldversuch}
\label{ZweiFeldVer}


erster feldversuch Ziel: testen des ablaufs der tap, mit drei verschiedenen LWC um den LWC zu beeinflussen, zum einen Wasser über den schnee ausschütten,
dann messen.

zum anderen mit Kältespray den Schnee einfrieren, dann sollte der LWC sehr tief sein. Wärmebildaufnahme der gekühlten Schneestelle.
Der Schnee hat sich nur sehr langsam wieder aufgewärmt, da der Schnee sehr gut isoliert ist.

durchgeführt am 2024-04-11 in davos, einige meter hoch auf der schatten talseite. Messstandort in Davos, unter dem Regenschirm ist das Tape gelagert Ein bild von ’normalem’ schnee schlussfolgerungen: schneedreieck funktioniert bei inhomogenem schnee mit eis nicht. handlich aufwendig, belichtung einseitig, gewichte kommen auf der schablone sich in den weg, zweiter Feldversuch Ziel: neues design. variabler (höherer) anpressdruck, Vergleich mit denothmeter Ich bin sehr zufrieden mit diesen Ergebnissen. Die Technologie des 5999 water indicator tape hat einen TRL von 9 um Qualitätssicherung zu machen. Um den LWC von Schnee zu messen, befand sich das 5999 am Anfang dieser Arbeit bei TRL 1, mit diesen Versuchen hat das 5999 den TRL 5 erreicht.
