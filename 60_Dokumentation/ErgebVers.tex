\textb{erster feldversuch}
\label{ErstFeldVer}
Ziel: testen des ablaufs der tap, mit drei verschiedenen LWC

um den LWC zu beeinflussen, zum einem Wasser über den schnee ausschütten, dann messen.

zum anderen mit Kältespray den schnee einfrieren, dann soll der LWC sehr tief sein. \caption{Wärmebildaufnahme der gekühlten Schneestell. }

der schnee hat sich nur sehr langsam wieder aufgewärmt, da schnee sehr gut isoliert.

durchgeführt am 2024-04-11 in davos, einige meter hoch auf der schatten tal seite.
\caption{Messstandort in Davos, unter dem regenschrim ist das Tape gelagert} 

\caption{Ein bild von 'normalem' schnee}

schlussfolgerungen: schneedreieck funktioniert bei inhomogenem schnee mit eis nicht.
handlich aufwendig, belichtung einseitig, gewichte kommen auf der schablone sich in den weg,

\textb{zweiter Feldversuch}
\label{ZweiFeldVer}
Ziel: neues design. variabler (höherer) anpressdruck, vergleich mit denothmeter
