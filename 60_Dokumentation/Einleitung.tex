Gleitschneelawinen gefährden Menschenleben und sind bisher schwer vorhersagbar. Durch die Klimaerwärmung werden sie häufiger auftreten. Ein wichtiger Indikator für die Bildung dieser Lawinen ist der Anteil von flüssigem Wasser im Schnee, der Anteil wird im weiteren Text als Liquid Water Content (LWC) genannt.

Ziel dieser Arbeit ist, einen Sensor zu entwickeln, der den LWC von Schnee misst, und damit die Lawinenvorhersage verbessert.


Das flüssige Wasser im Schnee ist ein entscheidender Parameter um das Verhalten des Schnees an einem Lawinen gefährdeten Hang  vorherzusagen. 

Die heute eingesetzten Messgeräte nutzen unterschiedliche Ansätze. Diese Methoden haben unterschiedliche Nachteile, zum Beispiel, dass sie zwei Arbeitsschritte brauchen, um den LWC als das Verhältnis von flüssigem Wasser zum vorhandenen Schnee  messen. 

In der folgenden Arbeit wird kurz auf die Eigenschaften von Schnee und die Gefährdung durch Lawinen eingegangen. Nach einer Recherche zu den physikalischen Prinzipien zur Messung des LWC werden User Stories für das Produkt entwickelt. Nach dieser Klärung wird die Arbeit in drei weiteren Schritten fortgesetzt.

\begin{enumerate}
\item Vorstudie: Unterschiedliche physikalische Prinzipien zur Messung des LWC werden erst theoretisch und dann praktisch miteinander verglichen.

\item Bau der Funktionsmuster: Hier wird ein vielversprechendes physikalisches Prinzip ausgewählt und es werden Funktionsmuster gebaut. Dieser Teil wird mittels Agiler Hardware Entwicklung einem Kanban Board geplant.

\item Dokumentation der Produktentwicklung.
\end{enumerate}

Es wurden verschiedene theoretische Ansätze, wie zum Beispiel User Storys, der Produktentwicklung im Verlauf der Bachelorarbeit eingesetzt.

Um den Sensor herzustellen, wurden entsprechend der agilen Hardware Entwicklung möglichst rasch Iterationen von Sensoren hergestellt, getestet, angepasst und erneut getestet.


\iffalse


Ziel dieser Arbeit ist, einen Sensor zu entwickeln, der flüssiges Wasser im Schnee misst.
ziel dieser Arbeit ist die entwicklung eines innovativen sensors um die scheefeuchtigkeit zu messen.

Die schneefeuchtigkeit ist ein entscheidenen Parameter um Gleitschneelawinen abzuschetzten. seit 40 jahren ist wird Thema beforscht. Es gibt verschiedenste Techniken um den Schaum aus Eis, Wasser und Luft zu messen. heutige Produkte konnen den LWC messen, haben aber verschiedene schwerwigeende nachteile.

um dieses Produktentwicklung an zu gehen werden verschiedene techniken der Produktentwicklung eingesetzt. um ein sensor zu erreichen der einsatztfahig ist, wurde nach aglier hardware entwicklung moglichst schnell Itterationen von sensoren entwickelt.

\fi
