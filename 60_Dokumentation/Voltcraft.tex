die gaphit sonden, zwischen denen die spanung aufgebaut und der wiederstand gemessen wird sind im messkopf zu gut geschützt. daher kann keine Messung gemacht werden wenn die Probe in schnee gedrückt wird.

Mögliche lösung: Verlängerung der Graphit proben

mit stahlplatten

Verbindung des Graphits mit der Platte: kleben oder konstant drückne oder verschrauben.

in gaphit spahnend zu arbeiten ist anspruchsvoll und dreckig.
konstant drücknen ist fehleranfällig
Kleben: herstellen von leitfähigem Klebstoff:

test graphitpluver: 66 \% gewichtsprozent Graphitpulver, 33 \% Ergo 7410 Epoxy Klebstoff

test Aluminiumpulver: 66 \% gewichtsprozent Aluminiumpulver, 33 \% Ergo 7410 Epoxy Klebstoff


Ergebniss: nach 24 h, sodass der ergo 7410 aushärten konnte.
Alle Klebestellen sind angeschliffen worden als oberflächenvorbereitung

Wiederstand zwischen Punkt A B 2.6 \ohm

wiederstand ziwschen Punkt A C 0.2 \ohm

Wiederstand zwischen Punkt A D keine verbindung

Mechanische stabilität von Test Aluminiumpluver nicht so gut

Ist es möglich auf die stahlplatte zu verzichten und die Verlängerung mit der Graphit Epoxy mischung zu machen?

zwisched die beiden grafit stäbe ist eine PAAM Platte geklebt. alle offenen stellen des Epoxy/grafits ist mit reinem epoxy überzogen um kriechspannungen durch wasser zu verhindern.

Arbeitsschutzt, erklären

Schnee ist wasser das vom Boden verdampft, sich dann in der Atmosspäre an einem Nukleus kondesiert oder resubliemiert und dann auf den Boden zurück fällt.

Im Alltag weiss man, dass man mit den Harrfön nicht in die Dusche gehen darf, da Wasser elektrisch leitend ist. Diese Schlussfolgerung ist nicht sehr prezise. denn reines Wasser ist nicht leitend, sonder die  Ione (Salze) die im 'normalen' Wasser gelöst sind. Auf sehr geringem Niveu ist auch reines Wasser leitend, da sich spontan  1*10 ^ 7 M  Hydroniumionen (H_3 O^+) bilden und den pH Wert 7 bilden.

Die Hypotese ist, dass sowohl die Verunreinigungen durch die Nuklei und die Hydroniuminonen genügend leitfägkeit bilden um einen Messwert im \mu S (Siemens = 1/\Ohm) Bereich zu messen.

Im Feldversuch konntekeine Leitfähigkeit gemessen werden.

EIne erweiterung dieser Messung ist, einen stoff zum schnee dazu zu geben, der gut leitfähig ist. dann wird der Versuchsaufbaumehr in die richtung \ref{} wo die Ausbreitung eines Stoffes im Schnee beobachtet wird. hier wäre diese beobachtung dann über die Leitfähigkeit und nicht wie in \ref{} optisch.
