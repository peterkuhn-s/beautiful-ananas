\iffalse
edie gaphit sonden, zwischen denen die spanung aufgebaut und der wiederstand gemessen wird sind im messkopf zu gut geschützt. daher kann keine Messung gemacht werden wenn die Probe in schnee gedrückt wird.

Mögliche lösung: Verlängerung der Graphit proben

mit stahlplatten

Verbindung des Graphits mit der Platte: kleben oder konstant drückne oder verschrauben.

in gaphit spahnend zu arbeiten ist anspruchsvoll und dreckig.
konstant drücknen ist fehleranfällig
Kleben: herstellen von leitfähigem Klebstoff:

test graphitpluver: 66 \% gewichtsprozent Graphitpulver, 33 \% Ergo 7410 Epoxy Klebstoff

test Aluminiumpulver: 66 \% gewichtsprozent Aluminiumpulver, 33 \% Ergo 7410 Epoxy Klebstoff


Ergebniss: nach 24 h, sodass der ergo 7410 aushärten konnte.
Alle Klebestellen sind angeschliffen worden als oberflächenvorbereitung

Wiederstand zwischen Punkt A B 2.6 \ohm

wiederstand ziwschen Punkt A C 0.2 \ohm

Wiederstand zwischen Punkt A D keine verbindung

Mechanische stabilität von Test Aluminiumpluver nicht so gut

Ist es möglich auf die stahlplatte zu verzichten und die Verlängerung mit der Graphit Epoxy mischung zu machen?

zwisched die beiden grafit stäbe ist eine PAAM Platte geklebt. alle offenen stellen des Epoxy/grafits ist mit reinem epoxy überzogen um kriechspannungen durch wasser zu verhindern.

Arbeitsschutzt, erklären

Schnee ist wasser das vom Boden verdampft, sich dann in der Atmosspäre an einem Nukleus kondesiert oder resubliemiert und dann auf den Boden zurück fällt.

Im Alltag weiss man, dass man mit den Harrfön nicht in die Dusche gehen darf, da Wasser elektrisch leitend ist. Diese Schlussfolgerung ist nicht sehr prezise. denn reines Wasser ist nicht leitend, sonder die  Ione (Salze) die im 'normalen' Wasser gelöst sind. Auf sehr geringem Niveu ist auch reines Wasser leitend, da sich spontan  1*10 ^ 7 M  Hydroniumionen (H_3 O^+) bilden und den pH Wert 7 bilden.

Die Hypotese ist, dass sowohl die Verunreinigungen durch die Nuklei und die Hydroniuminonen genügend leitfägkeit bilden um einen Messwert im \mu S (Siemens = 1/\Ohm) Bereich zu messen.

Im Feldversuch konntekeine Leitfähigkeit gemessen werden.

EIne erweiterung dieser Messung ist, einen stoff zum schnee dazu zu geben, der gut leitfähig ist. dann wird der Versuchsaufbaumehr in die richtung \ref{} wo die Ausbreitung eines Stoffes im Schnee beobachtet wird. hier wäre diese beobachtung dann über die Leitfähigkeit und nicht wie in \ref{} optisch.

Die Graphitsonden, zwischen denen die Spannung aufgebaut und der Widerstand gemessen wird, sind im Messkopf zu gut geschützt. Daher kann keine Messung durchgeführt werden, wenn die Probe in den Schnee gedrückt wird.

\fi

\subsection{Mögliche Lösung: Verlängerung der Graphitsonden}

Um dieses Problem zu beheben, könnte eine Verlängerung der Graphitsonden mit Stahlplatten in Betracht gezogen werden.

\subsection{Verbindung des Graphits mit der Stahlplatte}

Es gibt verschiedene Methoden, um die Graphitsonden mit den Stahlplatten zu verbinden:

\begin{itemize}
    \item \textbf{Kleben:} Eine Möglichkeit besteht darin, die Graphitsonden mit leitfähigem Klebstoff an die Stahlplatten zu kleben. Dies erfordert die Herstellung oder den Kauf von geeignetem leitfähigem Klebstoff.
    \item \textbf{Konstant drücken:} Eine andere Möglichkeit wäre, die Graphitsonden und die Stahlplatten konstant gegeneinander zu drücken. Diese Methode ist jedoch fehleranfällig und könnte im Feld unpraktisch sein.
    \item \textbf{Verschrauben:} Schließlich könnten die Graphitsonden und die Stahlplatten miteinander verschraubt werden, um eine stabile und zuverlässige Verbindung zu gewährleisten.
\end{itemize}

\subsection{Herausforderungen}

\begin{itemize}
    \item \textbf{Bearbeitung von Graphit:} Die Bearbeitung von Graphit ist anspruchsvoll und erzeugt viel Schmutz, was die Arbeitsumgebung belasten kann.
    \item \textbf{Fehleranfälligkeit beim Drücken:} Die Methode des konstanten Drückens ist anfällig für Fehler und könnte die Genauigkeit der Messungen beeinträchtigen.
\end{itemize}

\subsection{Test des Graphitpulvers und Aluminiumpulvers}

\begin{itemize}
    \item \textbf{Graphitpulver:} 66 \% Gewichtsprozent Graphitpulver, 33 \% Ergo 7410 Epoxy Klebstoff
    \item \textbf{Aluminiumpulver:} 66 \% Gewichtsprozent Aluminiumpulver, 33 \% Ergo 7410 Epoxy Klebstoff
\end{itemize}

\subsubsection{Ergebnisse}

Nach 24 Stunden, sodass der Ergo 7410 aushärten konnte, wurden die Klebestellen als Oberflächenvorbereitung angeschliffen. Die Messergebnisse sind wie folgt:

\begin{itemize}
    \item Widerstand zwischen Punkt A und B: \SI{2.6}{\ohm}
    \item Widerstand zwischen Punkt A und C: \SI{0.2}{\ohm}
    \item Widerstand zwischen Punkt A und D: keine Verbindung
\end{itemize}

Die mechanische Stabilität des Tests mit Aluminiumpulver war nicht so gut.

\subsection{Möglichkeit der direkten Verwendung von Graphit-Epoxy-Mischung}

Es stellt sich die Frage, ob auf die Stahlplatte verzichtet und die Verlängerung direkt mit der Graphit-Epoxy-Mischung durchgeführt werden kann.

Zwischen den beiden Graphitstäben ist eine PAAM-Platte geklebt. Alle offenen Stellen des Epoxy/Graphits sind mit reinem Epoxy überzogen, um Kriechspannungen durch Wasser zu verhindern.

\section{Arbeitsschutz und Schnee}

Schnee ist Wasser, das vom Boden verdampft, sich dann in der Atmosphäre an einem Nukleus kondensiert oder resublimiert und dann auf den Boden zurückfällt.

Im Alltag weiß man, dass man mit dem Haartrockner nicht in die Dusche gehen darf, da Wasser elektrisch leitend ist. Diese Schlussfolgerung ist jedoch nicht sehr präzise, denn reines Wasser ist nicht leitend, sondern die Ionen (Salze), die im 'normalen' Wasser gelöst sind. Auf sehr geringem Niveau ist auch reines Wasser leitend, da sich spontan \SI{1e-7}{\molar} Hydroniumionen (\ce{H3O+}) bilden und den pH-Wert 7 ergeben.

Die Hypothese ist, dass sowohl die Verunreinigungen durch die Nuklei als auch die Hydroniumionen genügend Leitfähigkeit bilden, um einen Messwert im \si{\micro\siemens} (Siemens = \si{1/\ohm}) Bereich zu messen.

\subsection{Feldversuch}

Im Feldversuch konnte keine Leitfähigkeit gemessen werden.

Eine Erweiterung dieser Messung wäre, dem Schnee einen Stoff zuzugeben, der gut leitfähig ist. Dann wird der Versuchsaufbau mehr in die Richtung \ref{} gehen, wo die Ausbreitung eines Stoffes im Schnee beobachtet wird. Hier wäre diese Beobachtung dann über die Leitfähigkeit und nicht wie in \ref{} optisch.
