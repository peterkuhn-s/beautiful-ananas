Anstatt die einzelnen Iterationen der Funktionsmuster genau zu beschreiben werde hier nur herasstehende Eigenschaften beschrieben.


\textbf{1. Itteration}
Eine neue Messung mit neuem Tape kann gemacht werden, indem das alte feuchte Tape mit einem Wasserdichten klebeband abgedeckt wird. dann kann auf das trockene Klebeband ein neues tape von hand aufgetragen werden.

die optische auswertung kann mit einem Smartphone gemacht werden, somit muss kein extra material mitgenommen werden, denn das smartphone hat noch andere einsetzte.

\textbf{2. Itteration}
Die Tapehalter werden mit Elastomeren sicher an die Lichtbox angedruckt. das erlaubt es die lichtbox zu rotieren.

die lichtbox ist in zwei kompartmets aufgeteilt. das erste kompartment ist schwarz um streulicht von ausen zu minimieren. das zweiet kompartmet is in einer hellen farbe um das licht der LED gleichmasiger auf die tapes zu reflektieren.

mit einem innerten gas unter druck, wird der schnee der an dem tape haften bleibt abgeblasen.

Das Tape wird auf eine block auf extrudiertem polystyron (XPS) montiert, so kann der einfluss eines warem tapehalters auf den schnee reduziert werden.

\textbf{3. Itteration}
Die gewichte haben markierungen die es erlauben die Probe mit definierten kinetischer energie auf den schnee auf zu bringen.

die gewichte werden 20 cm oberhalb des schwerpunkts von kunststofffuhrungen gehalten. das verhindert, dass ein tape wahrend der messung um kippt.

\textbf{4. Itteration}
die lichtbox kann zusammengefaltet werden. das erlaubt einen pratztbaren transport an die versuchsstellet.

Die Beleuchtung der Tapse in der Lichtbox wird mit zwei LED panels und diffusoren gemacht, dadurch werden die taps gleichmasig ausgeleuchtet.

mit kaltemittel und einer warmebildkamera wird sichergestellt, dass die taps keine eigene ware haben die den schnee aufschmeltzten.

die gewichte sind modular mit 50 g platten zusammen gesetzt werden. so kann die messung auf einer vielzahl

\textbf{5. Itteration}
Um die optische auswertung durch zu fuhren wird ein Raspberry Pi mit dem HQ Kameramodul  benutzt. dadurch wird das aufgenomme bild einfacher weiter zu verarbeiten.

die lichtbox ist in einer Kunststoffkiste untergebracht, somit kann das streulicht der umwelt effektiv blokiert werden.
