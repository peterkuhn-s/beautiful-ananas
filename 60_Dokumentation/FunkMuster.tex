
In diesem Kapitel werden die verschiedenen Funktionsmuster vorgestellt, die im Verlauf der Entwicklung entstanden sind. Ziel der Funktionsmuster war es, verschiedene Ansätze zur Messung und Auswertung des Flüssigwassergehalts (LWC) im Schnee zu erproben und zu optimieren. Jedes Funktionsmuster adressiert spezifische Herausforderungen und bringt neue Ideen ein, um die Messgenauigkeit und Benutzerfreundlichkeit zu verbessern. Anstatt jede Iteration im Detail zu beschreiben, werden hier die herausragenden Eigenschaften und innovativen Lösungen der einzelnen Entwicklungsstufen präsentiert. Dies soll einen Überblick über die Fortschritte und technischen Lösungen bieten, die im Laufe des Projekts entwickelt wurden.
