Die Messprinzipien versuchen auf unterschiedliche Weise, einen Einflüsse des flüssigen Wassers auf den Schnee abzubilden.

Die Entscheidung, welches Prinzip weiterverfolgt wird, wurde in Absprache mit meinem Betreuer getroffen.

Die Technik des Water Indicator Tapes wurde gewählt, weil sie am elegantesten und mit den wenigsten Umwegen, direkt mit dem flüssigen Wasser im Schnee interagiert.

Es besteht die Vermutung, dass mit dem Tape nicht nur das Verhältnis von Rot und Weiss Flächen ausgewertet werden kann, sondern auch die Grösse und Verteilung der roten Bereiche könnten Informationen über die Geometrie des Schnees liefern. Wenn es gelingt, Aussagen über die Schneegeometrie zu treffen, würde das Water Indicator Tape eine Funktion bieten, die derzeit von kommerziellen Produkten noch nicht erreicht wird.
