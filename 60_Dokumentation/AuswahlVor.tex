Die sechs getesteten Messprinzipien versuchen auf unterschiedliche Weise, einen Einfluss des flüssigen Wassers auf den Schnee abzubilden.

Die Entscheidung, welches Prinzip weiterverfolgt wird, wurde in Absprache mit meinem Betreuer getroffen.

Die Technik des Water Indicator Tapes wurde gewählt, weil sie am elegantesten und mit den wenigsten Umwegen, direkt mit dem flüssigen Wasser im Schnee interagiert.


Es wird vermutet, dass das Tape mehr kann, als nur den LWC durch das Verhältnis von Rot und Weiss zu messen. Die Grösse und Verteilung der roten Bereiche könnten Informationen über die Geometrie des Schnees liefern.

Wenn dies gelingt, könnte das Water Indicator Tape eine Funktion bieten, die kommerzielle Produkte bisher nicht haben.
