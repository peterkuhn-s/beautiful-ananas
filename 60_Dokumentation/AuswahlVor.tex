Die Messprinzipien versuchen auf unterschiedliche Weise, einen Einflüsse des flüssigen Wassers auf den Schnee abzubilden.

Die Entscheidung, welches Prinzip weiterverfolgt wird, wurde in Absprache mit meinem Betreuer getroffen.

Die Technik des Water Indicator Tapes wurde gewählt, weil sie am elegantesten und mit den wenigsten Umwegen, direkt mit dem flüssigen Wasser im Schnee interagiert.

Es besteht die Hoffnung, dass mit dem Tape mehr als nur das Verhältnis von Rot zu Weiss ausgewertet werden kann. Die Grösse und Verteilung der roten Bereiche könnte Aufschluss über die Geometrie des Schnees geben. Falls eine Aussage über die Geometrie des Schnees gemacht werden kann, erreicht das Water Indikator Tape etwas, das die Kommerziennen Produkte noch nicht können.
