


Gleitschneelawinen entstehen, wenn die gesamte Schneedecke auf einem glattem Untergrund wie einem Grashang oder glatten Felsen abrutscht. Dies passiert sowohl bei trockener, kalter als auch bei nasser, isothermer Schneedecke. Typisch für Gleitschneelawinen ist eine dicke Schneedecke ohne oder mit nur wenigen Schwachschichten. Diese Lawinenart wird fast ausschließlich natürlich ausgelöst und kündigt sich oft nur durch sogenannte Gleitschneerisse, sogenannte “Fischmäuler” an. Der Auslösemechanismus beruht auf dem Verlust der Reibung zwischen Schnee und Boden aufgrund von flüssigem Wasser. Man sollte sich nicht in der Nähe von Gleitschneerissen aufhalten, da diese die Lawinengefahr anzeigen, auch wenn sie  keine unmittelbar bevorstehende Auslösung vorhersagen können.

Die Entstehung von Gleitlawinen ist stark von der Feuchtigkeit im Schnee abhängig. Diese Feuchtigkeit sammelt sich zwischen den Eiskristallen und stammt aus verschiedenen Quellen:

\begin{itemize}
    \item Schmelzender Schnee, hauptsächlich durch primare und sekundäre Strahlung.
    \item Regen, der auf die Schneedecke fällt.
    \item Feuchtigkeit aus dem Boden, insbesondere aus wasserführenden Schichten.
\end{itemize}
