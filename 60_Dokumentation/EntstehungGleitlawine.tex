Gleitschneelawinen entstehen, wenn die gesamte Schneedecke auf glattem Untergrund, wie Grashängen oder glatten Felsen, abrutscht. Dies kann sowohl bei trockener, kalter als auch bei nasser, isothermer Schneedecke passieren. Typisch für Gleitschneelawinen ist eine dicke Schneedecke ohne oder mit wenigen Schwachschichten. Diese Lawinenart wird fast ausschließlich natürlich ausgelöst und kündigt sich oft durch Gleitschneerisse an, deren Vorhersage jedoch schwierig ist.  Der Auslösemechanismus beruht auf dem Verlust der Reibung zwischen Schnee und Boden aufgrund von flüssigem Wasser. Zur Vermeidung sollte man sich nicht in der Nähe von Gleitschneerissen aufhalten, da diese die Lawinengefahr anzeigen, aber keine unmittelbare Auslösung vorhersagen können.

Die Entstehung von Gleitlawinen ist stark von der Feuchtigkeit im Schnee abhängig. Diese Feuchtigkeit sammelt sich zwischen den Eiskristallen und stammt aus verschiedenen Quellen:
\begin{itemize}
    \item Schmelzender Schnee, hauptsächlich durch primare und sekundäre Strahlung.
    \item Regen, der auf die Schneedecke fällt.
    \item Feuchtigkeit aus dem Boden, insbesondere aus wasserführenden Schichten.
\end{itemize}
