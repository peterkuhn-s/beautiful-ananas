Der Liquid Water Content (LWC) ist ein entscheidender Parameter in der Meteorologie und Glaziologie, der den Gehalt an flüssigem Wasser in Schnee beschreibt. Der LWC beeinflusst die physikalischen Eigenschaften des Schnees, wie seine Dichte, Wärmeleitfähigkeit und mechanische Stabilität. Typischerweise wird der LWC als Verhältnis des Volumens oder Gewichts von flüssigem Wasser zum Gesamtvolumen oder Gewicht der Eiskristalle ausgedrückt.

Seit 40 Jahren werden Techniken erforscht um den LWC zu messen. Es gibt unterschiedliche Methoden das heterogene Gemisch aus festen, flüssigen und gasförmigen Stoffen, diesen Schaum aus Eis, Wasser und Luft zu messen.
