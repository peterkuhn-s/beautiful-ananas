die verschiedene iterationen haben teilprobleme gelöst, die teilweise in späteren Funktionsmusterr nicht erneut umgesetzt wurden. deswegen ist hier eine Liste an Eigenschaften die von den Iterationen gut gelöst wurden.

Aus der Vorschudie 
\begin{itemize}
\item Fur eine neue messung wird das Gewicht (Holzstab) vorne miteinem neuen wasserdichet schwarzem klebeband für die nächste messung bereit gemacht
\end{itemize}
Messung in Davos
\begin{itemize}
\item Die Tape Holders werden an die Lichtbox fixiert mit einem Gummi Band.
\item die lichtbox hat einen Dunke Kammer um Streulicht zu reduzieren
\item die lichtbox hat eine helle Kammer um die LED beleuchtung weiter zu difundieren
\end{itemize}
Messung auf dem Titlis
\begin{itemize}
\item das gewicht für den Anpressdruck ist variabel, anpassbar auf den schnee
\item die lichtbox ist in einem flachen zustand zu transportieren und wird er am Versuchsort aufgefalltet.
\item die Beleuchtung ist symetisch mit zwei difundierten LED Lampen
\end{itemize}
Schaumodell
\begin{itemize}
\item Es wird ein Rasprebbry Pi mit der HQ Cammera Benutzt. das heisst, die bildauswertung und die Datenbank können lokal laufen. und der bildausschnitt ist statisch.
\item in der Lichtbox wird ein sehr schwarzer Vlis benutzt um das Streulicht unter kontrolle zu halten.
\end{itemize}
