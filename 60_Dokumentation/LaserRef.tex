

\textbf{Funktionsweise}:

Mit einem Laser wird der Schnee sowohl durchleuchtet für die Refraktion als auch angeleuchtet für die Reflexion. In dem Bild \ref{fig:LaserHypothese} ist eine schematische Darstellung der Hypothese dargestellt. Die Grösse Wasseroberfläche und damit die Brennweite ändern sich je nach dem, wie viel Volumen Wasser auf den Eiskristallen ist. Die Effekte der Wasserlinsen sollten in der Refraktion sichtbar werden, indem sich helle und dunkle Bereiche bilden.


Die Effekte der Linsen sollten in der Refraktion sichtbar werden, indem es helle und dunkle Bereiche sich bilden.


\begin{figure}
    \centering
    \includegraphics[width=0.8\textwidth]{Bilder/Reflaktion.jpeg}
    \caption{Hypothese wie das flüssige Wasser die optischen Eigenschaften der Eiskristalle beeinflusst}
    \label{fig:LaserHypothese}
\end{figure}

\textbf{Beispiele in anderen Sektoren}:

Refraktion wird in der Kristallografie angewendet. Die Reflexion von Wasser an einer Glasscheibe wird genutzt, um bei Autos Niederschlag auf der Windschutzscheibe zu messen.


\textbf{Benutzte Mittel für den Versuchsaufbau}:

Als Laserquelle wurde ein grüner Kreuzlaser (<5mW) genutzt. Um sowohl die Reflexion als auch die Refraktion gleichzeitig zu sehen, wurde die Schneeprobe auf einen Mikroskop-Objektträger platziert. Die Ergebnisse des Lasers wurden jeweils auf weissem Papier dargestellt. Die Refraktion wird auf dem Papier an der Unterseite der in Abbildung \ref{fig:LaserRef} zu sehende Holzplatte dargestellt. Mit einem Smartphone wurde eine Videoaufnahme gemacht, wie sich die Ergebnisse des Lasers verändern, während der Schnee schmilzt. Mit einem Spiegel wurde sowohl die Reflexion unten als auch die Refraktion oben gleichzeitig in einem Bild dargestellt.



In Bild \ref{fig:LaserAufbau} ist die Anordnung der verschiedenen Teile auf dem Stativmaterial zu sehen.


\begin{figure}
    \centering
    \includegraphics[width=0.8\textwidth]{Bilder/signal-2024-03-10-112013_006.jpeg}
    \caption{Versuchsaufbau der Laser Reflexion und Refraktion}
    \label{fig:LaserAufbau}
\end{figure}


\textbf{Funktionsweise des Versuchsaufbaus}:

Der Schnee wird im trockenen Zustand bei -10 Grad Celsius aus dem Gefrierschrank auf den gekühlten Objektträger gelegt. Dann wird beobachtet, wie sich die Ergebnisse ändern, wenn der Schnee an der Raumtemperatur schmilzt. Dieser Schmelzvorgang dauerte rund 3 Minuten. Dieser Aufbau ist suboptimal, denn die konstante Reflexion des Objektträgers muss aus dem Laserergebnis herausgerechnet werden.


\begin{figure}[H]
    \centering
    \includegraphics[width=0.8\textwidth]{Bilder/Screenshotfrom2024-04-0413-27-28.png}
    \caption{Messgrössen für die Reflexion und Refraktion, Veränderung über Zeit}
    \label{fig:LaserRef}
\end{figure}



\textbf{Messgrössen}:

Die Inhomogenität des Laserlichts und die Intensität können begutachtet werden. 



\textbf{Aussagekraft der Ergebnisse über den LWC}:

Die Ergebnisse werden direkt von Wasser beeinflusst. Um den Gewichts-LWC zu erhalten, ist aber die Geometrie der Eiskristalle von Bedeutung. Daher ist das Ergebnis nicht direkt mit den LWC überführbar. Mit der 3D-Geometrie der Kristalle wäre die Aussagekraft besser.

\textbf{Reflexion zum Versuchsaufbau}:

Da zwei physikalische Messmethoden gleichzeitig getestet wurden, war der Versuchsaufbau nicht optimal für beide Messgrössen.

Mit den Ergebnissen der Refraktion bin ich zufrieden. Es ist eine klare Veränderung sichtbar wenn der Schnee schmiltz.

Um vergleichbare Werte für den LWC zu bekommen, ist die Kristallgeometrie aber von Bedeutung. Die Messung der Geometrie übersteigt das Ausmass der BA.

Um die Messung der Refraktion durchzuführen, muss eine physikalisch unverändert Schneeprobe durchleuchtet werden. Dies stellt eine  Herausforderung, dar, da der Schnee  aus der Schneedecke aufwendig extrahiert werden muss.

Das Ergebnis der Reflexion ist schwer zu beurteilen. In \cite{Donahue.2022} ist die Reflexion von EM-Wellen bereits mit Erfolg untersucht worden, in Abbildung \ref{fig:IRPaper} ist ein Ergebniss abgebieldet.



\textbf{Verbesserungen des Versuchsaufbaus}:

Um bessere Reflexionsergebnisse zu bekommen, keinen Objektträger nutzen, sondern direkt auf Schnee leuchten. Für eine statische Messung einer Schneeprobe muss die Luft um den Schnee herum gekühlt sein. Ein Ansatz dafür wird im Vorversuch \ref{sec:TinteVersuchsaufbau} umgesetzt. Mit dem Laser wird Energie in den Schnee eingebracht. Um das Schmelzen und damit Verfälschen des LWC zu minimieren, sollte ein möglichst schwacher Laser eingesetzt werden.


