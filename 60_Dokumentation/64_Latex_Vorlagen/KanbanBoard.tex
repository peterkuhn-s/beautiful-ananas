\subsection{Kanban Board}
\label{KanbanBoard}
Währen der Experemtierphase der Semesterarbeit habe ich mit diesem Kanban Board gearbeiten. Hier ist der Stand nach Woche 48 abgebildet. Die Prioritäten konnten mit dem Kanban Board flexibel auf den verlauf der Semesterarbeit angepasst werden.

Backlog:
\begin{itemize}
\item Einlegeplatte neu konstruieren, so dass das ost gadget auch funktioniert
 \item Präsentation der SA machen
 \item Simon Grim wegen der Oberflächenbehandlung von Abstandhaltern fragen
  
\end{itemize}

To Do:
\begin{itemize}
\item Lineare ML mit Kombinationen, und loocv sampling
 \item Temperatursensor der Messzelle verifizieren
 \item Wage digital auslesen
 \item Datamatirx mit Raspberry pi und kamera modul auslesen
 
\end{itemize}

in Progess:
\begin{itemize}
\item automatische speicherung der Messzelllen daten
 \item Dokumentation der SA schreiben
 \item Lohnfertiger Schilling in Mollis Offerte zu Abstandhalter einholen
 \item Kanban Board Kritik von Arberding einholen
 \item 5 achsen 3 d Drucker anfragen

 
\end{itemize}

Verification:
\begin{itemize}
\item 10-fold cross validation der ML
 \item schwindungsverlauf messen
 \item DoE für kleinen Versuch generieren
 \item DoE für grossen Versuch generieren
 \item Passive Abstandhalter testen
 \item Aluminium Abstandhalter testen
   \item Zentral punkt bestimmen
\end{itemize}

Resolved:
\begin{itemize}
\item Fertigungszeichnung von Abstandhalter
\item ML für dummy daten aus CVS
\item material für validirungsverusche bestimmen
 \item lineare ML
 \item messprezision von Messzelle bestimmen
 \item alternative zu Lupolen 1800 H aus Lupolen 1800 Familie suchen
 \item Daten von Messzelle empfangen
 \item 4 Aluminium Abstandhalter selber drehen
 \item kleiner DoE abfahren
   \item Messzelle höhe Kalibiren
   \item grosser DoE abfahren
     \item Matierialtests mini DoE abfahren
 
\end{itemize}

Closed:
\begin{itemize}
\item ML mit Postgres verbinden
 
\end{itemize}
