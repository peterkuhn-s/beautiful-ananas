
\subsection{Durchführung des Grösseren DoE}
\label{GrosserDoE}
Die Anzahl von Halbschalen pro Versuchsdurchlauf wurde von 16 in \ref{kleinerDoE} auf 8 reduziert. So war es möglich mehr Versuchsdurchläufe an einem Arbeitstag durch zu führen.

Die Abweichungen der Parameter wurde reduziert, da die Cluster bei \ref{kleinerDoE} genügen ausgeprägt waren. Dadurch hat sich auch die Wartezeit für die Temperaturänderungen reduziert.


 \begin{table}[htbp]
   
  \caption{Versuchsplan für den Grösseren DoE}
  \label{tab:grosserDoe}
  \begin{tabular}{cccc}
    \hline
    \textbf{Zylindertemperatur} & \textbf{Nachdruckhöhe} & \textbf{Vorlauftemperierung} & \textbf{Volumenstrom} \\
    \hline
    209 & 262 & 24 & 31 \\
    209 & 262 & 41 & 25 \\
    209 & 350 & 41 & 25 \\
    209 & 350 & 24 & 31 \\
    209 & 437 & 33 & 18 \\
    209 & 437 & 24 & 31 \\
    220 & 350 & 33 & 25 \\
    220 & 262 & 33 & 31 \\
    220 & 437 & 33 & 31 \\
    220 & 350 & 41 & 18 \\
    220 & 437 & 41 & 31 \\
    220 & 262 & 24 & 18 \\
    231 & 262 & 24 & 25 \\
    231 & 350 & 24 & 31 \\
    231 & 437 & 24 & 18 \\
    231 & 262 & 33 & 31 \\
    231 & 350 & 41 & 18 \\
    231 & 437 & 41 & 31 \\
    209 & 262 & 24 & 31 \\
 
  \end{tabular}
 \end{table}

 Die Einstellparameter, die an der Maschine eingestellt wurden, entsprechen grösstenteils den oben aufgelisteten Werten. Genauere Angaben sind in den Prozessparametern und Herstellungsprotokolls nachzusehen.

\begin{figure}
   
  \includegraphics[width=0.48\textwidth]{images/Screenshot from 2023-12-09 23-36-48.png}
  \caption{Die Gewichte der Halbschalen des grösseren DoEs. Es ist ein Basisgewicht bei 11.05 g zu erraten. Es gibt schöne Cluster von jeweils 8 Halbschalen die deutlich höhere Werte aufweisen}
  \label{fig:GDoEGe}
\end{figure}



\begin{figure}
   
  \includegraphics[width=0.58\textwidth]{images/Screenshot from 2023-12-10 00-18-45.png}
  \caption{Die Höhe der Halbschalen des grösseren DoEs. Die Cluster sind nicht mehr so ausgeprägt wie in \ref{fig:ErgKleinDoe}. Es ist gut der Zentralpunkt am Anfang und am Ende zu erkennen.}
  \label{fig:GDoEH}
\end{figure}



\begin{figure}
   
  \includegraphics[width=0.58\textwidth]{images/Screenshot from 2023-12-10 00-23-38.png}
  \caption{Der Aussendurchmesser der Halbschalen des Grösseren DoEs. Die Clusterbindlung ist stärker ausgebildet als in \ref{fig:AlleWerteKleinerDoEAusen}.
}
  \label{fig:GDoED}
\end{figure}


\begin{figure}
   
  \includegraphics[width=0.58\textwidth]{images/Screenshot from 2023-12-10 10-12-46.png}
  \caption{Der Innendurchmesser der Halbschalen des Grösseren DoEs. Die Clusterbindlung ist stärker ausgebildet als in \ref{fig:AlleWerteKleinerDoEAusen}.
}
  \label{fig:GDoEI}
\end{figure}




\begin{figure}
   
  \includegraphics[width=0.58\textwidth]{images/Screenshot from 2023-12-10 10-15-32.png}
  \caption{Der Rundheit Halbschalen des Grösseren DoEs. Hier ist ein Grundwert zu erkennen und einzelne Ausreisser. Die Ausreisser sind nicht durch die Einstellparameter beeinflusst.}
  \label{fig:GDoER}
\end{figure}


\subsection{Durchführung des Materialtest}

Für den Materialtest wurden zwei verschiedene Materialien verwendet: Lupolen 1800 H Schwarz und Lupolen 1800 P.

Lupolen 1800 H ist ein Polyethylen-Harz mit niedriger Dichte, das in einer Vielzahl von Verarbeitungsmethoden wie Spritzguss, Blasformen und Filmextrusion eingesetzt wird. Es zeichnet sich durch sehr gute Weichheit und Zähigkeit aus und bietet eine gute dimensionsstabile Eigenschaft. Wie Lupolen 1800 P wird auch Lupolen 1800 H in Form von Pellets geliefert und ist nicht additiviert. Typische Anwendungen für dieses Material umfassen Verschlusskappen, Deckel, Schaumstoffe und Champagnerkorken. Der Melt Flow von Lupolen 1800 H ist 1.5 (190 °C/2.16 kg) [g/10 min]


Weiss ist ein Masterbatch, das für die Einfärbung von Kunststoffen verwendet wird. Es handelt sich um ein Pigmentkonzentrat, das dem Kunststoff während des Spritzgussprozesses zugesetzt wird. Diese Art von Masterbatch wird typischerweise verwendet, um weissen Farbeffekte in Kunststoffprodukten zu erzeugen. Die genaue Zusammensetzung und die spezifischen Eigenschaften können je nach Hersteller variieren.


Schwarz ist ein Masterbatch, das für die Einfärbung von Kunststoffen verwendet wird. Es handelt sich um ein Pigmentkonzentrat, das dem Kunststoff während des Spritzgussprozesses zugesetzt wird. Diese Art von Masterbatch wird typischerweise verwendet, um schwarze Farbeffekte in Kunststoffprodukten zu erzeugen. Der genaue Zusammensetzung und die spezifischen Eigenschaften können je nach Hersteller variieren.

Lupolen 1800 P

Lupolen 1800 P ist ein Polyethylen-Harz mit niedriger Dichte, das für verschiedene Verarbeitungsmethoden wie Spritzguss verwendet wird. Es zeichnet sich durch eine hohe Fliessfähigkeit aus und bietet gute Weichheit, Zähigkeit und dimensionsstabile Eigenschaften. Lupolen 1800 P wird in Form von Pellets geliefert und ist nicht additiviert, was bedeutet, dass keine speziellen Zusatzstoffe enthalten sind. Dieses Material wird in Anwendungen wie Compoundierung, Spielzeugherstellung, Verschlusskappen, Teilen für den Maschinenbau sowie Sport- und Freizeitausrüstung eingesetzt.  Der Melt Flow von Lupolen 1800 P ist 15 (190 °C/2.16 kg) [g/10 min]. das Material konnte mit den gleichen Einstellparameter verarbeitet werde wie das Lupolen 1800 H.

\begin{table}[h]
   
  \caption{DoE}
  \begin{tabular}{cccccc}
    \hline
    zylindertemperatur & nachdruckhohe & vorlauftemperierung & volumenstrom \\
    \hline
    Schwarz  & & & \\
    220 & 350 & 33 & 25 \\
    220 & 437 & 41 & 31 \\
     220 & 262 & 24 & 18 \\
     220 & 350 & 33 & 25 \\
    \hline
    Lupolen 1800 P &  & & \\
     220 & 350 & 33 & 25 \\
     220 & 437 & 41 & 31 \\
     220 & 262 & 24 & 18 \\
     220 & 350 & 33 & 25 \\
    \hline
  \end{tabular}
\end{table}


\begin{figure}
   
  \includegraphics[width=0.58\textwidth]{images/Screenshot from 2023-12-10 10-26-16.png}
  \caption{Die Höhe. Die vier Cluster sind erkennbar, aber nicht so ausgeprägt wie in \ref{fig:GDoEH}.}
  \label{fig:1800PH}
\end{figure}



\begin{figure}
   
  \includegraphics[width=0.58\textwidth]{images/Screenshot from 2023-12-10 10-30-01.png}
  \caption{Der Aussendurchmesser. Es ist eine ausgepräge Clusterbinldung erkennbar.}
  \label{fig:1800PA}
\end{figure}



Die weiteren QM entsprechen mehr oder weniger den Erwartungen und werden deshalb hier nicht aufgezeigt. Mit den Scripts können die Grafiken aufgerufen werden. Die Cluster sind erkennbar, aber nicht so klar definiert wie bei \ref{fig:1800PA}

\begin{figure}
   
  \includegraphics[width=0.58\textwidth]{images/Screenshot from 2023-12-10 10-36-00.png}
  \caption{Die Messungen der Höhe. Die vier Cluster sind erkennbar, aber nicht so ausgeprägte wie in \ref{fig:GDoEH}.}
  \label{fig:1800HH}
\end{figure}


\begin{figure}
   
  \includegraphics[width=0.58\textwidth]{images/Screenshot from 2023-12-10 10-33-51.png}
  \caption{Der Aussendurchmesser. Die vier Cluster sind erkennbar, aber nicht so ausgeprägte wie in \ref{fig:1800PA}.}
  \label{fig:1800HA}
\end{figure}


\begin{figure}%r and l for right and left
   
  \includegraphics[width=0.48\textwidth, angle=90]{images/_MG_6018.JPG}
  \caption{Hier ist der Laser im Hintergrund mit einem Förderband. im vordergrund ist die Wage}
  \label{fig:wage}
\end{figure}





\subsection{Gewicht der Halbschalen von Hand erfassen}
In der Grafik \ref{fig:WageVonHand} ist klar zu sehen, dass die Abweichungen sehr signifikant sind. In der Grafik ist sind 4 Halbschalen sehr wiederholbar. Der normalisierte Wert ist konstant 1, die vier Halbschalen sind nicht von einander zu unterschieden.


\begin{figure}
   
  \includegraphics[width=0.48\textwidth]{images/Screenshot from 2023-12-09 12-42-33.png}
  \caption{Präzision der Wage \cite{SchwindungLit}}
  \label{fig:WageVonHand}
\end{figure}


Wenn die Rohdaten der anderen 4 Halbschalen betrachtet werden, ist erkennbar, dass die Zahlen gleich sind, aber voll mit Flüchtigkeitsfehlern. Jede Halbschale wurde sechs Mal gewogen, aber 1698998377 wird nur fünf Mal aufgelistet, dass heisst, es ist schon ein Flüchtigkeitsfehler in der Datamatrix.


\begin{table}[h]
   
  \caption{Ein Ausschnitt der Rohdaten der Gewichts Präzision Messung}
  \begin{tabular}{|c|c|c|c|}
    \hline
    \textbf{IDwagevonhand} & \textbf{DatamatrixVonHand} & \textbf{TimestampHand} & \textbf{gewichtVonHand} \\
    \hline
    156 & 1698998212 & 1699018366 & 11.011 \\
    163 & 1698998212 & 1699018455 & 11.011 \\
    173 & 1698998212 & 1699018579 & 11.011 \\
    140 & 1698998212 & 1699018021 & 11.011 \\
    150 & 1698998212 & 1699018290 & 11.11 \\
    132 & 1698998212 & 1699017064 & 11.105 \\
    154 & 1698998240 & 1699018341 & 11.095 \\
    161 & 1698998240 & 1699018434 & 11.095 \\
    133 & 1698998240 & 1699017076 & 11.095 \\
    143 & 1698998240 & 1699018055 & 11.095 \\
    172 & 1698998240 & 1699018567 & 11.095 \\
    146 & 1698998240 & 1699018244 & 11.095 \\
    142 & 1698998377 & 1699018043 & 11.013 \\
    164 & 1698998377 & 1699018466 & 11.013 \\
    148 & 1698998377 & 1699018267 & 11.013 \\
    169 & 1698998377 & 1699018533 & 11.013 \\
    158 & 1698998377 & 1699018398 & 11.013 \\
    \hline
  \end{tabular}
\end{table}

Um weniger Fehler zu machen, wurde das HMI geändert, so dass mehr Achtsamkeit erforderlich ist.
