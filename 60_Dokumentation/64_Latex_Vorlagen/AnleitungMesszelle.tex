
Starten der Messzelle
\begin{enumerate}
\item Hauptschalter auf Kniehöhen einschalten
\item kleiner Powerbutton in der mitte oben des blaune Bildschrims
\item lange warte bis die beiden gespiegelten Bildschrime oben die Kistler Applikation automatisch starten
\item überprüfen ob das HMI Fehler anzeigt, unter dem Menüpunkt links können die Fehler genau angezeigt werden und quitiert werden
\item Die Türzuhaltung muss mit dem weisen Knopf zwischen den Bildschirmen bestätigt werden
\item die Messzelle mit dem reset neben dem Türknopf 'freigeben'
\item den Roboter an seine Homeposition zurückfahren, Button gedrückt halten
  \item das Programm des Roboters starten, links unten
[{"PartId":"504","Level":"0","ResultState":"Nok","MeasurementsLength":18,"Measurements":[{"ActualValue":35.940796,"GaugeId":"00Gauge0K1: Höhe - 0,000 (± 0,000)","Nest":"0","ResultState":"Ok","WarningAlert":false},{"ActualValue":26.061357,"GaugeId":"00Gauge4S1: Temperatursensor","Nest":"0","ResultState":"Ok","WarningAlert":false},{"ActualValue":9.32226,"GaugeId":"00Gauge1K2: Innendurchmesser","Nest":"0","ResultState":"Ok","WarningAlert":false},{"ActualValue":72.018135,"GaugeId":"00Gauge5K2: Ø-Aussen (BestCircle)","Nest":"0","ResultState":"Ok","WarningAlert":false},{"ActualValue":0.30598307,"GaugeId":"00Gauge3K2: Konzentrizität","Nest":"0","ResultState":"Ok","WarningAlert":false},{"ActualValue":0.13129044,"GaugeId":"00Gauge2K2: Rundheit","Nest":"0","ResultState":"Ok","WarningAlert":false},{"ActualValue":71.94689,"GaugeId":"00Gauge6K2: Ø-Aussen (punktuell 0°)","Nest":"0","ResultState":"Ok","WarningAlert":false},{"ActualValue":72.094215,"GaugeId":"00Gauge7K2: Ø-Aussen (punktuell 120°)","Nest":"0","ResultState":"Ok","WarningAlert":false},{"ActualValue":72.0234,"GaugeId":"00Gauge8K2: Ø-Aussen (punktuell 240°)","Nest":"0","ResultState":"Ok","WarningAlert":false},{"ActualValue":26.457909,"GaugeId":"00Gauge9K3: Kanal R (Grauwert)","Nest":"0","ResultState":"Ok","WarningAlert":false},{"ActualValue":10.396678,"GaugeId":"00Gauge10K3: Kanal G (Grauwert)","Nest":"0","ResultState":"Ok","WarningAlert":false},{"ActualValue":8.1474333,"GaugeId":"00Gauge11K3: Kanal B (Grauwert)","Nest":"0","ResultState":"Ok","WarningAlert":false},{"ActualValue":1.6067556,"GaugeId":"00Gauge12K3: Kanal H (Grauwert) ","Nest":"0","ResultState":"Ok","WarningAlert":false},{"ActualValue":229.19911194,"GaugeId":"00Gauge13K3: Kanal S (Grauwert) ","Nest":"0","ResultState":"Ok","WarningAlert":false},{"ActualValue":26.457909,"GaugeId":"00Gauge14K3: Kanal V (Grauwert) ","Nest":"0","ResultState":"Ok","WarningAlert":false},{"ActualValue":0.0,"GaugeId":"00Gauge22K3: Datamatrix Size","Nest":"0","ResultState":"Error","WarningAlert":false},{"ActualValue":0.0,"GaugeId":"00Gauge23K3; Datamatrix Status","Nest":"0","ResultState":"Error","WarningAlert":false},{"ActualValue":0.0,"GaugeId":"00Gauge21K3: Datamatrix Wert","Nest":"0","ResultState":"Error","WarningAlert":false}]}]
  
  
\end{enumerate}

Fehlerbehebung an der Messzelle
\begin{enumerate}
\item Der roboterarm hat den Ball verlohren
\item Der roboter kann sich nicht merken in welchem moment ein Fehler aufgetreten ist. Das heisst der Messzyklus wird immer vom Anfang begonnen.
\item Die Türzuhaltung mit dem weisen knopf zwischen den Bildschirmen entriegen
\item eine der vier Türen aufmachen
\item den Ball aus der Messzelle nehmen
  \item Die fehler auf dem roboterdisplay bestätigen 
  \item den Rotorarm mit den kartesischen Kooridnatensteuerung in eine sichere position in der mitte fahren
    \item auf den oberen Button des Roboters drücken und gedrückt halten, der roboter fährt an seine Homeposition
  \end{enumerate}

