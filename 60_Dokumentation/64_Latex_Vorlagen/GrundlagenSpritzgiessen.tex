\subsection{Herstellungsprozess des Spritzgiessen}
Spritzguss ist ein Fertigungsprozess zur Herstellung von Teilen durch das Einspritzen von geschmolzenem Material in eine Form. Der Spritzguss kann mit einer Vielzahl von Materialien durchgeführt werden, hauptsächlich Metalle (wofür der Prozess als Druckguss bezeichnet wird), Gläser, Elastomere, Süsswaren und am häufigsten thermoplastische und duroplastische Polymere. Material für das Teil wird in ein beheiztes Fass eingespeist, gemischt (unter Verwendung einer Schneckenwelle) und in eine Kavität eingespritzt, wo es abkühlt und sich der Formgebung der Form anpasst. Nachdem ein Produkt entworfen wurde, normalerweise von einem Industriedesigner oder Ingenieur, werden Formen \ref{fig:Werkzeugu} von einem Werkzeugmacher aus Metall hergestellt, normalerweise entweder aus Stahl oder Aluminium, und präzisions gefertigt, um die Merkmale des gewünschten Teils zu bilden. Der Spritzguss wird weitgehend für die Herstellung verschiedener Teile verwendet, von den kleinsten Komponenten bis zu ganzen Karosserieteilen von Autos.



Der Spritzguss verwendet eine spezielle Maschine mit drei Hauptteilen: der Einspritzeinheit, der Form und der Klemme. Teile, die spritzgegossen werden sollen, müssen sehr sorgfältig entworfen werden, um den Spritzgussprozess zu erleichtern. Dabei müssen das für das Teil verwendete Material, die gewünschte Form und Merkmale des Teils, das Material der Form und die Eigenschaften der Spritzgussmaschine alle berücksichtigt werden. Die Vielseitigkeit des Spritzgusses wird durch diese Vielfalt an Design-Überlegungen und Möglichkeiten erleichtert.

\begin{figure}%r and l for right and left
   
  \includegraphics[width=0.48\textwidth]{images/_MG_6005.JPG}
  \caption{Die Düsenferne Seite des Werkzeugs, mit dem die Halbschalen hergestellt werden}
  \label{fig:Werkzeugu}
\end{figure}

\subsection{Schwindung beim Spritzgiessen}
Die Schwindung beim Spritzgussprozess ist ein wesentlicher physikalischer Aspekt, der bei der Herstellung von Kunststoffformteilen berücksichtigt werden muss.

Dieser Vorgang beschreibt die Volumenverkleinerung des Formteils während des Abkühlens und nach dem Entformen aus dem Werkzeug. Die Schwindung ist eine relative Grösse und wird in Prozent angegeben. Es handelt sich um die Differenz zwischen dem ursprünglichen Volumen der Werkzeugkavität und dem resultierenden Volumen des geformten Teils nach dem Abkühlen.

Um die Schwindung zu berücksichtigen und den Werkzeugbau zu optimieren, ist es üblich, Simulationen und iterative Tests durchzuführen. Dies ermöglicht es  dass die hergestellten Teile die gewünschten Masse und Eigenschaften haben. Die Schwindung ist ein komplexer Prozess, der von verschiedenen Faktoren beeinflusst wird, darunter der Werkstoff, die Verarbeitungsparameter und die geometrische Form des Teils.

Ein interessantes Merkmal der Schwindung ist, dass sie nicht sofort nach dem Spritzguss abgeschlossen ist. Vielmehr dauert es einige Zeit, bis die Schwindung vollständig stabilisiert ist. Diese sogenannte Nachschwindung setzt sich aus den weiteren Volumenänderungen zusammen, die nach dem Abkühlen und Entformen auftreten. Die genaue Kenntnis dieser Schwingungsdynamik ist entscheidend für die Herstellung von präzisen Kunststoffteilen und hat Auswirkungen auf Aspekte wie Werkzeugdesign, Produktionszeitpläne und Qualitätskontrolle.

Heute ist es üblich Spritzgussteile entweder inline zu vermessen, oder stichprobenweise nachdem die Schwindung abgeschlossen ist. Ein Problem der inline Messung ist, dass die Schwindung der Teile noch nicht abgeschlossen ist. Das heisst die Qualitätsmerkmale, die der Kunde braucht, können noch nicht gemessen werden.

Das Problem der Offline Stichprobe ist, dass falls ein Problem der Toleranzen gefunden wird, die Maschiene schon sehr viele weitere Teile produziert hat, die alle mit dem gleichen Problem behaftet sind. Dass heisst, ein grosser Ausschuss an Teilen wird produziert.

Mit den heutigen Bestrebungen der Nachhaltigkeit sollte eine reine Offline Messung der Qualitätsmerkmale vermieden werden.
