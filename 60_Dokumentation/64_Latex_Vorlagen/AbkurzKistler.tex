\subsection{Beschreibung der Messergebnisse}
\label{AbKist}

Die folgenden Informationen bieten eine detaillierte Übersicht über die Messergebnisse, wobei zuerst der genaue Name in der Datenbank angegeben wird, gefolgt von einer Erklärung und einem Beispielwert für die Werkzeugkavität. Alle Zahlen werden als float/numeric abgespeichert, wobei die Abweichungen in den Illustrationen überhöht dargestellt sind. Das angebebene Mass ist das der Werkzeugkavität.

\subsubsection*{ID}
Die ID ist eine eindeutige Identifikationsnummer für jeden Eintrag in der Datenbank.

\subsubsection*{Part-ID}
Die Part-ID ist eine Identifikationsnummer, die jeder Messung von Kistler vergeben wird.

\subsubsection*{Höhe}
Die Höhe einer Halbschale ist der maximale Abstand des Bodens bis zur Aussenkontur der Halbschale (36.665 mm).

\begin{wrapfigure}{r}{0.3\textwidth}
\centering
\includegraphics[width=0.28\textwidth]{images/_MG_6032Hohe.JPG}
\caption{Höhe der Halbschale}
\label{fig:DefH}
\end{wrapfigure}
-\\
\\
\\
\\
\\
\\
\subsubsection*{Temperatursensor}
Ein IR-Sensor, der die Temperatur der Halbschale bei Kamera 1 misst (25.29214 Grad Celsius).


\subsubsection*{Innendurchmesser}
Der Durchmesser eines Kreises, der das Loch in der Mitte der Halbschale approximiert (9.8552 mm).

\begin{wrapfigure}{r}{0.3\textwidth}
\centering
\includegraphics[width=0.28\textwidth]{images/_MG_6020Innendurchmesser.JPG}
\caption{Innendurchmesser der Halbschale}
\label{fig:DefI}
\end{wrapfigure}
-\\
\\
\\
\\
\\
\\

-\\
\\
\\
\\
\\
\\
\subsubsection*{Durchmesser aussen}

Der Durchmesser eines Kreises, der die Aussenkontur der Halbschale approximiert (73,152 mm).

\begin{wrapfigure}{r}{0.3\textwidth}
\centering
\includegraphics[width=0.28\textwidth]{images/_MG_6020Durchmesser.JPG}
\caption{Aussendurchmesser der Halbschale}
\label{fig:DefA}
\end{wrapfigure}
-\\
\\
\\
\\
\\
\\
\\
\\
\\
\\
\\
\\
\\
\\
\subsubsection*{Konzentrizität}
Ein Mass dafür, wie genau der Mittelpunkt des Loches mit dem Mittelpunkt der Aussenkontur übereinstimmt (0).

\begin{wrapfigure}{r}{0.3\textwidth}
\centering
\includegraphics[width=0.28\textwidth]{images/_MG_6020Konz.JPG}
\caption{Konzentrizität der Halbschale}
\label{fig:DefK}
\end{wrapfigure}
-\\
\\
\\
\\
\\
\\
\\
\\
\\
\\
\\
\\
\subsubsection*{Rundheit}
Der Grad der Rundheit der Aussenkontur der Halbschale (0).

\begin{wrapfigure}{r}{0.3\textwidth}
\centering
\includegraphics[width=0.28\textwidth]{images/_MG_6020Rundheit.JPG}
\caption{Rundheit der Halbschale}
\label{fig:DefR}
\end{wrapfigure}

-\\
\\
\\
\\
\\
\\
\\
\\
\\
\\
\\
\\
\\
\\
\\
\\
\\
\\
\\
\\
-\\

\subsubsection*{Aussen-0-Degree}

Der Durchmesser der Halbschale bei 0 Grad (72 mm).

\begin{wrapfigure}{r}{0.3\textwidth}
\centering
\includegraphics[width=0.28\textwidth]{images/_MG_602durchmesser0.JPG}
\caption{Durchmesser bei 0 Grad}
\label{fig:Def0}
\end{wrapfigure}

\begin{wrapfigure}{r}{0.3\textwidth}
\centering
\includegraphics[width=0.28\textwidth]{images/_MG_602durchmesser240.JPG}
\caption{Durchmesser bei 120 Grad}
\label{fig:Def120}
\end{wrapfigure}
-\\
\\
\\
\\
\\
\\
\subsubsection*{Aussen-120-Degree}
Der Durchmesser der Halbschale bei 120 Grad (72 mm).


-\\
\\
\\
\\
\\
\\
\\
\\
\\
\\
\\
\\

\subsubsection*{Aussen-240-Degree}
Der Durchmesser der Halbschale bei 240 Grad (72 mm).

\begin{wrapfigure}{r}{0.3\textwidth}
\centering
\includegraphics[width=0.28\textwidth]{images/_MG_602durchmesser120.JPG}
\caption{Durchmesser bei 240 Grad}
\label{fig:Def240}
\end{wrapfigure}

-\\
\\
\\
\\
\\
\\
\\
\\
\\
\\
\\
\subsubsection*{Channel-R, Channel-G, Channel-B, Channel-H, Channel-S, Channel-V}
Farbkanal-Werte der Halbschale (R: 182, G: 181, B: 172, H: 56, S: 10, V: 178).

\subsubsection*{Datamatrix-Size}
Grösse der Datamatrix auf dem Teil (10).

\subsubsection*{Datamatrix}
Datamatrix-Code, der auf der Halbschale gelasert ist (mit einem separaten Skript aus A und B zusammengesetzt: 1699003799).

\subsubsection*{Datamatrix-Wert-A, Datamatrix-Wert-B}
Die ersten fünf Zeichen (Wert-A: 16990.0) und die nächsten fünf Zeichen (Wert-B: 3799.0) der Datamatrix.
