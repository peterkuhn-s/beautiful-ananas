
\subsection{ML in der Schweiz}
Die Schweizer Regierung hat die Swiss AI Initiative ins Leben gerufen, mit dem Ziel, ein Zentrum für KI-Kompetenzen in der Schweiz aufzubauen. Die Initiative strebt die Zusammenarbeit von Universitäten (z. B. EPFL, ETH), Fachhochschulen und Industrieunternehmen an. Der Schwerpunkt liegt dabei auf generativer KI, wie beispielsweise ChatGPT. Ein integraler Bestandteil dieser Initiative ist das Swiss National Supercomputing Center (CSCS) in Lugano. \cite{SwissAI}

\section{Neue Kommerzielle Produkte}
Das Thema AI hat dieses Jahr den Höhepunkt im Gartner Hypezycle erreicht \cite{hypecycle}. Daher ist es nicht überraschend, dass neue Produkte auf den Markt gebracht wurden, die KI mit Spritzgusstechnologie verbinden. Einige Artikel \cite{NewsP}, \cite{ieeeP} beschreiben, wie AI die Industrie transformieren wird.

Die Recherche nach neuen Softwareprodukten wurde für diese Semesterarbeit kurzgehalten, da das IWK bereits die Messzelle erhalten hat und die ML-Techniken in der Vorlesung SML ausführlich erklärt wurden.

Die Produkte \cite{katulu}, \cite{WH}, die ich finden konnte, hatten das Ziel, die Zykluszeit zu reduzieren, die Anlaufzeit eines neuen Werkzeugs zu verkürzen oder ähnliche Ziele zu erreichen. Kein Produkt, das ich finden konnte, hatte das Ziel, Qualitätsmerkmale nach der Schwindung vorherzusagen.

In den letzten Tagen hat die Zeitschrift Nature einen relevanter Artikel \cite{NatureP} zu BPNN-Modellen veröffentlicht.
