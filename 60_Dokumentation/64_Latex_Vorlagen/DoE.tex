
\subsection{Vorgehen beim Erstellen der Versuchspläne}

Die Versuchsplanung erfolgte in aufeinander aufbauenden Schritten, wobei die Komplexität schrittweise gesteigert wurde. Zunächst wurde die Messzelle isoliert getestet. In einem zweiten Schritt wurde ein kleines Taguchi Design of Experiments (DoE) durchgeführt, gefolgt von einem umfangreicheren DoE in der dritten Iteration. Die Erkenntnisse aus jeder Iteration wurden genutzt, um das Vorgehen für die nächste Phase zu optimieren.
\subsection{Wert der Erfahrung bei ML}
Computer neigen dazu, 'Garbage in Garbage out' zu machen.

Daher wurde die Grundeinstellung der Einstellparameter vom erfahrenen Verfahrenstechniker Christoph gewählt. 


\subsection{Interaktionen der Parameter}

Ein klassisches Taguchi Design ist nicht optimal für die Abbildung von Interaktionen. Die traditionelle Auswahl von 8 aus den $81 = 3^4$ möglichen Kombinationen im Taguchi-Ansatz ist ausreichend, wenn das Ziel die Maximierung der Outputs ist. Da das Hauptziel meines DoE jedoch das Trainieren eines maschinellen Lernmodells ist, sind Interaktionen von grösserer Bedeutung. Daher wurde die Anzahl der Kombinationen im letzten DoE verdoppelt. Natürlich auftretende Schwankungen wären noch besser geeignet, um das maschinelle Lernmodell zu trainieren.



\subsection{Generierung der Einstellparameter}

Der Versuchsplan für das Design of Experiments wurde durch Variationen um einen Zentralpunkt erstellt. Dieser Zentralpunkt repräsentiert die Einstellungen, die ein erfahrener Verfahrenstechniker für eine 'gute' Halbschale wählen würde.

Es wurden vier Parameter ausgewählt, die höchstwahrscheinlich einen Einfluss auf die gespritzten Halbschalen haben: Zylindertemperatur (Schmelzetemperatur), Temperatur der Temperierkreisläufe des Werkzeugs (Werkzeugtemperatur), Einspritzgeschwindigkeit (beeinflusst ebenfalls die Schmelzetemperatur) und die Nachdruckhöhe (beeinflusst die Schwindungskompensation im Werkzeug).

Der Versuchsplan wurde in einer Matrix festgehalten, die in der Literatur \cite{VorlesungWah} eingesehen werden kann:

\begin{table}[h]
\centering
\begin{tabular}{cccc}
%\hline
2 & 2 & 2 & 2 \\
2 & 3 & 3 & 3 \\
2 & 1 & 1 & 1 \\
3 & 2 & 3 & 1 \\
3 & 3 & 1 & 2 \\
3 & 1 & 2 & 3 \\
1 & 2 & 1 & 3 \\
1 & 3 & 2 & 1 \\
1 & 1 & 3 & 2 \\
2 & 2 & 2 & 2 \\
%\hline
\end{tabular}
\caption{Versuchsplan - L2 Plan für 4 Parameter mit jeweils 3 Stufen}
\label{tab:doe_matrix}
\end{table}

Nach Durchführung des Versuchsplans wurde der Zentralpunkt erneut wiederholt, um sicherzustellen, dass sich an der Maschine nichts verändert hat.

Hier sind sowohl der Zentralpunkt als auch die Abweichungen für den kleinen DoE \ref{kleinerDoe} abgebildet:

\begin{table}[h]
\centering
\begin{tabular}{ccc}
\hline
\textbf{Parameter} & \textbf{Zentralpunkt} & \textbf{Abweichung} \\
\hline
Zylindertemperatur & 220 & 0.12 \\
Nachdruck & 350 & 0.45 \\
Werkzeugtemperatur & 33 & 0.50 \\
Einspritzgeschwindigkeit & 25 & 0.50\\
\hline
\end{tabular}
\caption{Einstellungen für den kleinen DoE}
\label{tab:ein_klein_doe}
\end{table}

Die berechneten Werte sind in \ref{WerteKleinDoe} aufgeführt. Aufgrund von Problemen mit unvollständig gefüllten Halbschalen und ausgeprägten Unterschieden in den Einstellungen konnte, die Abweichung für den grösseren DoE reduziert werden \ref{tab:grosserDoe}.
