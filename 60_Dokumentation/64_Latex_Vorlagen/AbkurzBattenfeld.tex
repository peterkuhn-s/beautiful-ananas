

\begin{longtable}{|p{4cm}|p{10cm}|}
  \caption{Abkürzungen und ihre Bedeutungen} \\
  \hline
  \textbf{Abkürzung} & \textbf{Bedeutung} \\
  \hline
  \endhead
  id\_prozess & Ein Counter, der in der Battenfeld gesetzt werden kann \\
  zyklus & Ein Counter, der in der Battelfeld gesetzt werden kann \\
  datum & Das Datum der Battenfeld \\
  startzeit & Der Zeitpunkt, an dem die Battenfeld mit dem Aufdosieren des Materials beginnt \\
  auftragsnummer & Eine Nummer, die vergeben werden kann; bei meinen Versuchen nicht gesetzt \\
  projektnummer & Eine Nummer, die vergeben werden kann; bei meinen Versuchen nicht gesetzt \\
  projektbezeichnung & Projektbezeichnung \\
  versuchsbezeichnung & Eine Zeichenkette, die vergeben werden kann; bei meinen Versuchen nicht gesetzt \\
  maschinenbediener & Ein Name, der vergeben werden kann; bei meinen Versuchen nicht gesetzt \\
  material & Das benutzte Material; bei meinen Versuchen nicht hier erfasst \\
  werkzeug & Eine Zeichenkette, die vergeben werden kann; bei meinen Versuchen nicht gesetzt \\
  fliesszahl & wird aus Druckkurve berechtnet bar*sek\\
  einspritzarbeit & Die Arbeit, die von der Schneckenweg der Battenfeld verrichtet wird \\
  max\_einspritzdruck & Zielwert Maximaler Einspritzdruck, bei weggesteuerter Einspritzphase \\
  umschaltdruck & ergibt sich aus umschaltweg Umschaltdruck \\
  --umschaltweg & Einstellparameter für die schnecke\\
  umschaltpunkt & Zeitpunkt der sich ergibt\\
  --nachdruckarbeit & Die Arbeit, die von der Hubeinheit der Battenfeld in der Nachdruckphase geleistet wird \\
  integral\_nachdruck & Integral des Nachdrucks, ein Wert, der theoretisch ist \\
  avg\_nachdruck & Durchschnittlicher Nachdruck, konstanet wert gleich einstellparamt \\
  massepolster & Massepolster, die Masse an Schmelze, die zwischen der Schnecke und der Düse ist \\
  einspritzzeit & Die Zeit, die die Hubanlage braucht, um bei der weggesteuerten Einspritzphase bis der Umschaltpunkt erreicht ist \\
  dosierzeit & consDie Zeit, die von der Schnecke benötigt wird, um das Material aufzudosieren \\
  --dosierweg & Der Weg, den die Schnecke nach hinten gedrückt wird, beim Aufdosieren des Materials \\
  --kuhlzeit & Einstellwert, der getestet wird, bis das Teil ohne Schaden aus der Kavität entnommen werden kann \\
  zykluszeit & Die Zeit, die für einen vollständigen Spritzgusszyklus benötigt wird \\
  --mittlere\_drehzahl & Mittlere Drehzahl der Schnecke \\
  --staudruck & Der Druck, um das Schmelzmaterial wärend dem dosieren\\
  avg\_einspritzgeschwindigkeit & Durchschnittliche Einspritzgeschwindigkeit \\
  max\_einspritzgeschwindigkeit & Maximale Einspritzgeschwindigkeit \\
  max\_wid & Werkzeug innen Drucksensor wo am bauteil am dienstag\\
  umschaltdruck\_wid & Umschaltdruck des Sensors, wie ist der mit dem anderen Umschaltdruck verwandt? \\
  fliesszahl\_wid & gleicher integral \\
  integral\_wid & Integral des Werkzeuginnendrucks \\
  temperatur\_duse & Temperatur gemessen der Düse \\
  temperatur\_zylinder\_zone\_1 & Einstelltemperatur der Heizelemente in der Zylinderzone 1 \\
  temperatur\_zylinder\_zone\_2 & Einstelltemperatur der Heizelemente in der Zylinderzone 2 \\
  temperatur\_flansch & gekühlt 50 grad Gemessene Temperatur des Flanschs \\
  temperatur\_werkzeugheizung\_zone\_1 & Heisskanal Einstellwerte, nicht vorhanden im Unihokayballwerkzeug \\
  temperatur\_werkzeugheizung\_zone\_2 & Heisskanal Einstellwerte, nicht vorhanden im Unihokayballwerkzeug \\
  temperatur\_werkzeugheizung\_zone\_3 & Heisskanal Einstellwerte, nicht vorhanden im Unihokayballwerkzeug \\
  temperatur\_werkzeugheizung\_zone\_4 & Heisskanal Einstellwerte, nicht vorhanden im Unihokayballwerkzeug \\
  umgebungstemperatur & Gemessener Wert in der Nähe der Battenfeld Umgebungstemperatur \\
  luftfeuchtigkeit & Gemessener Wert in der Nähe der Battenfeld Luftfeuchtigkeit \\
  leistungsaufnahme\_gesamt & Der Battenfeld Gesamtleistungsbedarf \\
  gewicht & Des Bauteils, an diesem Zeitpunkt nicht integriert in die .dat der Battenfeld \\
  seriennummer & Seriennummer der Halbschale, ein Zeitstempel \\
  \hline
\end{longtable}
