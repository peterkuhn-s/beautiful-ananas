\subsection{Allgemeine Ausgangslage}
Aktuell werden Qualitätsdaten von Spritzgussteilen erst 16-24 Stunden nach der Produktion gemessen, und es fehlt ein Modell zur Vorhersage dieser Daten. Die Simulation des Spritzgiessprozesses kann die Bauteilgeometrie berechnen, hängt aber von Materialdaten und Prozessparametern ab. Die starke Wettbewerbssituation zwingt Unternehmen zur Suche nach neuen Qualitätsüberwachungs- und -kontrollsystemen. Dies betont die Notwendigkeit, innovative Lösungen zu entwickeln, um die Effizienz und Wettbewerbsfähigkeit in der Branche zu steigern.

Im Mittelpunkt des aktuellen Projekts in der SmartFactory @ OST stehen die hochaktuellen Themen Digitalisierung, intelligente Fabrik und Industrie 4.0.

\subsection{Aufgabenstellung der SA}
Im Rahmen der Semesterarbeit wird ein Korrelationsmodell entwickelt und validiert. Dieses Modell soll in der Lage sein, die Qualitätsdaten die 24 Stunden nach der Produktion messbar sein würden, auf Grundlage der Prozessdaten bereits unmittelbar nach der Herstellung zu schätzen. Die Korrelation wird unter Verwendung eines definierten Setups von Maschine, Werkzeug und Kunststoff mithilfe statistischer Versuchsplanung erarbeitet.

Des Weiteren wird die Übertragbarkeit der gewonnenen Erkenntnisse und Modelle auf andere Materialien überprüft.

\subsection{Position der SA im Inno Swiss Projekt}
Die Semesterarbeit ist die Erste, die mit dem fertig gestellten Funktionsmuster von Kistler arbeitet. Es sollen erste ML Modelle mit Daten erstellt und validiert werden. Meine Semesterarbeit hat nicht den Anspruch, dass alle Komplikationen gelöst werden.

Der Antrag zum Inno Swiss Projekt wurde im Sommer 2023 vom IKW gestellt.

\subsection{Vorgehen der Planung der SA}
Die Semesterarbeit wurde anfangs nach einem klassischen Wasserfallmodell geplant. In den ersten Wochen ist mir aufgefallen, dass ein Grossteil meiner Arbeit Software Entwicklung ist. In der Software Entwicklung ist eine agiele Planung weit verbreitet. Somit habe ich angefangen, meine Semesterarbeit in einem Hybriden Modell zu planen. Dieser Hybride Ansatz entspricht im Ansatz einem agilen HERMES Projektplan. Mit dem Agilen Entwicklungsteil konnte ich mich jeweils auf einige wichtige Ergebnisse konzentrieren. In \ref{KanbanBoard} ist das Kanban Board abgebildet zum Zeitpunkt des Endes meiner Semesterarbeit. Bei der Entwicklung der Software wurde jeweils ein minimal viable Product geschrieben, getestet und gegeben falls verbessert.


