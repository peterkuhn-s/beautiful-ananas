\subsection{Persönliche Erfahrung}
Ich hatte grosse Freude an dieser Arbeit, denn ich konnte die Theorie aus den Vorlesungen Wahrscheinlichkeitsrechnung, Datenbanksysteme und Statistical Machine Learning direkt bei einer Kunststoffanwendung umsetzen.

Ich bin zufrieden mit der Verteilung der geleisteten Arbeit. Ich habe 82 Skripts geschrieben und eingesetzt. Ich habe gegen Ende des Semesters mehr Arbeitstunden notiert, aber die geleisteten 310 Stunden sind über das ganze Semester verteilt. Die genaue Verteilung der geloggten Zeiten kann an diesem Pfad gefunden werden: /.git/HEAD.

Die Swiss AI Initiative zeigt mir, dass das IWK mit dieser Arbeit auf dem Weg in die Zukunft ist.

Durch meine praktische Arbeit und das Programieren habe ich einen Einblick in ein beeindruckendes Forschungsprojekt des IWKs bekommen.

\subsection{Fazit}
Ich bin sehr zufrieden mit der gesamten Arbeit. Da die Arbeit ein agiles Software Projekt war, bestand das Risiko, dass wir noch kein verwertbares Ergebnis erreichen. Erfreulicherweise konnte das Funktionsmuster der Messzelle in Betrieb genommen werden und 1000 Messungen durchführen. Die Datenbank, die ich aufgebaut habe funktioniert gut. Es ist gelungen, alle generierten Datenpunkte an einem Ort zu haben, ohne die Übersicht zu verlieren. Die ML-Modelle, die ich in der Semesterarbeit angewendet habe, haben brauchbare Ergebnisse gebracht. Die Trainingsdaten für die ML-Modelle konnten erzeugt werden. Die Validierung der Modelle konnte auf mehr Varianten durchgeführt werden als bei einer klassischen ML-Aufgabe. Ich bin zufrieden, dass ich sowohl die Datenverarbeitung mit ML als auch die Datengenerierung kontrollieren konnte und geordnet weitergebe. 

\subsection{Ausblick}
Die erstellten Skripte sind das minimal viable Produkt. Ein Refactoring ist empfehlenswert, um die Anwendererfahrung zu verbessern. Eine Auslagerung auf einen Server des IWK wäre denkbar. Die Skripte sind der Übersicht halber nicht miteinander verknüpft. Eine Anwendung, die die verschiedenen Aufgaben steuern kann, wäre erstrebenswert. Es sollten auch noch mehr Metadaten über die Halbschalen gesammelt und zusammengeführt werden. Zum Beispiel Materialkennwerte, Versuchszugehörigkeiten, benutzte Handlingsysteme, usw.

Die Datenbank kann jetzt mit Ansichten erweitert werden, um das Abrufen der gewünschten Daten zu vereinfachen.

Ein DoE, bei dem nicht die Einstellparameter aktiv geändert werden, könnte die Datengrundlage für ein Modell liefern, das näher an dem Endanwendungsbeispiel liegt. So könnte es möglich sein, ein Modell zu trainieren, das einzelne Halbschalen korrekt vorhersagt.

Sobald das Handlingsystem komplett von Robotern ausgeführt wird, werden die Modelle zur Vorhersage des Aussendurchmessers besser, da der menschliche Faktor bei gequetschten Halbschalen entfällt.

Das Random Forest Modell hat überraschend gute Vorhersagen gemacht. Es sollte genauer untersucht werden, wie dieses Modell die Vorhersagen generiert. Eine Möglichkeit wäre, die Inputdaten so weit zu reduzieren, bis das Modell keine guten Vorhersagen mehr machen kann.

Die Nutzung von der Lasso Regression ist zu empfehlen, da bei einer Lasso Regression die Inference möglich ist.
