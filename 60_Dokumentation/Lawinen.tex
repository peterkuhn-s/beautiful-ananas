Jedes Jahr sterben in der Schweiz ca. 10 Menschen durch Lawinen, wobei 8 der Todesfälle durch Schneebrettlawinen, und durch Gleitlawinen verursacht werden. Mit dem Klimawandel ändern sich Häufigkeit und Eigenschaften von Lawinen. Im Gegensatz zur Schneebrettlawine können Gleitlawinen bisher kaum frühzeitig erkannt und auch nicht präventiv durch Detonation ausgelöst werden.

\iffalse
Jedes Jahr sterben in der Schweiz ca. 10 Menschen durch Lawinen, wobei 8 durch Schneebrettlawinen und 2 durch Gleitlawinen verursacht werden. Mit dem Klimawandel ändern sich die Eigenschaften von Gleitlawinen. Diese können nicht präventiv durch Detonation ausgelöst werden und sind zeitlich schwer vorherzusagen.


jedes jahr 10 Tote. 8 schneebrettlawine. 2 Gleitlawinen.

mit Klimawandel änders sich Gleitlawinien. nicht preventiv mit einer Detonation auslösbar. nicht zeitlich vorhersagbar.


\section{Entstehung der Gleitlawine}



\section{Endziel der Arbeit}

Das Ziel dieser Arbeit ist es, den Schaden durch Gleitlawinen zu verringern. Dies soll durch die Entwicklung eines Sensors zur Messung der Schneefeuchtigkeit erreicht werden, da diese ein entscheidender Parameter für die Abschätzung der Lawinengefahr ist.

\section{User Story}

Um die Aufgabe der Produktentwicklung besser zu verstehen, wurden verschiedene User Stories erstellt. Eine dieser Stories beschreibt die Anwendung des entwickelten Sensors:

Alice führt an einem Hang eine Schneedeckenanalyse durch. Neben ihrer subjektiven Beurteilung trägt sie die Messwerte der Schneefeuchtigkeit ein, die sie mit dem neuen Sensor ermittelt hat.

\section{Anforderungen}

Es wurde eine Liste von Anforderungen erstellt, die das fertige Produkt erfüllen soll:
\begin{itemize}
    \item Die Methode soll eine Anzeige haben, die feststellt, wann eine Gleitlawine bevorsteht.
    \item Die Methode soll unabhängig von der Dichte des Schnees funktionieren.
    \item Der Messbereich des LWC (Liquid Water Content) soll von 1 \% bis 7 \% abgedeckt werden.
    \item Die Methode soll für einen Hang in der Schweiz einsetzbar sein.
\end{itemize}

Bis zum Abschluss der Arbeit konnten diese Anforderungen nur eingeschränkt bestätigt werden. Für eine aussagekräftige Statistik sind noch mindestens 1000 Messungen erforderlich.

\section{Planung der Arbeit}

Die Arbeit gliedert sich in drei Teile:
\begin{enumerate}
    \item In einer Vorstudie werden verschiedene physikalische Prinzipien zur Messung des LWC theoretisch und praktisch verglichen.
    \item Es wird ein Funktionsmuster gebaut, basierend auf dem vielversprechendsten physikalischen Prinzip. Dieser Teil wird nach agiler Hardware-Entwicklung mit einem Kanbanboard geplant.
    \item Der dritte Teil beschreibt die Dokumentation der Produktentwicklung und die entwickelte Messtechnik.
\end{enumerate}
\fi
