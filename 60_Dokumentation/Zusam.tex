Diese Arbeit untersucht den Liquid Water Content (LWC) im Schnee, ein entscheidender Parameter, der die Struktur, Stabilität und das Verhalten von Schneedecken beeinflusst und damit für die Vorhersage von Lawinen, insbesondere von Gleitschneelawinen, von entscheidender Bedeutung ist.

Ihr Hauptziel besteht in der Entwicklung eines Sensors zur präzisen Messung des Liquid Water Content (LWC) von Schnee. Die Einleitung bietet eine grundlegende Einführung in die physikalischen Eigenschaften von Schnee sowie die spezifischen Risiken von Lawinen.

Im theoretischen Hintergrund wird die Schneemetamorphose erläutert und ihr Einfluss auf die Stabilität von Schneedecken diskutiert, wobei besondere Aufmerksamkeit den Gleitschneelawinen gilt, die durch das Abgleiten von homogenen Schneemassen auf glatten Oberflächen entstehen.

Zahlreiche wissenschaftliche Veröffentlichungen befassen sich mit verschiedenen Methoden zur LWC-Messung, darunter elektrische, dielektrische, optische und thermische Techniken. Eine Vorstudie identifizierte das vielversprechendste Prinzip zur LWC-Messung, wobei fünf verschiedene Messmethoden getestet wurden.

Das Water Indicator Tape, ursprünglich für die Qualitätssicherung in der Elektronik entwickelt, wurde mittels agiler Hardware-Entwicklung zu einem Messsystem weiterentwickelt. In fünf Iterationen wurde der Messablauf optimiert, was zu einem zuverlässigen System führte, das nicht nur den LWC, sondern auch geometrische Eigenschaften des Schnees erfassen kann. Dies übertrifft die Fähigkeiten kommerzieller Produkte und ermöglicht umfassendere Analysen der Schneemetamorphose und Lawinengefahr. Mit weiteren Verbesserungen kann das System zukünftig die Messgenauigkeit und Anwendungsbreite erhöhen und detailliertere Einblicke in die physikalischen Prozesse von Schneedecken und Lawinenentwicklungen bieten.

Der iterative Entwicklungsprozess des Sensors wird beschrieben, wobei agile Hardware-Entwicklungsmethoden und ein Kanban-Board eingesetzt werden, um die Herausforderungen und Lösungen detailliert zu dokumentieren.

Die Ergebnisse der Feldversuche werden analysiert, wobei die Messergebnisse verschiedener Andruckzeiten und LWC-Werte verglichen werden, um die Zuverlässigkeit des entwickelten Systems zu bewerten.

Abschließend werden Vorschläge für die Weiterentwicklung des Messsystems präsentiert, um die Messgenauigkeit und Anwendungsbreite zu verbessern und weitere Forschungsarbeiten zu unterstützen.
