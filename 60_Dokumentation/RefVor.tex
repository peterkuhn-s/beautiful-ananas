In der Aufgabenstellung stand unter anderem, dass der zu entwickente Sensor unabhänigh von Grössen wie Dichte des Schnees sein sollen.

In den getestent fünf sehr unterschiedlichen physikalischen Prinzipien konnten drei gefunden werden die eine Interaktion mit Schnee darstellen konnten. Die Methoden sind, so wie ich sie mir vorstelle, von der Geometie des Schnees stark beeinflusst. Das ist eine Eigenschaft des Schnees die noch schwieriger zu messen ist, wie die Dichte.

Der LWC hat einen Enormen Einfluss auf den Schnee, aber die Geschichte des Schees hat ebenfalls eine grossen Einfluss auf alle Eigenschaften des Schnees.


In der Aufgabenstellung wurde erwähnt, dass der zu entwickelnde Sensor unabhängig von Grössen wie der Dichte des Schnees funktionieren soll.

Von den fünf getesteten physikalischen Prinzipien konnten drei eine Interaktion mit Schnee erziehlen. Diese Methoden sind jedoch stark von der Geometrie des Schnees beeinflusst, was eine noch schwierigere Messgrösse darstellt als die Dichte.

Der LWC hat einen enormen Einfluss auf den Schnee, aber auch die Vorgeschichte des Schnees spielt eine bedeutende Rolle für dessen Eigenschaften.

Somit wurden, in Absprache mit dem Betreuer, die Anforderungskriterene für Sensor gelockert, mit dem Ziel einen innovativen Ansatz verfolgen zu können.
