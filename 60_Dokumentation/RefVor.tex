In der Aufgabenstellung wurde erwähnt, dass der zu entwickelnde Sensor unabhängig von Grössen wie der Dichte des Schnees funktionieren soll.

Von den fünf getesteten physikalischen Prinzipien konnten drei eine Interaktion mit Schnee erzielen. Diese Methoden sind jedoch stark von der Geometrie des Schnees beeinflusst, was eine noch schwierigere Messgrösse darstellt als die Dichte.

Der LWC hat einen enormen Einfluss auf den Schnee, aber auch die Vorgeschichte des Schnees spielt eine bedeutende Rolle für dessen Eigenschaften.

Somit wurden, in Absprache mit dem Betreuer, die Anforderungskriterien für den Sensor gelockert, um einen innovativen Ansatz verfolgen zu können.
