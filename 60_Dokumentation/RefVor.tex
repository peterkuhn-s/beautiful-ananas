In der Aufgabenstellung wurde erwähnt, dass der zu entwickelnde Sensor unabhängig von Grössen wie der Dichte des Schnees funktionieren soll.

Von den fünf getesteten physikalischen Prinzipien, zeigten drei einen erfolgreichen Ansatz um den LWC abzubilden. Diese Methoden sind jedoch stark von der Geometrie des Schnees beeinflusst, was eine noch schwierigere Messgrösse darstellt als die Dichte.

Der LWC hat einen enormen Einfluss auf die Eigenschaften Schnee. Aber die gesamte Vorgeschichte des Schnees spielt eine bedeutende Rolle für dessen Eigenschaften.

Somit wurden, in Absprache mit dem Betreuer, die Anforderungskriterien (Dichteunabhängikeit der Messung) für den Sensor gelockert, um einen innovativen Ansatz verfolgen zu können.
