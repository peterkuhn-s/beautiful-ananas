
Von den sechs getesteten physikalischen Methoden, zeigten drei einen erfolgreichen Ansatz um den LWC abzubilden. Diese Methoden sind jedoch stark von der Geometrie des Schnees beeinflusst, was eine noch schwierigere Messgrösse darstellt als die Dichte.

Der LWC hat einen enormen Einfluss auf die Eigenschaften des Schnee. Das spiegelt sich in der Vielfallt an möglichen Messmethoden wieder. Die gesamte Vorgeschichte und Metamorphose des Schnees spielt ebenfalls eine bedeutende Rolle für dessen Eigenschaften.

In der Aufgabenstellung \ref{sec:aufgabe} wurde erwähnt, dass der zu entwickelnde Sensor unabhängig von Grössen wie der Dichte des Schnees funktionieren soll. Siehe Kapitel \ref{sec:Meta}.


Es ist nicht sicher, ob die getesteten Methoden dichteunabhängig sind. Daher wurden, in Absprache mit dem Betreuer, die Anforderungskriterien bezüglich der Dichteunabhängigkeit der Messung für den Sensor gelockert, um einen innovativen Ansatz verfolgen zu können.
