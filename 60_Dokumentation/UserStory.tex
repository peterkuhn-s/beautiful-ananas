

Die User Stories werden genutzt, um sich früh Gedanken über den Einsatz des fertigen Produkts zu machen.

\begin{itemize}

\item Alice führt an einem Hang eine Schneedeckenanalyse durch. Sie hat alles Material und die Messgeräte in ihrem Rucksack mitgebracht. Mit einer Schaufel gräbt sie einen Schneegraben für die Messungen. Neben ihren üblichen Messungen und ihrer subjektiven Beurteilung trägt sie noch die Messwerte der Schneefeuchtigkeit in das Protokoll ein.

\item Barbara sitzt in der Einsatzzentrale an ihrem Computer und sieht eine Warnung aufleuchten. Die Warnung wurde von den Sensor im Lawinenhang ausgelöst. Sie ruft sofort bei der Rhätischen Bahn an und kann den Zug stoppen, bevor er von der Lawine erfasst wird.

\item Chloe führt eine neue Simulation durch. Die Simulation berechnet aus Meteorologiedaten den LWC und somit die Lawinengefahr in der Schweiz. Dazu benutzt sie die neuen Trainingsdaten des vergangenen Jahres, die mit den Sensoren aufgezeichnet wurden.

\item Dorothea überlegt, ob sie an diesem Hang mit ihren Skiern eine Abfahrt wagen soll. Mit einem handlichen Grerät überprüft sie schnell die Schneefeuchtigkeit und kann so eine sichere Entscheidung treffen.

\item Ester wirft aus dem Helikopter das Sensorpaket, um den Hang, der sonst nicht zu erreichenden ist, zu überwachen. In sechs Monaten wird das Paket im abgetauten Hang wieder eingesammelt.

\item Greta trainiert ihr Reinforcement Maschien Learning Modell auf Bildern von Schneeflocken. Dazu braucht sie hochauflösende Bilder und den dazugehörigen LWC der Probe.

\end{itemize}

Die unterschiedlichen User Stories beschreiben komplett unterschiedliche Produkte. Da noch nicht entschieden werden kann, welche die korrekte Anwendung ist, wird zuerst die unterschiedlichen Methoden erkundet. Sobald eine Methode gefunden ist, wird diese zu einer konkreten Anwendung ausgearbeitet.

Zu diesem Zeitpunkt ist der Einsatz weiterer abstrakter Planungstechniken, wie zum Beispiel Black Box und Musskriterien, nicht sinnvoll, da dadurch spannende Möglichkeiten ausgeschlossen werden könnten.
\iffalse
Jede weitere abstrakte Planung, wie zum Beispiel Black Box und Musskriterien, macht somit noch keinen Sinn, jetzt schon definiert zu werden, da innovative Produkte während der Arbeit damit eingeschränkt würden.




Um die Aufgabe der produktentwicklung besser zu verstehen, wurden User storys geschrieben. In \ref{userstoryvoll} sind alle 6 User Storys.

das ziel der Userstorys ist es fruh sich gedanken uber das fertige produkt zu machen. Hier ist die User story die das schlussendlichten entwickelten sensor beschreibt.

Alice macht an einem hang einen schedeckenanalyse, mit der schaufeln. neben ihrer subkektiven beurteilung tragt sie noch die messwerte der schneefeuchtikeit ein.


Die Userstorys beschreiben komplett unterschiedliche Produkte. da ich nicht entscheiden kann oder will, welches die korrekte Anwendung ist, werde ich zu erst die Technologie erkunden und dann die anwendung in ein Produkt finden.

jede weiteren pflichenheft aktivitaten, wie zum beispiel black box, Musskriteren, usw. machen keinen sinn jetzt schon definiert zu werden. da spanneden Entdecknugen währen der arbeit damit eingeschränkt werden.

\fi
