Um die Aufgabe des Produkts besser zu verstehen, wurden User Stories geschrieben. In Kapitel \ref{userstoryvoll} sind alle 6 User Storys zu lesen.

User Stories werden genutzte, um sich früh Gedanken über den Einsatz des fertige Produkt zu machen. Hier ist die User story, die das schlussendlich entwickelte Funktionsmuster beschreibt:

Alice macht in einem Hang einen Schedeckenanalyse. Sie hat alles Material und Messgeräte in ihrem Rucksack mitgebracht. Mit einer Schaufeln gräbt sie einen Schneegraben. Neben ihrer üblichen Messungen und ihrer subjektiven Beurteilung trägt sie noch die Messwerte der Schnee Feuchtigkeit in das Protokoll ein.

Die unterschiedlichen User Stories beschreiben komplett unterschiedliche Produkte. Da noch nicht entscheiden werden kann, welche die korrekte Anwendung ist, werden zuerst die Technologie erkundet. Sobald eine Technologie gefunden ist wird diese zu einer kornkreten Anwendung ausgearbeitet.

Jede weitere abstrakte Planung, wie zum Beispiel Black Box und Musskriterien machen somit noch keinen Sinn, jetzt schon definiert zu werden, da innovative Produkte während der Arbeit damit eingeschränkt würde.



\iffalse

Um die Aufgabe der produktentwicklung besser zu verstehen, wurden User storys geschrieben. In \ref{userstoryvoll} sind alle 6 User Storys.

das ziel der Userstorys ist es fruh sich gedanken uber das fertige produkt zu machen. Hier ist die User story die das schlussendlichten entwickelten sensor beschreibt.

Alice macht an einem hang einen schedeckenanalyse, mit der schaufeln. neben ihrer subkektiven beurteilung tragt sie noch die messwerte der schneefeuchtikeit ein.


Die Userstorys beschreiben komplett unterschiedliche Produkte. da ich nicht entscheiden kann oder will, welches die korrekte Anwendung ist, werde ich zu erst die Technologie erkunden und dann die anwendung in ein Produkt finden.

jede weiteren pflichenheft aktivitaten, wie zum beispiel black box, Musskriteren, usw. machen keinen sinn jetzt schon definiert zu werden. da spanneden Entdecknugen währen der arbeit damit eingeschränkt werden.

\fi
