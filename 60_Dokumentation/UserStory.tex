Um die Aufgabe des Produkts besser zu verstehen, wurden User Stories geschrieben. In Kapitel \ref{userstoryvoll} sind alle 6 User Storys zu lesen.

User Stories werden genutzte, um sich früh Gedanken über den Einsatz des fertige Produkt zu machen. Hier ist die User story, die das schlussendlich entwickelte Funktionsmuster beschreibt:

Alice macht in einem Hang einen Schedeckenanalyse. Sie hat alles Material und Messgeräte in ihrem Rucksack mitgebracht. Mit einer Schaufeln gräbt sie einen Schneegraben. Neben ihrer üblichen Messungen und ihrer subjektiven Beurteilung trägt sie noch die Messwerte der Schnee Feuchtigkeit in das Protokoll ein.


Alice macht in einem Hang einen Schedeckenanalyse. Sie hat alles Material und Messgeräte in ihrem Rucksack mitgebracht. Mit einer Schaufeln gräbt sie einen Schneegraben. Neben ihrer üblichen Messungen und ihrer subjektiven Beurteilung trägt sie noch die Messwerte der Schnee Feuchtigkeit in das Protokoll ein.

Barbara sitzt an in der Einsatzzentrale ihrem computer und sieht eine Warnung aufleuchtet. Sie ruft  sofort be der ratischen bahn an und kann den zo so stoppen bevor er von der lawine erfasst wird.

Chloe macht eine neue simulation um die vorhersage der scheefeuchtikeit aus den meteodaten zu machen. dazu benutzt sie die neue trainingsdaten des vergangene Jahres 

Dorothea weiss nicht, ob sie an diesem Hang eine Abfahrt wagen soll. mit ihren skiern. Sie überprüft mit ihrem handlichen Messgerät schnell die Scheefeuchtigkeit an und kann so eine sicher Entscheidung treffen.

Ester wirft aus dem Helikopter das Sensorpaket, um den Hang zu überwachen. In sechs Monaten wird das Paket im abgetauten Hang wieder eingesammelt.

Greta trainiert ihr reinforced ML Modell auf Bildern von Schneeflocken. Dazu braucht sie hochauflösende Bilder und den dazugehörigen LWC der Probe.


Die unterschiedlichen User Stories beschreiben komplett unterschiedliche Produkte. Da noch nicht entscheiden werden kann, welche die korrekte Anwendung ist, werden zuerst die Technologie erkundet. Sobald eine Technologie gefunden ist wird diese zu einer kornkreten Anwendung ausgearbeitet.

Zu diesem Zeitpunkt ist der Einsatz weitere abstrakte Planungstechniken, wie zum Beispiel Black Box und Musskriterien nicht sinnvoll, weil daruch spanneden Möglichkeiten ausgeschlossen werden würden.

Jede weitere abstrakte Planung, wie zum Beispiel Black Box und Musskriterien machen somit noch keinen Sinn, jetzt schon definiert zu werden, da innovative Produkte während der Arbeit damit eingeschränkt würde.



\iffalse

Um die Aufgabe der produktentwicklung besser zu verstehen, wurden User storys geschrieben. In \ref{userstoryvoll} sind alle 6 User Storys.

das ziel der Userstorys ist es fruh sich gedanken uber das fertige produkt zu machen. Hier ist die User story die das schlussendlichten entwickelten sensor beschreibt.

Alice macht an einem hang einen schedeckenanalyse, mit der schaufeln. neben ihrer subkektiven beurteilung tragt sie noch die messwerte der schneefeuchtikeit ein.


Die Userstorys beschreiben komplett unterschiedliche Produkte. da ich nicht entscheiden kann oder will, welches die korrekte Anwendung ist, werde ich zu erst die Technologie erkunden und dann die anwendung in ein Produkt finden.

jede weiteren pflichenheft aktivitaten, wie zum beispiel black box, Musskriteren, usw. machen keinen sinn jetzt schon definiert zu werden. da spanneden Entdecknugen währen der arbeit damit eingeschränkt werden.

\fi
