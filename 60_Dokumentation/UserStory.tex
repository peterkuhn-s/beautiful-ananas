Um die Aufgabe der produktentwicklung besser zu verstehen, wurden User storys geschrieben. In \ref{userstoryvoll} sind alle 6 User Storys.

das ziel der Userstorys ist es fruh sich gedanken uber das fertige produkt zu machen. Hier ist die User story die das schlussendlichten entwickelten sensor beschreibt.

Alice macht an einem hang einen schedeckenanalyse, mit der schaufeln. neben ihrer subkektiven beurteilung tragt sie noch die messwerte der schneefeuchtikeit ein.


Die Userstorys beschreiben komplett unterschiedliche Produkte. da ich nicht entscheiden kann oder will, welches die korrekte Anwendung ist, werde ich zu erst die Technologie erkunden und dann die anwendung in ein Produkt finden.

jede weiteren pflichenheft aktivitaten, wie zum beispiel black box, Musskriteren, usw. machen keinen sinn jetzt schon definiert zu werden. da spanneden Entdecknugen währen der arbeit damit eingeschränkt werden.

