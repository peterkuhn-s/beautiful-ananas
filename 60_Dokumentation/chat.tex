

Die Untersuchung und Messung des LWC im Schnee erweist sich als herausfordernd aufgrund der komplexen Eigenschaften und der Inhomogenität des Schnees. Der LWC ist ein kritischer Parameter, der die physikalischen Eigenschaften des Schnees, wie zum Beispiel Dichte, Wärmeleitfähigkeit und mechanische Stabilität, stark beeinflusst.

In dieser Arbeit wurden folgende fünf verschiedene Methoden zur Messung des LWC getestet:

\begin{enumerate}
    \item Phasenübergang ausgelöst durch Vibration
    \item Elektrischer Widerstand
    \item Diffusion von Flüssigkeit
    \item Refraktion und Reflexion eines Lasers
    \item Water Indicator Tape
\end{enumerate}

Das Water Indicator Tape wurde dann mit der Methodik des agilen Hardware Developments zu einem Messsystem entwickelt. Das Tape stammt ursprünglich aus der Qualitätssicherung in der Elektronik und wird verwendet, um das Eindringen von Wasser nachzuweisen. Bei der Messung mit dem Tape gibt es Hinweise, dass das Tape nicht nur den LWC, sondern auch Informationen über die Geometrie des Schnees liefern kann. Dies eröffnet neue Perspektiven für die Messung und Analyse von Schnee.

Insgesamt hat die Arbeit gezeigt, dass das entwickelte Messsystem ein vielversprechendes Werkzeug zur Bestimmung des LWC darstellt. Durch kontinuierliche Verbesserungen und Anpassungen kann es in zukünftigen Forschungsarbeiten weiter optimiert und verfeinert werden, um die Messgenauigkeit, Messpräzision und Anwendungsbreite zu erhöhen.
