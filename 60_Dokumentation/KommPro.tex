

Heute gibt es kommerziell erhältliche Produkte, die den LWC von Schnee messen. Die Produkte nutzen die unterschiedliche dielektrische Konstante von Eis und Wasser. Hier zu erwähnen sind der SLF Snow Sensor, auch Denothmeter genannt und die Finnish Snow Fork. Sensoren aus dem Agrikultur Bereich, die die Bodenfeuchtigkeit messen, sind auch im Schnee einsetzbar.

Ein Nachteil der Produkte ist, dass, um auf einen prozentualen LWC zu kommen, die Dichte des Schnees separat gemessen werden muss. Die räumliche Auflösung der Produkte ist im Bereich von Zentimetern.
