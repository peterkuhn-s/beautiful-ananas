Die Methode der Diffusion beobachtet, wie sich ein Stoff im Schnee ausbreitet. Für den Vorversuch wurde der Schnee unter ein Stereo-Mikroskop platziert. Der Versuch dauert etliche Minuten. Um zu verhindern, dass der Schnee von der warmen Raumluft aufgeschmolzen wird, ist der Schnee in einer Röhre aus Eis platziert. Während das -10 Grad Celsius kalte Eis langsam schmilzt, kann der Versuch durchgeführt werden. In der Abbildung \ref{fig:DiffMess} ist der Versuchaufbau dargestellt. Die Auswertung bei dem Vorversuch erfolgt visuell, indem beobachtet wird, wie sich blaue Tinte im Schnee ausbreitet.

Eine Kombination dieses Ansatzes mit der Leitfähigkeitsmessung (siehe \ref{sec:Volt}) ist möglich, wenn ein leitfähiger Stoff eingesetzt wird.

Ich vermute, dass dieser Ansatz von der Geometrie des Schnees beeinflusst wird. Der LWC ist wahrscheinlich weniger einflussreich als die Geometie der Eiskristalle.

\begin{figure}[H]
    \centering
    \includegraphics[width=0.8\textwidth]{Bilder/freistellen.jpeg}
    \caption{Aufbau einer Messung wobei der Schnee durch eine Eis Röhre gekühlt wird}
    \label{fig:DiffMess}
\end{figure}
