


in dieser produktentwicklungsaufgabe wurde eine Innovativer sensor entwickelt um den Liquid water Contet von schnee zu messen.

dazu wurden verschiedene physikalische wirkprinzipien getesten. die vielversprechendste technik in der ein Water indikator Tape aus der qualitassicherung im elektronikbrachnche visuell auswertet wird, wurde uber 5 iterationen entwickelnt um die interaktion des schnees und Taps zu verstehen.


Die messmethode zeigt die fahigkeiten den lwc von schnee zu erfassen. bis jetzt ist es aber noch nicht sicher ob die prazision und genauigkeit ausreichend ist um in ein produkt umgesetzt zu werden.


In dieser Arbeit wurde ein innovativer Sensor zur Messung des Flüssigwassergehalts (Liquid Water Content, LWC) in Schnee entwickelt. Verschiedene physikalische Prinzipien wurden getestet, um die beste Methode zur Bestimmung des LWC zu identifizieren. Die vielversprechendste Technik erwies sich als der Einsatz eines Wasserindikatorbands aus der Qualitätssicherung in der Elektronikbranche, welches visuell ausgewertet wird. Über fünf Iterationen hinweg wurde der Sensor weiterentwickelt, um die Interaktion zwischen Schnee und dem Indikatorband besser zu verstehen.

Die Auswertung erfolgt durch visuelle und digitale Analyse des 3M 5559 Water Indikator Tapes. Das Tape, das bei Kontakt mit Wasser rot wird, wird für definierte Zeitspannen auf die Schneeoberfläche gelegt und anschließend fotografiert. Die visuelle Beurteilung erfolgt durch die einfache Betrachtung der roten Verfärbung auf dem Tape. Für eine präzisere Analyse wird die Bildverarbeitung eingesetzt, bei der Software den Anteil der roten zu weißen Fläche berechnet, um den Flüssigwassergehalt zu quantifizieren.

Die entwickelte Messtechnik zeigt die Fähigkeit, den LWC im Schnee zu erfassen, jedoch ist die Präzision und Genauigkeit der Messungen noch nicht ausreichend, um den Sensor als marktfähiges Produkt zu etablieren. Weitere Optimierungen und umfangreiche Tests sind erforderlich, um die Zuverlässigkeit und Praktikabilität des Sensors zu gewährleisten. Die Ergebnisse dieser Arbeit liefern einen wichtigen Beitrag zur Weiterentwicklung von Messmethoden für den LWC im Schnee und könnten zukünftig zur Verbesserung der Vorhersage und Prävention von Gleitschneelawinen beitragen.
