Die vorliegende Arbeit befasst sich mit der Herausforderung, den Anteil von flüssigem Wasser im Schnee, auch Liquid Water Content (LWC) genannt, im Schnee zu messen. Der LWC ist ein wesentlicher Parameter, der die physikalischen Eigenschaften von Schnee und damit das Verhalten von Schneedecken maßgeblich beeinflusst. Durch ein genaues Verständnis dieses Parameters wird die Vorhersage von Lawinen verbessert, insbesonders die Vorhersage von Gleitschneelawinen, die auf dem Untergrund abgleiten. Ziel der Arbeit ist es, eine innovative Methode zur Bestimmung des LWC zu entwickeln, die über die Möglichkeiten derzeitiger kommerzieller Produkte hinausgeht.

Zur Erreichung dieses Ziels wurde ein systematisches Vorgehen gewählt, das eine Vorstudie zur Identifikation der vielversprechendsten Messtechniken einschloss. Sechs unterschiedliche Prinzipien wurden getestet: Mechanische Anregung durch Vibration, elektrischer Widerstand, Diffusion von Flüssigkeit, Laser Refraktion und Reflexion sowie Water Indicator Tape. Das Water Indicator Tape erwies sich als am geeignetsten und wurde mittels agiler Hardware-Entwicklung zu einem funktionierenden Messsystem weiterentwickelt. Dieser iterative Entwicklungsprozess umfasste fünf Iterationen, in denen der Messablauf kontinuierlich optimiert wurde.

Die Ergebnisse der Messungen mit dem entwickelten System sind vielversprechend. Das Water Indicator Tape ermöglichte nicht nur eine Messung des LWC, sondern lieferte auch Informationen über die geometrischen Eigenschaften des Schnees. Diese erweiterten Messmöglichkeiten eröffnen neue Perspektiven für die Analyse und das Verständnis von Schneemetamorphose, LWC und Lawinengefahr. Insgesamt zeigt die Arbeit, dass das entwickelte Messsystem ein Werkzeug zur Bestimmung des LWC darstellt kann und durch weitere Verbesserungen in zukünftigen Forschungsarbeiten optimiert werden kann.

\iffalse

Dazu habe ich unterschiedliche  physikalische Wirkprinzipien getestet. 
Die vielversprechendste Technik basiert auf einem Water Indikator Tape- das ursprünglich für die Elektronik entwickelt wurde und  das visuell ausgewertet wird.

 Für eine definierte Zeitspanne wird das Tape auf die Schneeoberfläche gelegt. Anschliessend wird es fotografiert. Die Auswertung erfolgt durch visuelle und digitale Analyse.
Das Produkt durchlief 5 Iterationen. Die entwickelte Messtechnik zeigt die Fähigkeit, die Interaktion des gefrorenen und des flüssigen Wassers mit dem Tape zu erfasseberfasst werden.

Weitere Optimierung und Testung ist erforderlich, um die Genauigkeit und Präzision des Sensors zu gewährleisten.

Dies ist Voraussetzung dafür, den Sensor in der Zukunft zu einem marktfähigen Produkt weiter entwicklen zu können.

Die Ergebnisse dieser Arbeit können einen Beitrag zur Weiterentwicklung von Messmethoden für den LWC im Schnee leisten. Eine  verbesserte Vorhersage von Gleitschneelawinen ermöglicht adäquate Reaktionen auf dieses Naturereignis. 




in dieser produktentwicklungsaufgabe wurde eine Innovativer sensor entwickelt um den Liquid water Contet von schnee zu messen.

dazu wurden verschiedene physikalische wirkprinzipien getesten. die vielversprechendste technik in der ein Water indikator Tape aus der qualitassicherung im elektronikbrachnche visuell auswertet wird, wurde uber 5 iterationen entwickelnt um die interaktion des schnees und Taps zu verstehen.


Die messmethode zeigt die fahigkeiten den lwc von schnee zu erfassen. bis jetzt ist es aber noch nicht sicher ob die prazision und genauigkeit ausreichend ist um in ein produkt umgesetzt zu werden.


In dieser Arbeit wurde ein innovativer Sensor zur Messung des Flüssigwassergehalts (Liquid Water Content, LWC) in Schnee entwickelt. Verschiedene physikalische Prinzipien wurden getestet, um die beste Methode zur Bestimmung des LWC zu identifizieren. Die vielversprechendste Technik erwies sich als der Einsatz eines Wasserindikatorbands aus der Qualitätssicherung in der Elektronikbranche, welches visuell ausgewertet wird. Über fünf Iterationen hinweg wurde der Sensor weiterentwickelt, um die Interaktion zwischen Schnee und dem Indikatorband besser zu verstehen.

Die Auswertung erfolgt durch visuelle und digitale Analyse des 3M 5559 Water Indikator Tapes. Das Tape, das bei Kontakt mit Wasser rot wird, wird für definierte Zeitspannen auf die Schneeoberfläche gelegt und anschließend fotografiert. Die visuelle Beurteilung erfolgt durch die einfache Betrachtung der roten Verfärbung auf dem Tape. Für eine präzisere Analyse wird die Bildverarbeitung eingesetzt, bei der Software den Anteil der roten zu weißen Fläche berechnet, um den Flüssigwassergehalt zu quantifizieren.

Die entwickelte Messtechnik zeigt die Fähigkeit, den LWC im Schnee zu erfassen, jedoch ist die Präzision und Genauigkeit der Messungen noch nicht ausreichend, um den Sensor als marktfähiges Produkt zu etablieren. Weitere Optimierungen und umfangreiche Tests sind erforderlich, um die Zuverlässigkeit und Praktikabilität des Sensors zu gewährleisten. Die Ergebnisse dieser Arbeit liefern einen wichtigen Beitrag zur Weiterentwicklung von Messmethoden für den LWC im Schnee und könnten zukünftig zur Verbesserung der Vorhersage und Prävention von Gleitschneelawinen beitragen.

\fi
