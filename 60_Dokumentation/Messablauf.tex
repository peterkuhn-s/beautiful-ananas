um an möglichst viel schneetypen anwendbar zu sein, ist das andrücken kraftgesteuert. in der vorstudie \ref{} war die messung weggesteuert.

als einfaches kraft, war die gewichtskraft. um an mehr schnee von puder bis zu hartem zeug. ist 80 \% des maximalen traglast des schnees die anpresskraft. mit ca. 50 g blechen kann das gewicht zusammengesetzt werden. das maximal gewicht ist der maximal gemessene wert in dem feldversuch \ref{}


die messung wird wie folgt durchgefuhrt

\begin{enumerate}
\item Ein schnee der moglichst homogen und von menschen unbeeinflusst ist finden
\item mit einer schaufel oder ahnlichem wird ein kleiner schneegraben geschaufelt.
\item eine schaubere horizontale flache im schnee mit der blechklinge frei legen
\item das stativmaterial wird im schnee aufgebaut
\item mit der federwage oder durch ausprobieren die maximale Traglast des schnees ermitteln
\item die Gewichte der tape holders zu 80 der maximalen tragnlast des schnees zusammen schrauben
\item die tape halter aus der zweifachen geschutzten beuteln entnehmen
\item die tape halter mit den gewichten zusammen steckne
\item mit den kaltespray die taps runter kuhlen
\item mit der warmebildkamera uberprufen, ob die tapes die richtige temperatur haben
\item die tapes auf den schnee vorsichtig aufsetezen
\item mit dem magnetischen halter die gewichte an das stativmaterial befestigen
\item 120 sekunden warten so dass das wasser aus dem schnee auf das Tape ubergehen kann
\item mit der durckluft allfallige schneeflocken vom tape entfernen
\item 300 sekunden warten bis das tape einen stabilen zustand erreicht hat
\item die tape holders in die lichtbox befestigen
\item ein bild der taps machen
\end{enumerate}
