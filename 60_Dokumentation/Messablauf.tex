Um eine Messung mit dem Tape zu machen, muss das Tape mit dem zu messenden Schnee in Kontakt gebracht werden. Der Messablauf, wie der Schnee vorbereitet wird und wie das Tape mit dem Schnee in Kontakt kommt, ist das Hauptergebniss der Bachlorarbeit.

Um bei möglichst vielen Schneetypen anwendbar zu sein, ist das Andrücken des Tapes an den Schnee Kraft gesteuert. In der Vorstudie \ref{sec:5559} war die Messung Weg gesteuert.

Als Kraft für den Anpressdruck wurde die Gewichtskraft gewählt. So kann die Messung von Puderschnee bis hin zu Firn durchgeführt werden.

Die Messung mit dem vierten Funktionsmuster wird wie folgt durchgeführt. Die Funktionsmuster werden in \ref{sec:EigenFunk} beschrieben:

\begin{enumerate}
\item Einen Schnee finden, der möglichst homogen und von Menschen unbeeinflusst ist.
\item Mit einer Schaufel einen kleinen Schneegraben schaufeln.
\item Auf den jeweils zu messenden Höhen je eine saubere horizontale Fläche im Schnee mit einer Blechklinge freilegen.
\item Das Stativmaterial im Schnee aufbauen.
\item Mit der Federwaage die maximale Traglast des Schnees an der freigelegten Stelle ermitteln.
\item Die Gewichte der Tape-Halter mit 80 \% der maximalen Traglast des Schnees zusammenschrauben.
\item Die Tape-Halter aus den vor Feuchtigkeit und anderen Umwelteinflüssen geschützten Beuteln entnehmen.
\item Die Tape-Halter mit den 36 g schweren Gewichten zusammenstecken.
\item Mit dem Kältespray die Tapes runterkühlen auf 0 Grad.
\item Mit der Wärmebildkamera überprüfen, ob die Tapes die richtige Temperatur haben.
\item Die Tapes vorsichtig horizontal auf den Schnee aufsetzen.
\item Mit dem magnetischen Halter die Gewichte an das Stativmaterial befestigen.
\item 2 Minuten warten, sodass das Wasser aus dem Schnee auf das Tape übergehen kann.
\item Mit Druckluft aus einer Dose allfällige Schneeflocken vom Tape entfernen.
\item 5 Minuten warten, bis das Tape einen stabilen Zustand erreicht hat.
\item Die Tape-Halter in der Lichtbox befestigen.
\item Ein Bild der Tapes aufnehmen.
\end{enumerate}
