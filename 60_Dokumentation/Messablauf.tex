

Um an möglichst viele Schneetypen anwendbar zu sein, ist das Andrücken kraftgesteuert. In der Vorstudie \ref{sec:5559} war die Messung weggesteuert.

Als Kraft für den Anpressdruck wurde die Gewichtskraft gewählt. Es wird 80 \% der maximalen Traglast des Schnees für die Anpresskraft genutzt. So kann die Messung in Puderschnee bis hin zu Firn durchgeführt werden. Mit 36 g schweren Blechstücken kann das Gewicht zusammengesetzt werden. Das maximale Gewicht ist der maximal gemessene Wert der Traglast des Schnees im Feldversuch.

Die Messung wird wie folgt durchgeführt:

\begin{enumerate}
\item Einen Schnee finden, der möglichst homogen und von Menschen unbeeinflusst ist.
\item Mit einer Schaufel oder Ähnlichem wird ein kleiner Schneegraben schaufeln.
\item Auf der zu messenden Höhe eine saubere horizontale Fläche im Schnee mit der Blechklinge freilegen.
\item Das Stativmaterial wird im Schnee aufgebaut.
\item Mit der Federwaage oder durch Ausprobieren die maximale Traglast des Schnees ermitteln.
\item Die Gewichte der Tape-Halter zu 80 \% der maximalen Traglast des Schnees zusammenschrauben.
\item Die Tape-Halter aus den zweifach geschützten Beuteln entnehmen.
\item Die Tape-Halter mit den Gewichten zusammenstecken.
\item Mit dem Kältespray die Tapes runterkühlen 0 Grad.
\item Mit der Wärmebildkamera überprüfen, ob die Tapes die richtige Temperatur haben.
\item Die Tapes vorsichtig horinzontal auf den Schnee aufsetzen.
\item Mit dem magnetischen Halter die Gewichte an das Stativmaterial befestigen.
\item 120 Sekunden warten, sodass das Wasser aus dem Schnee auf das Tape übergehen kann.
\item Mit Druckluft allfällige Schneeflocken vom Tape entfernen.
\item 300 Sekunden warten, bis das Tape einen stabilen Zustand erreicht hat.
\item Die Tape-Halter in der Lichtbox befestigen.
\item Ein Bild der Tapes aufnehmen.
\end{enumerate}
