\documentclass[a4paper,12pt]{article}

% Pakete laden
\usepackage[T1]{fontenc}
\usepackage[utf8]{inputenc}
\usepackage[ngerman]{babel}
%\usepackage{lmodern}
\usepackage{geometry}
%\usepackage{setspace}
\usepackage{graphicx}
%\usepackage{amsmath}
%\usepackage{amssymb}
%\usepackage{booktabs}
%\usepackage{natbib}
%\usepackage{tikz}
%\usepackage{caption}
%\usepackage{listings}
%\usepackage{ragged2e}
%\usetikzlibrary{positioning}
%\usetikzlibrary{shapes}
%\usepackage{afterpage}
%\usepackage{float}
%\usepackage{titlesec}
\usepackage{hyperref}%hyperref links

\usepackage{fancyhdr}%footer
\usepackage{graphicx}%images
\usepackage{wrapfig}%images in text
\usepackage{adjustbox}
\usepackage{pdfpages}  % Include this line to use the pdfpages package
\usepackage{longtable}%for table that spread over multiple pages
\usepackage{float}%fix figure position
\usepackage{listings}%code import
\usepackage[style=numeric, backend=biber]{biblatex}%literaturverzeichniss
\usepackage{acronym}%abkurzungen
\usepackage{multicol}%mehrene spalten

\addbibresource{Literaturverzeichniss.bib}%"?
%\usepackage{lastpage}
\usepackage[yyyymmdd]{datetime}%display the date better
\renewcommand{\dateseparator}{-}

%\bibliographystyle{unsrt}





% Kopfzeile
\pagestyle{fancy}
\fancyhf{}
\fancyhead[L]{\includegraphics[width=5 cm]{Bilder/OST_Logo.png}}
\fancyhead[R]{Maschinentechnik | Innovation}
\geometry{headsep=1.6cm}%sonst fängt das logo zu weit unten

% Fusszeile 
\rfoot{Seite \thepage}
%\rfoot{\thepage}
\renewcommand{\footrulewidth}{0pt}
\fancyfoot[L]{Peter Kuhn} % author on the left
\fancyfoot[C]{Bachelorarbeit FS 2024} % title in the c enter




% Titelblatt
\begin{document}


\begin{titlepage}
  \end{titlepage}
% Import the PDF
%toChange
\includepdf{Titelseite.pdf}
%import something from scribus


\pagestyle{empty}
\section*{Abstract}
problem

vorstudien

ergebniss von Funktionsmuster


\newpage
\subsection*{Beschreibung der Abkürzungen}
\begin{acronym}[XXXXX]
    \setlength{\itemsep}{-\parsep}
    \acro{BA}{Bachelorarbeit}
    \acro{LWC}{Liquid Water Content}
    \acro{SLF}{Schweizerisches Institut für Schnee- und Lawinenforschung}
    \acro{IPEK}{Institut für Produktentwicklung}
    \acro{TRL}{Technology Readiness Level}
    \acro{ML}{Maschinelles Lernen}
    \acro{IR}{Infra Rot}
    \acro{FDM}{Fused Deposition Modeling}
    \acro{FS}{Fruhlings Semester}
    \acro{OST}{Ost Schweizer Fachhochschule}
    \acro{MHz}{mega Herz}
    \acro{GPS}{Gobal Positioning System}
    \acro{MRI}{Magnetic Resonance  Imaging}
    \acro{Tape}{Water Indicator Tape 5559 von 3M}
    \acro{CAD}{Computer Aided Design}
    \acro{RGB}{Rot Grün Blau}
    \acro{DB}{Daten Bank}
    \acro{XPS}{extruded polystyrene}
    \acro{s}{Sekunde}
\end{acronym}

\newpage
% Inhaltsverzeichnis
\tableofcontents
%\pagenumbering{Roman}
\newpage
\pagestyle{fancy}

\setcounter{page}{1}
\section{Einleitung}
Gleitschneelawinen gefährden Menschenleben und sind bisher schwer vorhersagbar. Durch die Klimaerwärmung werden sie häufiger auftreten. Ein wichtiger Indikator für die Bildung dieser Lawinen ist der Anteil von flüssigem Wasser im Schnee.

Ziel dieser Arbeit ist, einen Sensor zu entwickeln, der den Liquid Water Content (LWC) von Schnee misst, und damit die Lawinen- Vorhersage verbessert.


Das flüssige Wasser im Schnee ist ein entscheidender Parameter um das Verhalten des Schnees an einem Lawinen gefährdeten Hang  vorherzusagen. 

Die vorhandenen Messgeräte nutzen unterschiedliche Ansätze, haben aber Nachteile zum Beispiel, dass sie das Verhältnis von flüssigem Wasser zum vorhandenen Schnee nicht in einem Arbeitsschritt erfassen.

Ich habe verschiedene theoretische Ansätze der Produktentwicklung im Verlauf der Bachelorarbeit genutzt.

Um den Sensor herzustellen wurde entsprechend der agilen Hardware Entwicklung möglichst rasch Iterationen von Sensoren als Produkt hergestellt, getestet und angepasst.

\iffalse


Ziel dieser Arbeit ist, einen Sensor zu entwickeln, der flüssiges Wasser im Schnee misst.
ziel dieser Arbeit ist die entwicklung eines innovativen sensors um die scheefeuchtigkeit zu messen.

Die schneefeuchtigkeit ist ein entscheidenen Parameter um Gleitschneelawinen abzuschetzten. seit 40 jahren ist wird Thema beforscht. Es gibt verschiedenste Techniken um den Schaum aus Eis, Wasser und Luft zu messen. heutige Produkte konnen den LWC messen, haben aber verschiedene schwerwigeende nachteile.

um dieses Produktentwicklung an zu gehen werden verschiedene techniken der Produktentwicklung eingesetzt. um ein sensor zu erreichen der einsatztfahig ist, wurde nach aglier hardware entwicklung moglichst schnell Itterationen von sensoren entwickelt.

\fi


\subsection{Lawinen in der Schweiz}
Jedes Jahr sterben in der Schweiz ca. 24 Menschen durch Lawinen, wobei ca. 19 der Todesfälle durch Schneebrettlawinen, und durch ca. 5 Gleitschneelawinen verursacht werden. Mit dem Klimawandel ändern sich Häufigkeit und Eigenschaften von Lawinen. Im Gegensatz zur Schneebrettlawine können Gleitschneelawinen bisher kaum frühzeitig erkannt und auch nicht präventiv durch Detonation ausgelöst werden. \cite{WSLSLF.2024}


\iffalse
Jedes Jahr sterben in der Schweiz ca. 10 Menschen durch Lawinen, wobei 8 durch Schneebrettlawinen und 2 durch Gleitlawinen verursacht werden. Mit dem Klimawandel ändern sich die Eigenschaften von Gleitlawinen. Diese können nicht präventiv durch Detonation ausgelöst werden und sind zeitlich schwer vorherzusagen.


jedes jahr 10 Tote. 8 schneebrettlawine. 2 Gleitlawinen.

mit Klimawandel änders sich Gleitlawinien. nicht preventiv mit einer Detonation auslösbar. nicht zeitlich vorhersagbar.


\section{Entstehung der Gleitlawine}



\section{Endziel der Arbeit}

Das Ziel dieser Arbeit ist es, den Schaden durch Gleitlawinen zu verringern. Dies soll durch die Entwicklung eines Sensors zur Messung der Schneefeuchtigkeit erreicht werden, da diese ein entscheidender Parameter für die Abschätzung der Lawinengefahr ist.

\section{User Story}

Um die Aufgabe der Produktentwicklung besser zu verstehen, wurden verschiedene User Stories erstellt. Eine dieser Stories beschreibt die Anwendung des entwickelten Sensors:

Alice führt an einem Hang eine Schneedeckenanalyse durch. Neben ihrer subjektiven Beurteilung trägt sie die Messwerte der Schneefeuchtigkeit ein, die sie mit dem neuen Sensor ermittelt hat.

\section{Anforderungen}

Es wurde eine Liste von Anforderungen erstellt, die das fertige Produkt erfüllen soll:
\begin{itemize}
    \item Die Methode soll eine Anzeige haben, die feststellt, wann eine Gleitlawine bevorsteht.
    \item Die Methode soll unabhängig von der Dichte des Schnees funktionieren.
    \item Der Messbereich des LWC (Liquid Water Content) soll von 1 \% bis 7 \% abgedeckt werden.
    \item Die Methode soll für einen Hang in der Schweiz einsetzbar sein.
\end{itemize}

Bis zum Abschluss der Arbeit konnten diese Anforderungen nur eingeschränkt bestätigt werden. Für eine aussagekräftige Statistik sind noch mindestens 1000 Messungen erforderlich.

\section{Planung der Arbeit}

Die Arbeit gliedert sich in drei Teile:
\begin{enumerate}
    \item In einer Vorstudie werden verschiedene physikalische Prinzipien zur Messung des LWC theoretisch und praktisch verglichen.
    \item Es wird ein Funktionsmuster gebaut, basierend auf dem vielversprechendsten physikalischen Prinzip. Dieser Teil wird nach agiler Hardware-Entwicklung mit einem Kanbanboard geplant.
    \item Der dritte Teil beschreibt die Dokumentation der Produktentwicklung und die entwickelte Messtechnik.
\end{enumerate}
\fi


\subsection{Entstehung der Gleitschneelawine}



Gleitschneelawinen entstehen, wenn die gesamte Schneedecke auf einem glattem Untergrund wie einem Grashang oder glatten Felsen abrutscht. Dies passiert sowohl bei trockener, kalter als auch bei nasser, isothermer Schneedecke. Typisch für Gleitschneelawinen ist eine dicke Schneedecke ohne oder mit nur wenigen Schwachschichten. Diese Lawinenart wird fast ausschließlich natürlich ausgelöst und kündigt sich oft nur durch sogenannte Gleitschneerisse, sogenannte “Fischmäuler” an. Der Auslösemechanismus beruht auf dem Verlust der Reibung zwischen Schnee und Boden aufgrund von flüssigem Wasser. Man sollte sich nicht in der Nähe von Gleitschneerissen aufhalten, da diese die Lawinengefahr anzeigen, auch wenn sie  keine unmittelbar bevorstehende Auslösung vorhersagen können.

Die Entstehung von Gleitlawinen ist stark von der Feuchtigkeit im Schnee abhängig. Diese Feuchtigkeit sammelt sich zwischen den Eiskristallen und stammt aus verschiedenen Quellen:

\begin{itemize}
    \item Schmelzender Schnee, hauptsächlich durch primare und sekundäre Strahlung.
    \item Regen, der auf die Schneedecke fällt.
    \item Feuchtigkeit aus dem Boden, insbesondere aus wasserführenden Schichten.
\end{itemize}



\subsection{User Story}
Um die Aufgabe des Produkts besser zu verstehen, wurden User Stories geschrieben. In Kapitel \ref{userstoryvoll} sind alle 6 User Storys zu lesen.

User Stories werden genutzte, um sich früh Gedanken über den Einsatz des fertige Produkt zu machen. Hier ist die User story, die das schlussendlich entwickelte Funktionsmuster beschreibt:

Alice macht in einem Hang einen Schedeckenanalyse. Sie hat alles Material und Messgeräte in ihrem Rucksack mitgebracht. Mit einer Schaufeln gräbt sie einen Schneegraben. Neben ihrer üblichen Messungen und ihrer subjektiven Beurteilung trägt sie noch die Messwerte der Schnee Feuchtigkeit in das Protokoll ein.

Die unterschiedlichen User Stories beschreiben komplett unterschiedliche Produkte. Da noch nicht entscheiden werden kann, welche die korrekte Anwendung ist, werden zuerst die Technologie erkundet. Sobald eine Technologie gefunden ist wird diese zu einer kornkreten Anwendung ausgearbeitet.

Jede weitere abstrakte Planung, wie zum Beispiel Black Box und Musskriterien machen somit noch keinen Sinn, jetzt schon definiert zu werden, da innovative Produkte während der Arbeit damit eingeschränkt würde.



\iffalse

Um die Aufgabe der produktentwicklung besser zu verstehen, wurden User storys geschrieben. In \ref{userstoryvoll} sind alle 6 User Storys.

das ziel der Userstorys ist es fruh sich gedanken uber das fertige produkt zu machen. Hier ist die User story die das schlussendlichten entwickelten sensor beschreibt.

Alice macht an einem hang einen schedeckenanalyse, mit der schaufeln. neben ihrer subkektiven beurteilung tragt sie noch die messwerte der schneefeuchtikeit ein.


Die Userstorys beschreiben komplett unterschiedliche Produkte. da ich nicht entscheiden kann oder will, welches die korrekte Anwendung ist, werde ich zu erst die Technologie erkunden und dann die anwendung in ein Produkt finden.

jede weiteren pflichenheft aktivitaten, wie zum beispiel black box, Musskriteren, usw. machen keinen sinn jetzt schon definiert zu werden. da spanneden Entdecknugen währen der arbeit damit eingeschränkt werden.

\fi


\subsection{Planung der Arbeit}
Die Arbeit wird in drei Teile aufgegliedert.

in einer Vorstudie werden unterschidilche physikalische Prinzipien zur messung des LWC theoritisch und praktisch mit eineander verglichen.

bau den Funktionsmusters. hier wird ein vielversprechendes physikalisches prinzip ausgewählt und ein Funktionsmuster gebaut. dieser teil wird nach agiler hardware Entwicklung mit einem Kanbanboard geplant.

Der dritte Teil wird genutzt um die Dokumentation der Produktentwicklung und Entwicklete Messtechnik beschrieben.


\section{Liquid Water Content}
Der Liquid Water Content (LWC) ist ein entscheidender Parameter in der Meteorologie und Glaziologie, der den Gehalt an flüssigem Wasser in Schnee beschreibt. Der LWC beeinflusst die physikalischen Eigenschaften des Schnees, wie seine Dichte, Wärmeleitfähigkeit und mechanische Stabilität. Der LWC wird als Verhältnis des Volumens oder Gewichts von flüssigem Wasser zum Gesamtvolumen oder Gewicht der Eiskristalle ausgedrückt.

Schnee ist ein heterogenes Gemisch aus festen, flüssigen und gasförmigen Stoffen, eine Art Schaum aus Eis, Wasser und Luft. Seit 40 Jahren werden Techniken erforscht und Methoden entwickelt, um den LWC zu messen. 

Seit 40 Jahren werden Techniken erforscht um den LWC zu messen. Es gibt unterschiedliche Methoden das heterogene Gemisch aus festen (Eis), flüssigen (Wasser) und gasförmigen (Luft) Stoffen zu messen.


\subsection{Physicalische Prinzipien}
Bei der Recherche nach physikalischen Prinzipien zur Messung des LWC wurden unterschiedliche, aktuelle Techniken gefunden. Hier ist eine Liste:

%\begin{multicols}{2}
\textbf{Absorption elektromagnetischer Wellen}
\begin{itemize}
\item GPS \cite{Koch.2019}, \cite{Koch.2014}
\item Radar \cite{Bonnell.2021}, \cite{ilmsens-short-range-radar}
\item IR \cite{Donahue.2022}
\item Satellitenaufnahmen \cite{Tsang.2022}

\end{itemize}

\textbf{Dielektrische Konstante}
\begin{itemize}
\item bei 20 MHz über eine Platte \cite{slf}
\item zwischen einer Gabel \cite{fork}
\item Resonanz in einem Zylinder \cite{a2photonicsensors}, \cite{nasa-snowex-2020}
\item Sonstige Anordnungen aus der Agrikultur \cite{Mavrovic.2020}, \cite{PerezDiaz.2017}
\end{itemize}

\textbf{Akustische Wellen}
\begin{itemize}
\item Absorption normaler Schall \cite{Kinar.2007}
\end{itemize}


\textbf{Elektrische Eigenschaften}
\begin{itemize}
\item ohmscher Widerstand \cite{Abdelaal.2022}
\end{itemize}

\textbf{Mechanische Eigenschaften}
\begin{itemize}
\item Scherkräfte \cite{Hao.2021}, \cite{jstage-snow-density}
\item Dichte \cite{nasa-snow-density}

\end{itemize}

%\columnbreak
\textbf{Thermische Eigenschaften}
\begin{itemize}
\item Schmelzenergie mit DSC \cite{mt-density-meter}
\item mit heissem Wasser \cite{Fasani.2023}
\item mit kalter Flüssigkeit

\end{itemize}

\textbf{Kapillarkräfte}
\begin{itemize}

\item Oberflächenspannung \cite{AlamShibly.2017}
\end{itemize}

\textbf{Sonstiges}
\begin{itemize}
\item optische Beurteilung \cite{miro}
\item Luftfeuchtigkeit verändern \cite{joule-thomson-wiki}, \cite{sensirion-sht4xa-sensors}
\item Luftwiderstand
\item MRI \cite{Adachi.2020}, \cite{Nowogrodzki.2018}, \cite{Yamaguchi.2023}
\item Raman-Spektroskopie \cite{Reichardt.2022}
\item Neutronen-Scattering \cite{Lombardo.2023}
\end{itemize}
%\end{multicols}


Im Folgenden sind potenzielle Techniken aufgeführt, die zur Messung des LWC im Schnee verwendet werden könnten, aber zu denen keine spezifischen Veröffentlichungen gefunden wurden.

\textbf{Absorption elektromagnetischer Wellen}
\begin{itemize}
\item Mikrowellen
\end{itemize}



\textbf{Akustische Wellen}
\begin{itemize}
\item Absorption Ultraschall
\item Lamb-Welle \cite{lamb}
\item Emission akustischer Wellen
\end{itemize}


\textbf{Mechanische Eigenschaften}
\begin{itemize}
\item Eigenschwingungen
\item Vibrationsübertragung
\item Eindrückwiderstand mit Vibration
\item Viskosität
\item Vibrationsbohrer
\end{itemize}


\textbf{Optische Eigenschaften}
\begin{itemize}
\item Reflexion
\item Refraktion
\item Polarimetrie
\end{itemize}

%\columnbreak
\textbf{Thermische Eigenschaften}
\begin{itemize}
\item mit kalter Flüssigkeit
\item Heizung (elektrisch, Mikrowelle)
\item Taupunktspiegel
\item Leitfähigkeit von Wärme
\end{itemize}

\textbf{Separation}
\begin{itemize}
\item Zentrifuge
\item Quetschen
\item Absaugen
\end{itemize}

\textbf{Kapillarkräfte}
\begin{itemize}
\item Water Indicator Tape
\item Staub/Flüssigkeit beim Ausbreiten im Schnee beobachten (optisch, fluoreszent, elektrisch, Event-Kamera) \cite{kennedylabs-download}
\item Oberflächenspannung \cite{AlamShibly.2017}
\end{itemize}

\textbf{Sonstiges}
\begin{itemize}
\item optische Beurteilung \cite{miro}
\item Luftfeuchtigkeit verändern \cite{joule-thomson-wiki}, \cite{sensirion-sht4xa-sensors}
\item Luftwiderstand
\item Infrarot-Spektroskopie
\end{itemize}
%\end{multicols}


\subsection{Kommerzielle Produkte}
heute gibt es kommerziell erhaltliche produkte die den LWC von schnee messen. Die Proukte nutztne die unterschiedliche dielektischen Konstante von Wasser. hier zu erwahnen sind der SLF Snow Sensor auch Denothmeter genannt und die Finnish snow fork. sensoren aus Agrikulturbereicht die die bodenfeuchtigkeit messen, sind auch im SChnee einsetzbar.

Ein nachteil der produkte ist, dass um auf einen prozentualen LWC zu kommen, die dichte des schnees seperat gemessen werden muss. die raumliche auflosung der produkte ist im Bereich von centimetern.


\subsection{Publizierte Methoden}
\label{sec:PubMeth}

%\subsection{Methoden in der Vorstudie}

\section{Vorstudie}

In der Vorstudie werden verschiedene physikalische Methoden zur Bestimmung des LWC im Schnee getestet.

Es wurden sechs Methoden gewählt, die eine hohe Erfolgswahrscheinlichkeit und einen abschätzbaren Aufwand haben, und nach meiner Recherche, nicht mit anderen Produkten konkurrenzieren.

Weitere Auswahlkriterien für die Methoden sind die Innovativität, die Eleganz des Prinzips und die Umsetzbarkeit im Rahmen dieser Bachelorarbeit.


\subsection{Water Indikator Tape}
Das Water Indikator Tape stammt ursprünglich aus der Qualitätssicherung im dem Elektronikbereich und wird beispielsweise in Handys verwendet, um das Eindringen von Wasser nach zu weisen. Wenn das Tape rot wird, ist Wasser eingedrungen und der Hersteller kann eine Garantieleistung ablehnen. Wenn das papierbasierte Klebeband nass wird, blutet die rote Farbe auf der Unterseite des Klebebands durch das weisse obere Papier hindurch. Das dann rote Tape zeigt dauerhaft das Vorhandensein von Wasser an.

Für diese Arbeit wurde das Produkt 5559 des Herstellers 3M ausgewählt, da es sich durch seine dünnere Dicke und damit schnellere Anzeigegeschwindigkeit auszeichnet. Andere Varianten von Water Indikator Tapes wurden ebenfalls getestet. Es ist teilweise in Europa erhältlich, jedoch verkauft der Produzent es nur in Rollen von 160 m Länge. Eine kleine Rolle wurde von einem Elektronikkomponenten-Vertrieb für die Tests bezogen. Bei der Patent-Recherche zur Messung des LWC im Schnee wurde keine Verwendung von Water Indikator Tapes festgestellt, was auf die Neuartigkeit der Methode hindeutet.

Um die Hydrophilie des Tapes zu beurteilen, wurde ein Wassertropfen auf das Tape gesetzt und der Kontaktwinkel gemessen. Im Bild \ref{fig:winkTropf} ist zu sehen, dass der Winkel zwischen dem Wasser und dem Tape etwa 90 Grad beträgt, was bedeutet, dass das Tape sich an der Grenze zwischen hydrophob und hydrophil befindet.

\begin{figure}
    \centering
    \includegraphics[width=0.8\textwidth]{Bilder/IMG_6683.JPG}
    \caption{Messung des Kontaktwinkels. Links ist ein Tropfen zu sehen, der mehrere Minuten auf dem Tape verweilt. Rechts ist ein neuer Wassertropfen, daneben noch der gefittete Kreis.} 
    \label{fig:winkTropf}
\end{figure}

\iffalse

Das 5559 Tape ist kostengünstig und zeigt innerhalb von weniger als 60 Sekunden das flüssige Wasser in einer Schicht an. Allerdings muss die Dichte des Schnees separat gemessen werden, da das Tape nur das flüssige Wasser anzeigt. Der Testaufbau beinhaltete das Kleben des 5559 Tapes auf ein rund 200 g schweres Objekt, das Freilegen einer neuen Schneeoberfläche mit einem Messer, das Auflegen des Tapes auf den Schnee und das Warten für 10, 30, 60 und 120 Sekunden. Anschließend wurde das Klebeband fotografiert und die rote gegen die weiße Fläche entweder optisch oder mithilfe von Python berechnet.

herkunft: Aus dem Elektronik bereich. zum beispiel in handys. wenn das tape rot geworden ist, ist wasser eingedrungen und der Hersteller kann eine garatieleistung ablehnen.

Funktionsweise: das papier basierte klebeband wird nass. die rote Farbe auf der Unterseite des Klebebands blutet durch das weisse obere Papier. die Roten Teile zeiget dann permanet wasser an.

Auswahl von 5559: der Hersteller 3M hat mehrere Produkte zu Water Indikator. 5559 zeichnet sich durch die dünnere Dicke und somit durch die schneller Anzeigegeschwindigkeit aus.

5559i ist auf einem transparenten substrat, was fraktisch für die optische auswertung wäre. Die Produkte sind in europa nur teilweise erhältlich. 3M verkauft nur Rollen mit 160 m. Zum testen wurde eine kleine rolle von einem Elektronikkomponenten Vertreiber gekauft.

Bei der Recherche zu LWC wurde keine verwendung von Water indicator tapes bemerkt. somit neuartig.

kostengünstig

zeitspanne pro messung weniger als 60 sek.

Dichte des Schnees muss seperat gemessen werden. 5559 zeigt nur das flüssige wasser in einer schicht an.

Testaufbau: 5559 auf etwas rund 200 g schweres kleben. neue Oberfläche von schnee mit Messer abschneiden/freilegen. 5559 auf schnee legen und 10, 30 60, 120 sek warten. foto von klebeband machen. mit python rote vs. weise fläche berechnen. oder nur optisch beurteilen.
\fi


\subsection{Elektrischer wiederstand}
die gaphit sonden, zwischen denen die spanung aufgebaut und der wiederstand gemessen wird sind im messkopf zu gut geschützt. daher kann keine Messung gemacht werden wenn die Probe in schnee gedrückt wird.

Mögliche lösung: Verlängerung der Graphit proben

mit stahlplatten

Verbindung des Graphits mit der Platte: kleben oder konstant drückne oder verschrauben.

in gaphit spahnend zu arbeiten ist anspruchsvoll und dreckig.
konstant drücknen ist fehleranfällig
Kleben: herstellen von leitfähigem Klebstoff:

test graphitpluver: 66 \% gewichtsprozent Graphitpulver, 33 \% Ergo 7410 Epoxy Klebstoff

test Aluminiumpulver: 66 \% gewichtsprozent Aluminiumpulver, 33 \% Ergo 7410 Epoxy Klebstoff


Ergebniss: nach 24 h, sodass der ergo 7410 aushärten konnte.
Alle Klebestellen sind angeschliffen worden als oberflächenvorbereitung

Wiederstand zwischen Punkt A B 2.6 \ohm

wiederstand ziwschen Punkt A C 0.2 \ohm

Wiederstand zwischen Punkt A D keine verbindung

Mechanische stabilität von Test Aluminiumpluver nicht so gut

Ist es möglich auf die stahlplatte zu verzichten und die Verlängerung mit der Graphit Epoxy mischung zu machen?

zwisched die beiden grafit stäbe ist eine PAAM Platte geklebt. alle offenen stellen des Epoxy/grafits ist mit reinem epoxy überzogen um kriechspannungen durch wasser zu verhindern.

Arbeitsschutzt, erklären

Schnee ist wasser das vom Boden verdampft, sich dann in der Atmosspäre an einem Nukleus kondesiert oder resubliemiert und dann auf den Boden zurück fällt.

Im Alltag weiss man, dass man mit den Harrfön nicht in die Dusche gehen darf, da Wasser elektrisch leitend ist. Diese Schlussfolgerung ist nicht sehr prezise. denn reines Wasser ist nicht leitend, sonder die  Ione (Salze) die im 'normalen' Wasser gelöst sind. Auf sehr geringem Niveu ist auch reines Wasser leitend, da sich spontan  1*10 ^ 7 M  Hydroniumionen (H_3 O^+) bilden und den pH Wert 7 bilden.

Die Hypotese ist, dass sowohl die Verunreinigungen durch die Nuklei und die Hydroniuminonen genügend leitfägkeit bilden um einen Messwert im \mu S (Siemens = 1/\Ohm) Bereich zu messen.

Im Feldversuch konntekeine Leitfähigkeit gemessen werden.

EIne erweiterung dieser Messung ist, einen stoff zum schnee dazu zu geben, der gut leitfähig ist. dann wird der Versuchsaufbaumehr in die richtung \ref{} wo die Ausbreitung eines Stoffes im Schnee beobachtet wird. hier wäre diese beobachtung dann über die Leitfähigkeit und nicht wie in \ref{} optisch.


\subsection{Laser Refraktion und Reflezion}



\textbf{Titel}: Laser Refraktion und Reflezion

\textbf{Fuktionsweise} Mit einem Laser wird der Schnee sowohl durchleuchtet für die Refraxion als auch angeleuchtet für die Reflexion. Flüssiges Wasser bildet wegen seiner Oberflächenspannung konkave linsen auf den Prismen des Eis kristalle. Die grösse und damit die Brennweite ändert sich, je nach dem wieviel Volumen Wasser auf den Eiskristallen ist. Die Effekte der Linsen sollten in der Refraxion sichtbar werden. i

In der Reflexion andert sich mit änderndem LWC auch die Oberfläche an der das licht gespiegelt wird.

Der TRL für Refraxion und Reflexion ist bei 2.

\textbf{Beispiele in anderen Sektoren}
Refraxion wird in der Kristalografie angewant.

und Reflexion wird bei einem auflicht mirkoskop immer angeendet.

in beiden Flällen ist hier das TRL 9.

\textbf{Literatur zu Reflexion}
Reflexion von IR
In 20XX hat Herr XX die Reflexion von IR genutzt um den LWC von Schnee zu bestimmen. Die Ergebnisse wurden im Journal XX veröffentilcht.

\textbf{benutzte Mittel für den Versuchsaufbau}
Als Laserquelle wurde ein grüner Bosch Quingo Kreuzlaser genutzt.

Um sowohl die Reflexion als auch die Refraxion gleichzeitig zu sehen, wurde die Schneeprobe auf einen Mikroskopier Objekt träger plaziert.

die Ergebnisse des Lasers wurden jeweils auf weissem Druckpapier dargestellt. Die Refraxion wird auf dem Papier an der Unterseite der Holzplatte dargestellt.

mit dem Fairphone 3 wurde eine Video aufnahme gemacht, wie sich die Ergebinnse des Lasers verändern.

mit einem Kosmetik Spiegel wurde sowohl die reflexion unten als auch die Refraxion oben gleichzeitig in einem Bild dargestellt.

Um alle Teile in fixen relationen zu halten wurde Stativmaterial genutzt.

In Bild \ref{LaserAufbau} ist die Anordnung der Verschieden Teile auf den Stativmaterial zu sehen.

\textbf{Funktionsweise des Versuchsaufbaus}
Der schnee wird in trockenem Zustand bei -10 gard aus dem Gefrieschrank auf den gekühlten Objektträger gelegt. dann wird beobachet wie sich die Ergebnisse ändern wenn der schnee an der Raumtemperatur Luft schmiltzt. Dieser Schmelzvorgang hat rund 5 min gedauert.

der laser scheint durch den Objektievtrager und den schnee durch. dann wird das Lich auf den Papier erneut in die Kamera reflektiert.

die reflexion geschiet zum einen direkt am objektträger, als auch danach im schnee. dieser Aufbau ist suboptimal, denn die konstante reflexion des Objetträgers muss aus dem Laser Ergebniss rausgerechnet werden.

Um Störlicht zu minimieren wurde zuerst eine Einhausung geplant. der durchgeführte verusch hat dann aber einfach in einenm abgedunkelten Raum statt gefunden.

\textbf{Messgrössen}
die Anhäufungen von Licht, und die Intesität wird begutachtet.

\textbf{Versuchsergebnisse}

Im bild \ref{LaserBase} ist die Reflexion und refraxion des Objektträges sichtbar. diese konstanten werte müssen von allen Ergebissen subrtahiert werden

\textbf{Aussagekraft der Ergebnisse über den LWC} Die Ergebnisse werden direkt von Wasser beeinflusst. Um den Gewichts LWC zu erhalten, ist aber die geometier der Eiskristalle von extermer Entscheidung. daher ist das Ergebniss nicht direkt mit den LWC überzuführen. Mit der 3D geometire der kristalle wäre ist die Aussagekraft gut.

\textbf{Reflexion zum Versuchsaufbau}
da zwei techniken gleichzeitig gemessen wurden, war der Versuchsaufbau nicht optimal für beide Messgrössen.

Mit den ergebnissen der refraxion bin ich sehr zu frienedn.

\textbf{Verbesserungen des Versuchsaufbaus}
um besser Reflexionsergebnisse zu bekommen keinen Objektrager nutzte, sondern direkt auf schnee.

Für eine statische Messung einer schneeprobe muss die Luft um den schnee herum gekühlt sein. ein Ansatz dafür wird im Vorversuch \ref{TinteVersuchsaufbau} umgestetzt.

Mit dem Laser wird Energie in den Schnee eingebracht. um das schmelzen und damit verfälschen des LWC zu minimiren sollte ein möglichst schwacher Laser eingesetzt werden.

\textbf{weiterverfolgung der physikarischen methoden}
Das Ergebniss der Refraxion zeigt, dass diese Moethode umgesetzt werden könnte. Um vergleichbare werte zu bekommen ist die kristallgeometrie abre von bedeutung. die messung der geometier übersteigt das ausmass der BA. Um eine Messung durchzuführen muss eine sChneeprobe durchleuchtet werden. um das zu erreichen muss der schnee physikalisch aus der schneedecke extrahiert werden. das ist aufwendig. daher wird die Refraxion nicht weiter verfolgt.

das Ergebniss der reflexion ist schwer zu beurteilen. in \ref{} ist die reflexion von EM Wellen bereits untersucht worden. daher wird die Reflexion nicht weiter untersucht.




\subsection{Vibration}

Diese Hypothese geht davon aus, dass sich der Schnee bei mechanische Anregung von einem festen in den flüssigen Zustand übergeht. Ob der Übergang stattfindet oder nicht, könnte vom LWC Wert abhängen.

Um die Idee zu testen, wird ein vibrierendes Objekt mit hoher Dichte auf den Schnee gelegt, und es wird beobachtet, wie sich das Objekt durch den Schnee bewegt.

Die Form und der Name des Objekts wurde vom AvaNode inspiriert. Der AvaNode ist eine laufende Produktentwicklung des Instituts IPEK der OST, mit dem Lawinenhänge überwacht werden sollen. Der für das Objekt und Verfahren gewählte Name ist VibraNode. Für die Umsetzung wurde ein morphologischer Kasten mit drei Varinanten erstellt.


\begin{figure}[H]
    \centering
    \includegraphics[width=0.9\textwidth]{Bilder/Unbenann2t.PNG}
    \caption{Morphologischer Kasten für VibraNode}
    \label{fig:MorphKasten}
\end{figure}




Die in Abbildung \ref{fig:MorphKasten} dargestellte Variante 1 wurde gewählt, da die Umsetzung und somit das Testen einfach ist. Die Schwächen, wie die fehlende Wiederverwendbarkeit, sind hier in der Vorstudie noch nicht gravierend. Die Schwäche der Wasserbeständigkeit wurde mit einer passenden Beschichtung gelöst.

Für den Test mit dem VibraNode wird der VibraNode auf die Schneedecke gelegt. Das Messergebniss ist, ob sich der VibraNode im Schnee versinckt.

Die Testergebnisse fielen negativ aus. Der VibraNode konnte trotz eines Drucks von 16'000 Pa, respektive seiner Dichte von 1600 kg/m3 nicht in den Schnee eindringen. Auch wenn der Schnee wurde mit flüssigem Wasser gesättigt war. Damit  stellt sich die neue Frage, ob der LWC einen kausalen oder nur einen korrelativen Zusammenhang mit Gleitschneelawinen hat, und wie weit die Vorgeschichte und andere Faktoren des Schnees mitbetrachtet werden muss. \cite{Altman.2015}


\subsection{Diffusion von Flüssigkeit}
\label{sec:TinteVersuchsaufbau}
mit handy und stereoskop aufbau.

schnee gekühlt, durch Eisring und eisunterlage.

gekühlt ist fast noch besser als perfekt isoliert.



\begin{figure}
    \centering
    \includegraphics[width=0.8\textwidth]{Bilder/freistellen.jpeg}
    \caption{Ablauf einer automatischen Messung}
    \label{fig:AutMess}
\end{figure}


\section{Funktionsmuster}

In diesem Kapitel wird das agile Hardware Developement und die verschiedenen Funktionsmuster des Sensors vorgestellt, die im Verlauf der Entwicklung entstanden sind. Ziel der Funktionsmuster war es, verschiedene Ansätze zur Messung und Auswertung des LWC im Schnee zu erproben und zu optimieren. Jedes Funktionsmuster adressiert spezifische Herausforderungen und bringt neue Ideen ein, um die Messgenauigkeit und Benutzerfreundlichkeit zu verbessern.

Die Darstellung der iterativen Entwicklung der Funktionsmuster in dieser linearen Dokumentation ist herausfordernd.


\subsection{Agiles Hardware Devolopment}


Die Planung der Arbeit wird mit einem agilen Kanban-Board durchgeführt. Ein Ziel der agilen Hardware Entwicklung ist möglichst schnell zu einem Funktionsmuster zu kommen und daraus für die nächste Iteration zu lernen.

Um diese schnelle Arbeitsweise zu ermöglichen, wurde folgende Priorisierung der Fertigungstechniken erstellt:

\begin{enumerate}
\item Bestehendes Objekt benutzen und modifizieren
\item Von Hand ausschneiden - Verwendung von Karton
\item IR-Lasercutter - Verwendung von Sperrholz
\item 3D-Druck in FDM
\item Einkaufsteile kaufen
\item Weitere eigene Fertigungsweisen (manuell drehen, fräsen, töpfern usw.)
\item Extern fertigen lassen
\end{enumerate}

Ein Endprodukt wird extern gefertigt werden müssen, um die Wertigkeit des Produkts an den Benutzer zu vermitteln. Die Seriengrösse ist je nach Endprodukt dazugehöriger User Story sehr unterschiedlich.


\subsection{Eigenschaften der Varianten}
\input{EigenFunktVar.tex}

\subsection{Messablauf}


Um an möglichst viele Schneetypen anwendbar zu sein, ist das Andrücken kraftgesteuert. In der Vorstudie \ref{} war die Messung weggesteuert.

Als einfache Kraft war die Gewichtskraft. Um an mehr Schnee von Puder bis zu hartem Zeug anzuwenden, sind 80 \% der maximalen Traglast des Schnees die Anpresskraft. Mit 36 g Blechstücken kann das Gewicht zusammengesetzt werden. Das maximale Gewicht ist der maximal gemessene Wert der Traglast des Schnees im Feldversuch \ref{}.

Die Messung wird wie folgt durchgeführt:

\begin{enumerate}
\item Einen Schnee finden, der möglichst homogen und von Menschen unbeeinflusst ist.
\item Mit einer Schaufel oder Ähnlichem wird ein kleiner Schneegraben geschaufelt.
\item Eine saubere horizontale Fläche im Schnee mit der Blechklinge freilegen.
\item Das Stativmaterial wird im Schnee aufgebaut.
\item Mit der Federwaage oder durch Ausprobieren die maximale Traglast des Schnees ermitteln.
\item Die Gewichte der Tape-Halter zu 80 % der maximalen Traglast des Schnees zusammenschrauben.
\item Die Tape-Halter aus den zweifach geschützten Beuteln entnehmen.
\item Die Tape-Halter mit den Gewichten zusammenstecken.
\item Mit dem Kältespray die Tapes runterkühlen.
\item Mit der Wärmebildkamera überprüfen, ob die Tapes die richtige Temperatur haben.
\item Die Tapes vorsichtig auf den Schnee aufsetzen.
\item Mit dem magnetischen Halter die Gewichte an das Stativmaterial befestigen.
\item 120 Sekunden warten, sodass das Wasser aus dem Schnee auf das Tape übergehen kann.
\item Mit Druckluft allfällige Schneeflocken vom Tape entfernen.
\item 300 Sekunden warten, bis das Tape einen stabilen Zustand erreicht hat.
\item Die Tape-Halter in der Lichtbox befestigen.
\item Ein Bild der Tapes machen.
\end{enumerate}


\subsection{Bildverarbeitung}
Die Auswertung des Tapes kann auf zwei Arten erfolgen. Eine einfache Variante, die bei den Vorversuchen genutzt wird, besteht darin mit den eigenen Augen die Grösse und Verteilung des Rots auf dem Tape abzuschätzen.



Um dieses subjektive Schätzung durch eine objektive Quantifizierung zu ersetzen, wurde eine Pipeline zur digitalen Auswertung entwickelt. In einem ersten Schritt, illustriert in Abbildung \ref{fig:Bildverarbeitnugskonzpet}, wird die Fotografie eines Tape ausgewählt und in ein schwarz-weisses Bild übersetzt, so dass die vorher roten Farbflecken nun deutlicher erkannt werden können.

Um aus dem Bild quantitative Zahlen zu bekommen, werden die Flecken nun einzeln erfasst und in einer Datenback gespeichert. In der Datenbank können dann statistische Aussagen über die Verteilung, Grösse der Flächen, die Struktur und die prozentualen Flächenanteil gemacht werden.



\begin{figure}
    \centering
    \includegraphics[width=0.9\textwidth]{Bilder/Screenshotfrom2024-04-0112-59-42.png}
    \caption{Bildverarbeitung Konzept}
    \label{fig:Bildverarbeitnugskonzpet}
\end{figure}

\newpage


\subsection{Extrahieren von Informationen aus Bilddaten}
Um aus den Bilddaten die in den Feldversuchen gemacht werden Information zu gewinnen, müssen die Daten stukturiert werden. Um das effizient zu speichenrn und machtige datenabfrage machen zu könne (zb pattern matching) wird ein datenbanksystem gebraucht.

Die Daten werden im Feld in die Datenbank geschrieben, und dann zu einen späteren zeitpunkt analysiert.

Im folgenden sind die schritte zur Datenbankauslegung dargestellt.

\subsubsection{Anforderungs Analyse}
Die Anforderungen leiten sich aus der Funktionsweise  des Messaufbaus ab.

Die Datenbank in dieser Bacherlorarbeit wird klein sein, da die Feldversuche zeitintensiev sind. Die vermutung ist, dass maximal 1000 Messungen mit je 3 Taps und je 100 Kreisen.

Mit der Datenbank haben vier Benutzter zu tun.
Die zwei angenelntern Endbenutztre
Die Kamera, die die Bilder der Taps macht und auswerten, muss die Auswertungen in die DB schreiben.

der Versuchsdurchführen gibt zusätzliche Informationen über den Versuch an, die muss er in die DB schreiben.

der Experte Endbenutzter.
Der Analyst wird dann die Daten abfragen, und hoffentlich Information daraus gewinnen.

der DB Administrator wird im Normalbetrieb nicht benötigt, aber soll auch bedacht werden.

\subsubsection{Konzeptueller DB Entwurf}
Mit dem UML (united Modeling lingueage) wird die Struktur der DB dargestellt. Diese Darstellung ist noch lösungsunabhänig.


\subsubsection{Logischer DB Entwurf}
Um die DB zu implemetieren wurde Postgres gewählt. Es ist eine open source system, dass neue features wie zum beispiel json date typen beinhaltet.


\subsubsection{Physischer Entwurf}
Die Datenbank wird mit dem Code \ref{} inizialisiert.

die verschiedenen benutzter, mit den unterschiedlichen Funktionen werden so gemacht.

Die tabels werden so umgesetzt.

Für die Beispieldaten wurden daten aus der Vorstudie \ref{} für eine Messung benutzt.

\subsubsection{Python abfragen mit der DB}
raspberry der schreibt

versuchsleiter der interaktiv schreibt

analyst der auf desktop geschrieben bekommt.

\subsubsection{Views für den Analysten}


\subsection{Wiederstand gegen Umwelteinflüsse}


Bei einer Vorbehandlung mit einem Lösungsmittel, getestet wurden Isopropanol, Nitroverdünner und Aceton, verfärbt sich das Tape temporär. Nachdem das Lösungsmittel abgedampft ist, ist eine Veränderung am Tape nicht mehr sichtbar. Wenn das Tape nun mit Wasser aktiviert wird, kann beobachtet werden, wie die vorbehandelten Bereiche die Feuchtigkeit stärker anzeigen.

Die Kältemittel aus \ref{sec:Mess} haben ebenfalls das Tape temporär verfärbt. Hier wurde keine Veränderung der Wasseranzeige beobachtet.

Bei Wärmeeinwirkung einer Heissluftpistole hat sich zum einen der Klebstoff gelöst und das weisse Papier des Tapes wurde braun. Die Bereiche des Tapes, die noch weiß waren, haben bei Wasser noch immer gut reagiert. Der braune Teil hat kein Wasser mehr anzeigen können.


\subsection{Mögliche Gründe der hohen Varianz}
in der messung in davos \ref{}

In dem Bild \ref{img:Varianz} wurde Schnee in einem 0.2 m^2 Fläche gemessen. Ziel war es die Varianz von vergleichbarem Schnee zu ermittelt.

mit ishiwaw \ref{} wurde das Problem der hohen varianz genauer analysiert.


\begin{figure}
    \centering
    \includegraphics[width=0.8\textwidth]{Bilder/IshikawaDavos.jpg}
    \caption{Ishikawa Fehler Analyse für die Messung des Funktionsmusters 2}
    \label{fig:IshikwaDavos}
\end{figure}



die wichtigsten Einflüsse sind
\begin{itemize}
\item das Tap ist anisotrop, durch die Herstellung. das ist besonders auffällig an den Rändern. die Lösung wurde in \ref{fig:Bildverarbeitnugskonzpet} schon intuitiv benutzt. Der Rand (etwa 2 mm) soll nicht beachtet werden. der bildausschnitt wird in Zukunft so gewählt, dass der Rand nicht im Bild ist.
\item das tap hat einen schlechten kontakt zum wasser auf dem schnee. um diese hypothese zu überprüfen wurde der Winkel eines wassertropfens auf dem tape gemessen. mit einem winkel von 90 grad ist das tap genau zwischen hydrophob und hydrophil.
\item die beleuchtung war nicht homogen. deswegen werden zeit LEDs mit difusoren in dem nächsten \ref{} funktionsmuster verbaut.
\item die tape holder standen nicht genau senkrecht. deswegen wurde die führung mit den magnet bögen und stativmaterial gebaut.
\item die eindrückenergie war inkostistent. deswegen wurden die haupzahl der versuche ohne extra energie durchgeführt. bei dem durchgang mit enerigie wurde das stativmaterial benutzt um eine gleiche Potentiolle energie sicher zu stellen.
\item der tapeholder und nicht das tape hatten kontakt zum schnee. deswegen wurde der neue tapeholder umkonstruiert sodass 40 mm nur der XPS schaumstoff mit dem tape in den schnee eindringen kann.
\item der XPS schaumstoff ist nicht flach. deswegen wurde eine schneidlehre gebaut um den XPS senkrecht  zu schneiden. weiter möglichkeiten wären eine Glassplatte (mikroskop objektivtrager) zwischen den XPS und das Tape zu machen. Oder ein plastisch verformbaren träger für das tape zu entwickeln.
\item die temperatur des tapes war vor dem schneekontakt die umgebungstemeperatur (viel). deswegen wurde jedes tape runtergekülht und mit der wärembildkamera überprüft.
\item die gewichte der tape holders war um rund 10 \% unterschiedlich, denn es wurden verschiedene Versionen benutzt. die neue hat nur eine einzieg version an tape holders.
\item der schnee ist inhomogen. die messung war unter einem baum, von dem schnee und eis runter gefallet war. das hat dazu geführt, dass im schnee centimeter grosse einregionen waren.
\item der schnee ist inhomogen in schichten. im den nächsten messunng wurde ein weniger geschichteter schnee gewählt.
\item der schnee ist inhomogen mit wasser störmen. die messung in davos war in der nähe eines Baches. die nächste messung wurde ein homogener schnee gewählt.
\item die ebende auf der das tape geklebt ist, ist nicht eben, sondern beim transpornt eingedrückt worden. um das problem zu reduzieren wurden Pelican Boxen für den Transport benutzt.
\end{itemize}

weiter mögliche gründe und die sturktiurung der Gründe können im Ishikawa Diadram gesehen werden. \ref{img:ishikawa}


\subsection{Ergebnisse der Versuche}
\textb{erster feldversuch}
\label{ErstFeldVer}
\caption{Wärmebildaufnahme der gekühlten Schneestell. }


\caption{Messstandort in Davos, unter dem Regenschrim ist das Tape gelagert} 

\caption{Ein bild von 'normalem' schnee}


\textb{zweiter Feldversuch}
\label{ZweiFeldVer}


erster feldversuch Ziel: testen des ablaufs der tap, mit drei verschiedenen LWC um den LWC zu beeinflussen, zum einen Wasser über den schnee ausschütten,
dann messen.

zum anderen mit Kältespray den Schnee einfrieren, dann sollte der LWC sehr tief sein. Wärmebildaufnahme der gekühlten Schneestelle.
Der Schnee hat sich nur sehr langsam wieder aufgewärmt, da der Schnee sehr gut isoliert ist.

durchgeführt am 2024-04-11 in davos, einige meter hoch auf der schatten talseite. Messstandort in Davos, unter dem Regenschirm ist das Tape gelagert Ein bild von ’normalem’ schnee schlussfolgerungen: schneedreieck funktioniert bei inhomogenem schnee mit eis nicht. handlich aufwendig, belichtung einseitig, gewichte kommen auf der schablone sich in den weg, zweiter Feldversuch Ziel: neues design. variabler (höherer) anpressdruck, Vergleich mit denothmeter Ich bin sehr zufrieden mit diesen Ergebnissen. Die Technologie des 5999 water indicator tape hat einen TRL von 9 um Qualitätssicherung zu machen. Um den LWC von Schnee zu messen, befand sich das 5999 am Anfang dieser Arbeit bei TRL 1, mit diesen Versuchen hat das 5999 den TRL 5 erreicht.


%\subsection{Vergleich der Ergebnisse mit Denothmeter}
%Das Ziel ist es die Statistik Daten aus \ref{} zu benutzten um ein statischtisches Regressions maschien learn modell auf zu bauen. der Input sind dann parameter wie zum beistiel das RedVsWhite oder der RadiusAvg. der output ist der LWCTape. um das modell zu trainiern wird der LWCDenoth benutzt.

dass heisst im besten fall kann das Tape die gleichen Werte wie das Denothmeter produzieren. aber mit dieser technik ist es nicht möglich eine qualitativ bessere über schnee und Gleitschneelawinen zu machen als das denothmeter.

Um ein modell zu trainiern das den Denothmeter gut abbieldet, muss das modell mit möglichst vielen schneetypen und entsprechenden LWCDenoth trainiert werden.

mein bauchgefühl sagt, dass für ein rubustes modell 100 LWCDenoth gut wären. Das gute an ML ist, dass die genauigkeit der modelle weitgehen linear mit der  anzahl traingsdaten steigt.

\caption{grafik aus der Vorlesung Deep Learning, Hannes Badertsch}

In den frühen phasen diesr BA wird nur das Konzept eingeführt und dan das eigene biologische neuronale Netztwerk im Hirn benutzt um die Bilder/Statisik mit den LWCDenoth zu korrelieren.

das ist die Userstory von XX aus \ref{}


um eine aussage über gleitschneelawinen zu machen die besser ist als die des Denothmeters, müssen die entsprechenden Daten gesammelt werden. dass heisst es braucht eine messkampanie die  kritische Hänge überwacht und dann kann das klassifikations modell trainiert werden, dass eine Serie von Tape Roh daten betrachtet und als output hat: Jetzt wird eine Gleitschneelawine statfinden.

das ist die Userstory von x aus \ref{}


\subsection{Verbesserungsmöglichkeiten des Funktionsmusters}
die geometrie des tapes ist

zwischen dem tape und dem gewicht kann ein flexibler schaumstoff eingesetzt werden, so konnte sich das tape an die unregemasigkeit des schnees besser anpassen.

das tape kann mit einer membran an der schnee gedruckt werden. so kann auf die kraft praktischer kontroliert werden. schwerkraft ist nicht praktisch um eine vollautomatisches produkt zu entwicklen.

die geometrie des harten xps schaums konnte durch eine Kegel ersetzt werden, so kann eine andere andruckkraft erreicht werden.


\subsection{Vollautomatische Durchführung der Messung}
Um die grossen Datenmengen, von über 1000 Datenpunkten, die für ein robustes ML aus \ref{sec:DB} benötigt werden, liefern zu können, muss die Messung teure menschliche Arbeitszeit drastisch reduzieren. Die Vorstudien in \ref{zweitFeldVer} erforderten rund 9 Stunden Arbeitszeit und hätten etwa 50 LWC-Tapes und 6 LWC-Denoth-Datenpunkte liefern können.

Ein grosser Vorteil des Tapes ist die feine örtliche Auflösung im zehntel Millimeter Bereich in der Messregion von 20 x 20 mm. Um diese feine Auflösung zu nutzen, ist es spannend, Messungen durch die Höhe der Schneedecke durchzuführen.

\begin{itemize}
\item Eine Möglichkeit (siehe Abbildung {fig:AutMess}) besteht darin, dass der Feldforscher mit einer Bohrmaschine ein Loch in den Schnee bohrt. Dann kann das Messsystem in das Loch herabgelassen werden und kontinuierlich Messungen durchführen, während es abgesenkt wird. Mit dieser Anordnung wird nicht mehr in der Horizontalen gemessen, sondern in der Vertikalen. Das wird eine umfassendere Aussage über eine Schneedecke liefern.

\item Den Anpressdruck seitlich aus zu üben ist schwierig. Die Schwerkraft funktioniert nicht direkt. Elastomere sind bei tiefen Temperaturen schwer einzuschätzen. Der Einsatz eines Elektromotors ist möglich, jedoch etwas umständlich aufgrund der Notwendigkeit einer Batterie. Eine Blattfeder oder eine Kompressionsfeder stellen vielversprechende Alternativen dar. Ein pneumatisches System bietet in der Wirkung Vorteile, ist jedoch in der Umsetzung anspruchsvoll.



\item Die flache Geometrie des harten XPS-Schaums kann durch eine Kugel oder einen Kegel ersetzt werden, so kann eine andere Druckkraft erreicht werden.


\item Das Tape kann auf eine Membran geklebt werden, die pneumatisch aufgeblasen wird und so an den Schnee angedrückt wird.

\item Zwischen dem Tape und dem Gewicht kann ein flexibler Schaumstoff eingesetzt werden, so kann sich das Tape an die Unregelmässigkeit des Schnees besser anpassen.

\item Um eine hohe Anpassbarkeit des steifen Tapes an den Schnee zu verbessern, kann das Tape in kleinere Stücke geschnitten werden. So kann sich der elastische Träger des Tapes effektiv an den Schnee anpassen.

\item Es ist auch möglich, dass ein vollautomatisiertes Messsystem über den Sommer an strategisch gewählten Orten aufgebaut wird und dann eingeschneit wird. Hier besteht die Schwierigkeit, an genügend ungetesteten guten Schnee zu gelangen, um eine feine zeitliche Auflösung zu ermöglichen. Mit der vierten Iteration konnte gezeigt werden, dass ein Messung an dem selben Schnee wiederholt werden kann.


\item Ein weiteres Konzept ist, dass das Messsystem von einem Helikopter aus abgeworfen wird. Durch die kinetische Energie schlägt das Messsystem dann durch die Schneedecke. In einer zweiten Phase wird das Tape an den Schnee angepresst und die Daten drahtlos an die Datenbank aus Kapitel \ref{sec:DB} übermittelt.

\end{itemize}

\begin{figure}
    \centering
    \includegraphics[width=0.8\textwidth]{Bilder/KonzeptAut.jpeg}
    \caption{Ablauf einer automatischen Messung}
    \label{fig:AutMess}
\end{figure}


\newpage
\section{Fazit}
\input{Ausb.tex}

\subsection{Presönliche Erfahrunng}
Ich hatte viel Freude mich in das mir unbekannte Thema des Schnees und des LWC einzuarbeiten. Die Arbeit hat mir einen kleinen und sehr interesanten Einblick in den Schnee gegeben. Das Thema des Schnees wird für die Schweiz in den kommenden Jahren mit dem Klimawandel noch weiter an Bedeutung gewinnen.

Die agile Hardware Development Methode liegt mir und es war spannend und lehrreich 5 Iterationen des Produktes zu erstellen, in der Hand zu haben, zu testen und zu verbessern.


\subsection{Fazit}
Die Erkenntnisse und Ergebnisse dieser Arbeit bieten eine solide Grundlage für zukünftige Entwicklungen und Forschungen im Bereich der Schneemessung. dass das Konzept vielversprechend ist und weiterhin potenzielle Verbesserungen und Anpassungen ermöglicht.
Entwicklung eines Messablauf mit dem Water indicator tape

verschiedene 5 ausprobierte, 3 erflogreich, 1 weiter entwickelt

nicht nur lwc sonder auch hoffnung auf geometrie des schnees

halbe seite


\section{Ausblick}
methode weiter verfolgen, good stuff


\subsection*{Danksagung}

Ein herzliches Dankeschön geht an Johannes Kuhn, Dr. med. Christine Kuhn, Dr. phil. Wolfgang Menzel, den Werkstattverein Coredump und das Werkzeughaus für ihre Unterstützung. Ich möchte mich auch beim IPEK, insbesondere bei Prof. Dr. Albert Loichinger und Christian Locher, bedanken. Ein besonderer Dank gilt meinen Mitstudierenden Oliver, Florian, Julian und Joel, die stets ein offenes Ohr für mich hatten.


\newpage
% Literaturverzeichnis
%\bibliography{mybibliography}
\section*{Literaturverzeichnis}
% Include all entries from the bibliography file
\cite{Abdelaal.2022}
\cite{Adachi.2020}
\cite{AlamShibly.2017}
\cite{Bonnell.2021}
\cite{Donahue.2022}
\cite{Fasani.2023}
\cite{Kinar.2007}
\cite{Koch.2019}
\cite{Koch.2014}
\cite{Lombardo.2023}
\cite{Mavrovic.2020}
\cite{Nowogrodzki.2018}
\cite{Hao.2021}
\cite{PerezDiaz.2017}
\cite{Yamaguchi.2023}
\cite{snowathome,instructables,joule-thomson-wiki,miro,popmech,nasa-snowex-2020,nasa-snow-density,ubc-met-concepts,jstage-snow-density,sciencelearn-snow-density,proquest-snow-density,nasa-snowex-2020-2,mt-density-meter,elibrary,micro-dehumidifier,kennedylabs-download,a2photonicsensors,ilmsens-short-range-radar,sensirion-sht4xa-sensors,ilmsens-impedance-spectroscopy,slf}




% Print the bibliography heading
\printbibheading



% Print the entire bibliography
\printbibliography


\newpage
\section*{Erklärung zur Urheberschaft}


Ich erkläre hiermit, dass ich die vorliegende Arbeit ohne Hilfe Dritter angefertigt habe. Ich habe nur die Hilfsmittel benutzt, die ich angegeben habe. Gedanken, die ich aus fremden Quellen direkt oder indirekt übernommen habe, sind kenntlich gemacht. Die Arbeit wurde bisher keiner anderen Prüfungsbehörde vorgelegt und auch noch nicht veröffentlicht.

KI-Einsatz ohne Kennzeichnungspflicht:

Ich bin mir bewusst, dass die Nutzung maschinell generierter Texte keine Garantie für die Qualität von Inhalten und Text gewährleistet. Ich versichere daher, dass ich die Text generierenden KI-Tools (ChatGPT 3.5 und ChatGPT 4o) lediglich als Hilfsmittel genutzt habe. Die vorliegenden Arbeit habe ich überwiegend selbst gestaltet. Ich verantworte die Übernahme jeglicher von mir verwendeter maschinell generierter Textpassagen selbst. Ich versichere, dass ich keine KI-Schreibwerkzeuge verwendet habe, deren Nutzung der Prüfer / die Prüferin explizit schriftlich ausgeschlossen hat.



Ort/Datum: Rapperswil, 2024.06.18 \\
Unterschrift:\\
Peter Kuhn


\newpage
\listoffigures


\listoftables % Tabellenverzeichnis erstellen
\newpage
\section*{Digitaler Anhang}


\subsection{Endziel des Arbeit}
verringerung des Schadens durch Gleitlawinen


\subsection*{User Storys}
\label{userstoryvoll}

Alice macht an einem hang einen schedeckenanalyse, mit der schaufeln. neben ihrer subkektiven beurteilung tragt sie noch die messwerte der schneefeuchtikeit ein.

bob sitzt an seime comupter und sieht eine warnung aufleuchte. er furt sofort be der ratischen bahn an und kann den zo so stoppen bevor er von der lawine erfasst wird.

celebor macht eine neue simulation um die vorhersage der scheefeuchtikeit asu den meteodaten zu machen. dasu benutzt er die neue trainsdaten die vergangene jaher gemessen werden.

dorothe weiss nicht ob sie auf desem hang enie abfahr wagen soll. sie schaut schneell die scheefeuchtikeit an und kann sicher den hang abfahern.

ernst wirft aus dem helikopter das sesor paket um den hang zu uberwachen. In sechs monaten wird das paket im abgetauten hang wirder eingesammelt.

gido trainriet ein reinforced ML modell auf bildern von schneeflocken. dazu braucht er hochauflosende bilder und den dazugehorigen LWC der probe.


\subsection*{Anforderungen}
Die Methode soll einn anzeige haben, die Feststellen kann wann eine Gleitlawine bevorsteht.

Die Methode soll unabhängig von der Dichte des Schness funktionieren.

die methode soll den messbereich des LWC von 1 \% bis 7 \% abdecken.

die methode soll für einen Hang in der Schweiz einsetzbar sein.


\subsection*{Lebenslauf}
\includepdf{Bilder/lebenslauf-2.pdf}

\subsection*{Code}

\label{sect:code}

\lstinputlisting[language=SQL, basicstyle=\ttfamily\footnotesize, caption={SQL-Code für die Benutzerinitialisierung}, label={code:User}]{UserInizialise.sql}

Die pseudozufällige Passworder sind nicht optimal, besser wäre $SELECT gen_random_uuid();$


\lstinputlisting[language=SQL, basicstyle=\ttfamily\footnotesize, caption={SQL-Code für die DBinitialisierung}, label={code:DBIni}]{DBInizialise.sql}

\lstinputlisting[language=SQL, basicstyle=\ttfamily\footnotesize, caption={SQL-Code für die Views}, label={code:ViewDB}]{ViewsDB.sql}

\lstinputlisting[language=SQL, basicstyle=\ttfamily\footnotesize, caption={SQL-Code für Beispiel Daten}, label={code:ExamData}]{IsertSomeData.sql}

\lstinputlisting[language=Python, basicstyle=\ttfamily\small, caption={Bilderkennung und verarbeitung}, label={code:RaspKam}]{imageToCircle3.py}

\lstinputlisting[language=Python, basicstyle=\ttfamily\small, caption={Bilderkennung und verarbeitung}, label={code:FeldUser}]{splitUpImage.py}



\end{document}
