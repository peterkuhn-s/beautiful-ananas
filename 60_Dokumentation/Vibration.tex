avanode vibriert. wenn kurz vor gleitlawine wird der schnee zur flüssigkeit. der avanode sinkt auf grund der hohen dichte und verändert dabei die position.

vibraNode


Die Form wird von dem AvaNode übernommen. Um eine hohe formfreiheit und eine hohe dichte zu ereichen wird der VibraNode aus Ton gebaut. Der ungebrannte Ton wird durch Epoxy harz und Acryl Farbe vor Wasser geschützt.

der erste test hat nicht funktioniert. Ich stand auf dem schnee, neben dem Virbanode, ich habe rund die vierfache auflagefläche, aber das 60 fache gewicht. das heisst der schnee war ungeeignet und nicht kurz vor einer gleitschneelanwine.

zumindest an der Oberfläche.


mit dem virbanode ist es sicher nicht mögliche den LWC fest zu stellen. auch nachdem der schnee mit wasser übergossen worden ist, ist der VurbaNode nicht eingesunken.

ist der LWC die einscheidende grösse für gleitschneelawinen?
