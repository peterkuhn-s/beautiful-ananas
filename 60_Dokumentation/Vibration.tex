die hypothese ist, dieses ansatztes ist, dass sich der schnee durch mechanische anregung von seinem Festen in den flussigen zustand ubergeht.

Um die Idee zu testen wird vibrierendes objekt mit hoher dichte auf den schnee gelegt, und es wird beobachtet, wie sich das objekt durch den schnee bewegt.


Die Form wird von dem AvaNode übernommen. Um eine hohe formfreiheit und eine hohe dichte zu ereichen wird der VibraNode aus Ton gebaut. Der ungebrannte Ton wird durch Epoxy harz und Acryl Farbe vor Wasser geschützt. Ein Name, analog zum avanode, ist vibraNode.

Die Testergebnisse sich negativ. der VibraNode konnte trotzt seiner Dichte von 1600 kg/m^3 nicht in den schnee eindringen. Der schnee wurde mit flussigem wasser gesattigt, und hat trotztdem seine mechanischen eingenschaften erhalten.

Jetzt stellt sich die frage ob der LWC einen kausalen oder eien korrelativen Zusammenhang mit gleitschneelawinen hat. und wieweit die vorgeschichte des schnees mitbetrachtet werden muss.
