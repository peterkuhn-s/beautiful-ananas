um die grossen datenmengen die für ein robustes ML aus \ref{} benötigt werden liefern zu könne, muss die Messung die teure menschliche Arbeitszeit drastisch reduzieren. Die Vorstudien in \ref{} waren rund 9 h Arbeitszeit und haben rund 50 LWCTape und 6 LWCDenoth geliefert.

Ein grosser vortei des Taps ist die feine ortiliche auflösung von 20x20 mm. Um diese feine auflösung zu nutzte ist es spannend durch die Höhe der schneedecke die messungen durch zu führen.

in den kleinen vorvesuchen und den funktiosmuster ist die idee, dass ein Feldvorscher einen schneegraben gräbt und so zugang an die unterschiedlichen schichten das schnees hat.

das graben eines schneegraben ist aufwendig und sollte daher vermieden werden.

eine mögilchketi sit das der feldvorscher mit einer bohrvmaschiene ein roch in den schnee macht. dann kann der feldvorscher das messsystem in das loch herablassen, das kontinuierlich eine messung durchführt, während es herab gelassen wird.

es ist auch mögilch, dass das Messsystem über den sommer an stategisch gewählten orten aufgebaut wird und es dann eingeschneit wird. hier ist die schwierigkeit an genügent 'ungetestetn' guten schnee dran zu kommen um eine feine zeitliche auflösung zu ermöglichen.

mit einem vorstudie kann überprüft werden wie sich schnee verhält, wenn der schnee ein zweites mal getestet wird, nud wie lange es braucht, bis sich der schnee vom test erholt hat. wenn mit wenig anpressdruck gearbeitet wird kann ich mir vorstellen, dass sich der schnee schnell erholt. das würde diesem Konzept massiv vereinfachen.

ein weiters Konzept ist, dass von einem helikopter aus das messsystem abgeworfen wird. durch die kinetische energie schlägt das messsystem dann durch die schneedecke. in einer zweiten phase wird dann das tap an den schnee angepresst und die daten drothlos an in die Datenbank aus \ref{} eingespeist.

Um den anpressdruck seitlich ausüben zu können funktioniert die schwerkraft nicht mehr. Elastomere sind bei tiefen temperaturen schwer einzuschätzen. ein elektormotor ist möglich, aber etwas mühsam mit der batterie. eine blatfeder, oder kompressionsfeder sind vielversprechende varianten.
