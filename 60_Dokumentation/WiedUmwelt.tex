Um zu verhindern, dass das Tape den Schnee aufschmiltz, wird es mit einem Kältemittel gekühlt. Deswegen ist es wichtig abzuklären wie das Tape auf andere Stoffe reagiert.

Bei einer Vorbehandlung mit einem Lösungsmittel, getestet wurden Isopropanol, Nitroverdünner und Aceton, verfärbt sich das Tape temporär. Nachdem das Lösungsmittel abgedampft ist, ist eine Veränderung am Tape nicht mehr sichtbar. Wenn das Tape nun mit Wasser aktiviert wird, kann beobachtet werden, wie die vorbehandelten Bereiche die Feuchtigkeit stärker anzeigen.

Die Kältemittel aus \ref{sec:Mess} haben ebenfalls das Tape temporär verfärbt. Hier wurde keine Veränderung der Wasseranzeige beobachtet.

Die Reaktionsfähigkeit des Tapes wurde bei -10 Grad getestet. Der Teperatureinsatzbereich ist vom Hersteller als -40 bis 121 Grad Celsius angegeben.

Bei Wärmeeinwirkung einer Heissluftpistole hat sich zum einen der Klebstoff gelöst und das weisse Papier des Tapes wurde braun. Die Bereiche des Tapes, die noch weiß waren, haben bei Wasser noch immer gut reagiert. Der braune Teil hat kein Wasser mehr anzeigen können.
