
 Um zu verhindern, dass das Tape den Schnee anschmilzt und die Werte verfälscht, muss das Tape mit einem Kältemittel gekühlt werden. Daher ist es wichtig abzuklären, wie das Tape auf chemische Stoffe und thermische Veränderungen reagiert.

Chemisch getestet wurden zuerst Isopropanol, Nitroverdünner und Aceton.  Das Tape verfärbte sich temporär. Nachdem das Lösungsmittel abgedampft war, konnten keine Veränderung am Tape mehr festgestellt werden. Wenn das Tape nun jedoch mit Wasser aktiviert wurde, konnte beobachtet werden, dass die vorbehandelten Bereiche stärker auf Feuchtigkeit  reagierten. 


Die Kältemittel aus \ref{sec:Mess} haben ebenfalls das Tape temporär verfärbt. Hier wurde keine Veränderung der Wasseranzeige beobachtet.

Die Reaktionsfähigkeit des Tapes wurde bei -10 Grad getestet.

Zusätlich wurden auch die thermischen Eigenschaften bei Wärmeeinwirkung getestet. Bei Wärmeeinwirkung mit einer Heissluftpistole hat sich der Klebstoff gelöst und das weisse Papier des Tapes wurde braun. Die Bereiche des Tapes, die nach dem Abkühlen noch weiss waren, haben auf Wasser noch immer gut reagiert. Die braun verfärbten Teile konnten kein Wasser mehr anzeigen. Die Hitzeentwicklung beispielsweise durch die Lagerung des Tapes in einem Fahrzeug, das in der Sonne steht, wären aber deutlich niedriger.

Der Temperatureinsatzbereich ist vom Hersteller als - 40 bis 121 Grad Celsius angegeben.
