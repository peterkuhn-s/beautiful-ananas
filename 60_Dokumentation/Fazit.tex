
Die Untersuchung und Messung des Liquid Water Content (LWC) im Schnee ist aufgrund der komplexen Eigenschaften und Inhomogenität des Schnees herausfordernd. Der LWC beeinflusst fast alle Eigenschaften des Schnees und ist daher ein kritischer Parameter. Es gibt zahlreiche Methoden zur Messung des LWC, wobei gängige kommerzielle Produkte häufig die dielektrische Konstante nutzen, um das flüssige Wasser zu messen. 

In dieser Arbeit wurden sechs verschiedene Methoden zur Messung des LWC getestet:

\begin{enumerate}
    \item Phasenübergang ausgelöst durch Vibration
    \item Elektrischer Widerstand
    \item Diffusion von Flüssigkeit
    \item Refraktion eines Lasers
    \item Reflexion eines Lasers
    \item Water Indicator Tape
\end{enumerate}

Das Water Indicator Tape wurde dann mit der Methodik des agilen Hardware Developments zu einem Messsystem entwickelt. Das Tape stammt ursprünglich aus der Qualitätssicherung in der Elektronik und wird verwendet, um das Eindringen von Wasser nachzuweisen. Bei der Messung mit dem Tape gibt es Hinweise, dass das Tape nicht nur den LWC misst, sondern gleichzeitig Informationen über die Geometrie des Schnees liefern kann. Dies eröffnet neue Perspektiven für die Messung und Analyse von Schnee.

Insgesamt hat die Arbeit gezeigt, dass das entwickelte Messsystem eine vielversprechende Methode zur Bestimmung des LWC darstellt. Durch kontinuierliche Verbesserungen und Anpassungen kann es weiter optimiert und verfeinert werden, um die Messgenauigkeit, Messpräzision und Anwendungsbreite zu erhöhen.
