Der weitere Verlauf der Forschung sollte darauf abzielen, die Präzision und Genauigkeit der Schneemessung zu erhöhen sowie die Anwendbarkeit des Systems in verschiedenen Umgebungen zu testen. Dabei sollten auch mögliche Verbesserungen im Hinblick auf die Automatisierung der Messungen und die Reduzierung des Arbeitsaufwands berücksichtigt werden.

Eine entscheidende Fragestellung besteht darin, die beobachtete Varianz in den Messungen zu verstehen. Es ist wichtig zu klären, ob diese Varianz auf Unterschiede im Liquid Water Content (LWC) des Schnees zurückzuführen ist oder ob sie einen Effekt der Messung selbst darstellt. Um eine statistisch fundierte Aussage treffen zu können, sollten über 30 Messungen von vergleichbaren Schneeproben durchgeführt werden.

Um die statistische Basis der Datenbank zu verbessern, müssen mehr als 1000 Messungen mit dem Tape und etablierten LWC-Messwerten durchgeführt werden. Dadurch können die Daten analysiert, validiert und das Messsystem weiterentwickelt werden.


Der weitere Verlauf der Forschung sollte darauf abzielen, die Präzision und Genauigkeit der Schneemessung zu erhöhen sowie die Anwendbarkeit des Systems in verschiedenen Umgebungen zu testen. Dabei sollten auch mögliche Verbesserungen im Hinblick auf die Automatisierung der Messungen und die Reduzierung des Arbeitsaufwands berücksichtigt werden.

Als Entscheidung soll die beobachtete Variante in den Messungen verstanden werden. ist es eine Variante im LWC des Schnees oder ein Effekt der Messung. Um eine statistisch fundierte Aussage zu machen, sollen über 30 Messungen von vergleichbaren Schnees gemacht werden.

Um die Statistik für die die Grundlage mit der Datenbank gelegt wurde, sind rund 1000 Messungen mit sowohl Tape als auche eine etablieren Messwerte des LWCs nötig.
