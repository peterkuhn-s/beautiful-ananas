


\textbf{Ausgangslage und Aufgabe}

In der Lawinenforschung bestehen offene Fragen hinsichtlich der Erfassung der Schneefeuchte in der Schneedecke. Diese ist einer der wesentlichen Einflüsse in der Schneedecke, die zu Gleitschneelawinen führen. Umfassende Untersuchungen am SLF (Fees et al) haben gezeigt, dass Feuchte, die aus dem Boden dem Schnee oder dem Niederschlag eindringen und sich an den Schneekristallen ablagern, im Bereich von 0.5 bis 10 \% Wassergehalt im Schnee einen erheblichen Einfluss auf die mechanische Stabilität sowie das Kriechverhalten der Schneedecke besitzen.
Bestehende Messverfahren sind das sog. Denothmeter, das als Gerät vom SLF und der Fa. FPGS Tann SG entwickelt sind. Hier wird komplex eine elektrische Kenngrösse erfasst, die jedoch ebenso stark von der Dichte des Schnees abhängig ist.

\textbf{Zielsetzung / Anforderungen}

Die Aufgabe dieser Arbeit ist es, ein Konzept zu entwickeln, mit dem die Schneefeuchte im Gelände als Sensor und als Laborgerät unabhängig von anderen Grössen wie der Dichte, Temperatur etc gemessen oder charakterisiert wird. Als wesentliche Hürde ist die Qualifikation und Kalibrierung der Messung anzusehen. Hier sind gegebenenfalls mehrere hoch priorisierte Verfahren nebeneinander zu erstellen und zu beurteilen, wie weit diese sich evtl auch in Kombination eignen, um eine Aussage über den Wassergehalt am Schneekristall zu treffen.


Die Arbeit wird in drei Teile gegliedert.

\begin{enumerate}
\item Vorstudie: Unterschiedliche physikalische Prinzipien zur Messung des LWC werden erst theoretisch und dann praktisch miteinander verglichen.

\item Bau des Funktionsmusters:  Hier wird ein vielversprechendes physikalisches Prinzip ausgewählt und ein Funktionsmuster gebaut. Dieser Teil wird nach agiler Hardware Entwicklung mit einem Kanban Board geplant. 

\item Dokumentation der Produktentwicklung. Die Darstellung der iterativen Entwicklung der Funktionsmuster in dieser linearen Dokumentation ist herausfordernd.
\end{enumerate}

